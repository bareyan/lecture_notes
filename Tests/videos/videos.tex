```latex
\begin{document}
\sloppy

\section*{Friedrich Nietzsche - How To Find Your Real Self (Existentialism)}

\subsection*{Introduction}

One of the most profound questions in philosophy and human existence is how to discover our true selves, define our values, and find meaning in our lives. Friedrich Nietzsche, a seminal thinker and precursor to existentialism, offers deep insights into these questions. Known for his unconventional ideas on morality and religion, Nietzsche provided guidelines for individuals to shape their future by taking responsibility for who they are. His teachings, found across several books, have influenced a wide range of people, from psychologists to artists and social revolutionaries. Nietzsche considered finding oneself a fundamental life endeavor. Here are four steps inspired by his philosophy to help you become who you truly want to be.

\section{Don’t follow the herd mentality}

\begin{quote}
    “What we’ve called universal values, what we have called truth, has always only ever been the personal expressions of those who promoted them.” -- Friedrich Nietzsche
\end{quote}

\subsection*{Explanation of Herd Mentality}
Nietzsche argued that rigid societal codes foster a "herd mentality." Similar to animal herds, this mentality prioritizes sameness, comfort, and preservation. Societal "moral codes," fabricated by individuals and imposed on others, aim to control human behavior. While offering protection from extreme actions, these codes limit individuality and creativity. Furthermore, strict, dogmatic judgments can provoke rebellion, leading to extreme antisocial attitudes and actions. If societal rules are overly tyrannical, the resulting rebellion can be equally harsh. Conversely, flexible moral codes allow for change without forceful opposition.

\subsection*{Fear of the Unknown}
In reality, morality is often preached strictly. This stems not only from societal, religious, and educational structures but also from the majority's fear of rejection for standing out. Individuals who appear "too different" are often perceived as terrifying or dangerous, reflecting a deep-seated human fear of the unknown. This dynamic persists throughout history, from Galileo's persecution for unconventional thinking to schoolyard ridicule of different students. The "herd" often consists of individuals who have suppressed their creativity and goals, feeling insecure or threatened by those who exhibit such qualities. They fear change and the realization of their unfulfilled potential.

\subsection*{Overcoming Herd Mentality}
An individualistic and driven person must resist being dragged down by the herd's limitations. Instead, one must forge their own path, leaving the herd behind and shining brightly. Start by questioning and silencing internalized negative voices from parents, teachers, or partners who likely projected their own fears and limitations onto you, attempting to pull you back into the herd.

\subsection*{Importance of Courage}
Being strong-minded and courageous is essential for withstanding the discomfort of stepping outside one's comfort zone, a necessary part of becoming who you truly are.

\section{Embrace the Difficulty of Self-Discovery}

\begin{quote}
    “No price is too high to pay for the privilege of owning yourself.” -- Friedrich Nietzsche
\end{quote}

\subsection*{Suffering as a Part of Life}
Humans instinctively avoid pain and suffering. Modern comforts can make us ungrateful, forgetting that suffering is integral to life. Nietzsche believed spiritual growth occurs only when we willingly face life's challenges.

\subsection*{Schopenhauer as Educator}
In his 1873 essay, “Schopenhauer as Educator,” Nietzsche argued that maximizing potential requires taking a difficult path, often leading to isolation. Being a loner is challenging but is part of the price for self-ownership. To avoid being overwhelmed by the "tribe," one must distance oneself and strive for freedom, even if it brings difficulties. Refuse the easy path; embark on the quest for freedom, however frightening.

\subsection*{Freedom from Needs}
True freedom includes freedom from being driven by physiological and psychological needs. Instead of being controlled by impulses (like passionately complaining), become conscious of them and choose whether to act.

\subsection*{Inner Fight}
While resembling modern self-help, Nietzsche's philosophy focuses on a deeper inner fight. Motivational gurus often aim for worldly success (material, relational), while Nietzsche's struggle is for self-discovery, requiring different sacrifices. For instance, analyzing the root of a desire for confidence might reveal it's merely a wish to impress others or prove something to society. This analysis might lead one to abandon this desire and focus on deeper issues like self-discovery, even if it leads to loneliness.

\subsection*{Challenge Your Demons}
Refusing to compromise can lead to conflict, lifestyle changes, and ending relationships. It involves deep introspection into fears and darkness to rise above them. Break the chains of opinion and fear. Nietzsche encourages challenging your "demons" not to banish them, but to understand the deep meaning behind them.

\subsection*{Experience Life Firsthand}
Get out into the world, experience temptations, but remain fully conscious. Emerge with a distinct strength of character and richer inner nature. Without firsthand, fully aware experience, one cannot claim to have truly lived. The extent of your journey depends on your willingness to pay the price. To achieve self-ownership and avoid a meaningless life, find your inner genius by walking a unique path no one else can walk for you. Finding yourself means finding your uniqueness – the set of values and passions that truly represent you.

\section{Say yes to what gives you meaning}

\begin{quote}
    “He who has a why, can bear almost any how” -- Friedrich Nietzsche
\end{quote}

\subsection*{Meaning in a Secular Society}
Nietzsche proposed saying yes to whatever brings personal meaning. Historically, God provided meaning. However, in an increasingly secular and scientific world, Nietzsche noted that religion can no longer serve this function for many. He worried this could lead to nihilism – apathy and unwillingness to find meaning.

\subsection*{Solutions for Finding Meaning}
Nietzsche offered three potential solutions for individuals to find meaning:
\begin{itemize}
    \item \textbf{Humanities:} Replace religion with philosophy, art, music, literature, theater, etc. The humanities help contextualize suffering and efforts, showing our shared human experiences. They offer insights into common problems. Crucially, view them as tools for living, not just academic subjects – read history for its lessons, watch tragedies to see beauty in sadness, not just for entertainment.
    \item \textbf{Übermensch:} Become an Übermensch (superhuman) who creates their own meaning and values independent of external influences. This individual overcomes the meaning-of-life problem by inventing their own meaning and taking full responsibility. Nietzsche saw figures like Julius Caesar, Napoleon Bonaparte, The Buddha, and Goethe as close approximations. For ordinary people, this involves looking inward, evaluating genuinely held values versus societally imposed ones, and participating in humanity's ongoing psychological evolution.
    \item \textbf{Amor Fati:} Love your fate (\textit{Amor Fati}). This involves accepting everything in your life – good, bad, and ugly – as contributing to who you are now. Embracing failures alongside successes can reignite a love for life and reveal meaning even in dark times.
\end{itemize}

\section{Find your true values}

\begin{quote}
    “What, if some day or night a demon were to steal after you into your loneliest loneliness and say to you: This life as you now live it and have lived it, you will have to live once more and innumerable times more … Would you not throw yourself down and gnash your teeth and curse the demon who spoke thus?” -- Friedrich Nietzsche
\end{quote}

\subsection*{The Greatest Weight}
Nietzsche believed we must create our own values to lift the ‘Greatest Weight’. This metaphor represents the crushing feeling of repeating past mistakes due to unexamined values adopted from the "herd." The concept prompts the question: Is what you are doing truly meaningful, or just an enactment of herd beliefs? If your actions feel worth repeating eternally, you are likely fully individuated and strong enough to lift this weight.

\subsection*{Reevaluating Morals}
If you find yourself repeating mistakes and feeling crushed, you likely haven't reevaluated the morals imposed by your herd and are not yet a fully developed individual. Only by reevaluating your moral landscape can the "greatest weight" be lifted. Consider how often you tolerate negative situations (like gaslighting) or repeat patterns (like choosing partners with identical problems) simply because confronting them seems harder than conforming to societal expectations of being "good." True good lies beyond standard definitions of good and evil.

\subsection*{Self-Imposed Jail}
Most people live in self-imposed jails, constrained by pre-subscribed social beliefs that stifle the wildness and individualism of the human spirit. Many submit to the comfort of this cage. Nietzsche recognized we can escape this enclosure of forced beliefs and awaken to personal values.

\subsection*{Breaking Free}
To create your own values and meaning, undergo a transformation – a rebellious phase. Have the courage to break the chains of tradition, religion, and society, potentially distancing yourself from certain people. This break doesn't need to be violent; it can be smart, calm, but definitive.
Actionable steps include:
\begin{itemize}
    \item Make a list of everything and everyone you perceive as limiting your freedom (e.g., workplace rules, controlling partners, critical friends/family).
    \item Devise strategies for change: discuss issues with your spouse, address problems with colleagues/supervisors at work, find friends who appreciate you better.
\end{itemize}
When feeling crushed by the "greatest weight," don't hide your aspirations. Break out of your self-made, herd-based prison and pursue the dreams that give your life meaning. Nietzsche asks: “in every little thing ask yourself, do you desire this once more and innumerable times over?” If not, change yourself and reevaluate your values. Only then can the weight be lifted.

\subsection*{Conclusion}
Finding yourself, according to Nietzsche, involves rejecting the herd mentality, embracing the inherent difficulties of self-discovery, affirming what gives your life personal meaning, and forging your own values independent of societal dictates. This path requires courage and introspection but leads to authentic selfhood and a life imbued with purpose. It's about breaking free from constraints, taking responsibility for your existence, and ultimately, becoming who you truly are.

\end{document}
```