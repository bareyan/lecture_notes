```latex
\begin{document}
\sloppy

\section{Image Réciproque d'un Singleton par une Application Continue}

Soit $f$ une application de $\mathbb{R}$ dans $\mathbb{R}$. L'image réciproque de $\{0\}$ par $f$ est définie par :
\[
f^{-1}(\{0\}) = \{ x \in \mathbb{R} : f(x) = 0 \}
\]

Si $f$ est une application continue, alors $f^{-1}(\{0\})$ est un fermé dans $\mathbb{R}$, car $\{0\}$ est un fermé dans $\mathbb{R}$ et l'image réciproque d'un fermé par une application continue est un fermé.

\section{Adhérence et Densité}

Comme $f^{-1}(\{0\})$ est fermé (sous l'hypothèse que $f$ est continue), on a :
\[
\text{Adh}(f^{-1}(\{0\})) = f^{-1}(\{0\})
\]
où $\text{Adh}(A)$ désigne l'adhérence de l'ensemble $A$.

Pour que $f^{-1}(\{0\})$ soit dense dans $\mathbb{R}$, il faut par définition que son adhérence soit $\mathbb{R}$ tout entier :
\[
\text{Adh}(f^{-1}(\{0\})) = \mathbb{R}
\]
Ce qui implique $\mathbb{R} \subseteq \text{Adh}(f^{-1}(\{0\}))$.

\section{Conséquence de la Densité}

Si l'on suppose que $f$ est continue et que $f^{-1}(\{0\})$ est dense dans $\mathbb{R}$, alors nous avons deux conditions :
\begin{enumerate}
    \item $\text{Adh}(f^{-1}(\{0\})) = f^{-1}(\{0\})$ (car $f$ continue $\implies f^{-1}(\{0\})$ fermé)
    \item $\text{Adh}(f^{-1}(\{0\})) = \mathbb{R}$ (car $f^{-1}(\{0\})$ dense)
\end{enumerate}
En combinant ces deux égalités, on obtient :
\[
f^{-1}(\{0\}) = \mathbb{R}
\]
Ceci est équivalent à dire que pour tout $x \in \mathbb{R}$, $f(x) = 0$.

\subsection{Conclusion}
Alors, si $f$ est une application continue de $\mathbb{R}$ dans $\mathbb{R}$ et si l'ensemble de ses zéros $f^{-1}(\{0\})$ est dense dans $\mathbb{R}$, cela implique nécessairement que $f$ est l'application nulle :
\[
f^{-1}(\{0\}) \text{ est dense dans } \mathbb{R} \implies f^{-1}(\{0\}) = \mathbb{R} \iff \forall x \in \mathbb{R}, f(x) = 0.
\]

\end{document}
```