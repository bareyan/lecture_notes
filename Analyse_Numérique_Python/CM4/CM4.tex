```latex
\documentclass{article}
\usepackage{amssymb,amsmath,amsthm}
\usepackage{graphicx}
\usepackage{color}
\usepackage{float}
\usepackage{fancyhdr}
\usepackage{array}
\usepackage{listings}

\newtheorem{theorem}{Theorem}
\newtheorem{lemma}{Lemma}
\newtheorem{proposition}[theorem]{Proposition}
\newtheorem{definition}{Definition}
\newtheorem{remark}{Remark}
\newtheorem{solution}{Solution}
\newtheorem{example}{Example}

\usepackage[margin=1in]{geometry}

\begin{document}
\sloppy

\section{Introduction à l'interpolation}

\subsection{Définition}

\begin{definition}
Soit un nuage de points (exemple: un ensemble discret de points du graphe d'une fonction). Interpréter ce nuage de points correspond à chercher un polynôme de degré $N-1$ qui passe par chacun de ces points.
\end{definition}

\begin{itemize}
    \item Comment le construire ?
    \item $P_{N-1} \in \mathbb{P}_{N-1}[x]$
    \item $P_{N-1}(x_i) = y_i$
\end{itemize}

\begin{figure}[h]
    \centering
    \includegraphics[width=0.5\linewidth]{interpolation_graph.png}
    \caption{Illustration de l'interpolation polynomiale.}
    \label{fig:interpolation_graph}
\end{figure}

\begin{verbatim}
```python
#save_to: interpolation_graph.png
import matplotlib.pyplot as plt
import matplotlib.patches as patches
import numpy as np

x = np.array([0, 1, 2, 3, 4, 5])
y = np.array([1, 2, 1.5, 3, 2.5, 4])

plt.figure(figsize=(8, 6))
plt.plot(x, y, 'o', label='Points de données')

# Polynomial Interpolation
poly_coeffs = np.polyfit(x, y, len(x) - 1) # Fit a polynomial of degree N-1
poly = np.poly1d(poly_coeffs)
x_interp = np.linspace(0, 5, 100)
y_interp = poly(x_interp)
plt.plot(x_interp, y_interp, '-', label='Interpolation polynomiale (degré N-1)')

plt.xlabel('x')
plt.ylabel('y')
plt.title('Interpolation Polynomiale (Exemple)')
plt.legend()
plt.grid(True)
plt.savefig('interpolation_graph.png')
plt.close()
```
\end{verbatim}


\subsection{Motivations}

\begin{itemize}
    \item La solution d'un problème est fournie par une formule représentative : noyau de la chaleur (ex: convolution) et on cherche la solution en un nombre de points.
    $\implies$ on approche alors la fonction par un polynôme i.e. chercher le polynôme de degré ``bas'' proche de la fonction.
    \item La solution d'un problème n'est connue qu'à travers ses valeurs en un nombre fini de points et on souhaite l'évaluer partout.
    $\implies$ Interpolation.
    \item On peut utiliser l'interpolation dans :
    \begin{itemize}
        \item la résolution numérique
        \item la résolution numérique des Équations Différentielles Ordinaires (EDO)
        \item la visualisation scientifique
    \end{itemize}
\end{itemize}

\begin{definition}
Un tel polynôme est appelé \textbf{polynôme interpolateur de Lagrange} de degré $N-1$ de ces points.
\end{definition}

\subsection{Exemples d'interpolation}

\subsubsection{Théorème: Polynôme interpolateur de degré 1}

\begin{theorem}
Soient $(x_1, y_1)$ et $(x_2, y_2)$ deux points distincts de $\mathbb{R}^2$.
Il existe une unique droite $D$ passant par ces deux points.
\[
(x, y) \in D \iff (x-x_1)(y_2-y_1) - (y-y_1)(x_2-x_1) = 0
\]
Si de plus, $x_1 \neq x_2$, il existe un unique polynôme de degré 1 (i.e., $P \in \mathbb{P}_1[x]$) tel que $y = P(x)$.

avec
\[
P_1(x) = \frac{(x-x_2)}{(x_1-x_2)} y_1 + \frac{(x-x_1)}{(x_2-x_1)} y_2
\]
pour des abscisses $x_1, x_2$ distinctes.
\end{theorem}

\begin{example}
Montrons que $M(x,y)$ est sur la droite $(M_1M_2)$ si et seulement si les vecteurs $\overrightarrow{M_1M}$ et $\overrightarrow{M_1M_2}$ sont colinéaires.
Soient $M_1(x_1, y_1)$, $M_2(x_2, y_2)$ et $M(x,y)$.
\[
M \in (M_1M_2) \Leftrightarrow \overrightarrow{M_1M} // \overrightarrow{M_1M_2}
\]
\[
\Leftrightarrow \det(\overrightarrow{M_1M}, \overrightarrow{M_1M_2}) = 0
\]
\[
\Leftrightarrow \begin{vmatrix} x-x_1 & x_2-x_1 \\ y-y_1 & y_2-y_1 \end{vmatrix} = 0
\]
\[
\Leftrightarrow (x-x_1)(y_2-y_1) - (x_2-x_1)(y-y_1) = 0
\]
Si $x_1 \neq x_2$, alors on peut réécrire l'équation de la droite sous la forme $y=ax+b$:
\[
\implies y-y_1 = \frac{(y_2-y_1)}{(x_2-x_1)} (x-x_1)
\]
\[
\Leftrightarrow y = P_1(x)
\]
\end{example}

\begin{remark}
On a l'écriture équivalente de $P_1$ :
\[
P_1(x) = \frac{x-x_2}{x_1-x_2} y_1 + \frac{x-x_1}{x_2-x_1} y_2 = \frac{y_1-y_2}{x_1-x_2} x + \frac{x_2y_1-x_1y_2}{x_2-x_1} = a_1 x + a_0
\]
c'est l'écriture dans la base $(1, x)$ de $\mathbb{P}_1[x]$ (base canonique).
\end{remark}

\begin{itemize}
    \item $P_1(x) = \frac{x-x_2}{x_1-x_2} y_1 + \frac{x-x_1}{x_2-x_1} y_2$
    \[
    = \underbrace{\frac{x-x_2}{x_1-x_2}}_{\ell_1(x)} y_1 + \underbrace{\frac{x-x_1}{x_2-x_1}}_{\ell_2(x)} y_2
    \]
    C'est l'écriture dans la base $(\ell_1, \ell_2)$ de $\mathbb{P}_1[x]$ (base de Lagrange).
\end{itemize}

\begin{remark}
$\ell_1(x_1) = 1, \ell_1(x_2) = 0$

$\ell_2(x_1) = 0, \ell_2(x_2) = 1$
\end{remark}

\begin{itemize}
    \item $P_1(x) = y_1 + \frac{y_2-y_1}{x_2-x_1} (x-x_1)$
    C'est l'écriture dans la base $(1, x-x_1)$ de $\mathbb{P}_1[x]$ (base de Newton).
\end{itemize}


\subsubsection{Exemple: méthode de calcul employée}

Chercher le polynôme interpolateur de Lagrange aux points $(x_1, y_1), (x_2, y_2), (x_3, y_3)$.

\paragraph{Méthode 1.} $(x_1 \neq x_2, x_2 \neq x_3, x_1 \neq x_3)$

$P_2$ sera un polynôme de degré 2 :
\[
P_2(x) = a_0 + a_1 x + a_2 x^2
\]
Comme $P_2(x_i) = y_i$, pour $i=1, 2, 3$, on a le système d'équations linéaires:
\[
\begin{cases}
P_2(x_1) = y_1 \\
P_2(x_2) = y_2 \\
P_2(x_3) = y_3
\end{cases}
\implies
\begin{cases}
a_0 + a_1 x_1 + a_2 x_1^2 = y_1 \\
a_0 + a_1 x_2 + a_2 x_2^2 = y_2 \\
a_0 + a_1 x_3 + a_2 x_3^2 = y_3
\end{cases}
\]
Matriciellement :
\[
\begin{pmatrix}
1 & x_1 & x_1^2 \\
1 & x_2 & x_2^2 \\
1 & x_3 & x_3^2
\end{pmatrix}
\begin{pmatrix} a_0 \\ a_1 \\ a_2 \end{pmatrix} = \begin{pmatrix} y_1 \\ y_2 \\ y_3 \end{pmatrix}
\implies
\begin{pmatrix} a_0 \\ a_1 \\ a_2 \end{pmatrix} = \underbrace{ \begin{pmatrix}
1 & x_1 & x_1^2 \\
1 & x_2 & x_2^2 \\
1 & x_3 & x_3^2
\end{pmatrix}^{-1} }_{H^{-1}} \begin{pmatrix} y_1 \\ y_2 \\ y_3 \end{pmatrix}
\]
Matrice de Vandermonde mal-conditionnée mais facile à construire.

\begin{remark}
Pour 2 points :
\[
H = \begin{pmatrix} 1 & x_1 \\ 1 & x_2 \end{pmatrix}
\implies
H^{-1} = \frac{1}{x_2-x_1} \begin{pmatrix} x_2 & -x_1 \\ -1 & 1 \end{pmatrix}
\]
si $x_1 \neq x_2$.
\end{remark}

\paragraph{Méthode 2. Base de Newton}

\[
P_2(x) = a_0 + a_1 (x-x_1) + a_2 (x-x_1)(x-x_2)
\]
\[
\begin{cases}
P_2(x_1) = y_1 \implies a_0 = y_1 \\
P_2(x_2) = y_2 \implies a_0 + a_1 (x_2-x_1) = y_2 \\
P_2(x_3) = y_3 \implies a_0 + a_1 (x_3-x_1) + a_2 (x_3-x_1)(x_3-x_2) = y_3
\end{cases}
\]
\[
\implies
\begin{cases}
a_0 = y_1 \\
a_1 = \frac{y_2-y_1}{x_2-x_1} \\
a_2 = \frac{y_3 - y_1 - \frac{y_2-y_1}{x_2-x_1} (x_3-x_1)}{(x_3-x_1)(x_3-x_2)} = \frac{\frac{y_3-y_1}{x_3-x_1} - \frac{y_2-y_1}{x_2-x_1}}{(x_3-x_2)} = \frac{\frac{y_3-y_1}{x_3-x_1} - \frac{y_2-y_1}{x_2-x_1}}{x_3-x_2}
\end{cases}
\]
Cette construction est différentielle et facile à mettre à jour quand on rajoute un point supplémentaire (on rajoute uniquement une ligne).

\textbf{On a donc :}
\[
a_0 = y_1, \quad a_1 = \frac{y_2-y_1}{x_2-x_1}, \quad a_2 = \frac{\frac{y_3-y_1}{x_3-x_1} - \frac{y_2-y_1}{x_2-x_1}}{x_3-x_2}
\]

Le polynôme $P_2$ s'écrit donc :
\[
P_2(x) = y_1 + \frac{y_2-y_1}{x_2-x_1} (x-x_1) + \frac{\frac{y_3-y_1}{x_3-x_1} - \frac{y_2-y_1}{x_2-x_1}}{x_3-x_2} (x-x_1)(x-x_2)
\]

\begin{table}[H]
    \centering
    \begin{tabular}{|c|c|c|c|}
        \hline
         & $a_0$ & $a_1$ & $a_2$ \\
        \hline
        $x_1$ & $y_1$ & &  \\
        \hline
        $x_2$ & $y_1$ & $\frac{y_2-y_1}{x_2-x_1}$ &  \\
        \hline
        $x_3$ & $y_1$ & $\frac{y_2-y_1}{x_2-x_1}$ & $\frac{\frac{y_3-y_1}{x_3-x_1} - \frac{y_2-y_1}{x_2-x_1}}{x_3-x_2}$ \\
        \hline
    \end{tabular}
    \caption{Tableau des coefficients pour la base de Newton.}
    \label{tab:newton_coeffs}
\end{table}
Construction facile et différentielle par différences divisées : ajout d'un terme.

\paragraph{Méthode 3. Base de Lagrange}

\[
P_2(x) = \frac{(x-x_2)(x-x_3)}{(x_1-x_2)(x_1-x_3)} y_1 + \frac{(x-x_1)(x-x_3)}{(x_2-x_1)(x_2-x_3)} y_2 + \frac{(x-x_1)(x-x_2)}{(x_3-x_1)(x_3-x_2)} y_3
\]
\[
P_2(x) = \sum_{i=1}^3 \left( \prod_{\substack{j=1 \\ j \neq i}}^3 \frac{x-x_j}{x_i-x_j} \right) y_i
\]
\[
= \ell_1(x) y_1 + \ell_2(x) y_2 + \ell_3(x) y_3
\]

\begin{remark}
Pour deux points $(x_1, y_1)$ et $(x_2, y_2)$, le polynôme interpolateur de Lagrange de degré 1 est :
\[
P_1(x) = \frac{x-x_2}{x_1-x_2} y_1 + \frac{x-x_1}{x_2-x_1} y_2
\]
\end{remark}

\section{Polynôme interpolateur de Lagrange}

\subsection{Définitions et propriétés}

\subsubsection{Théorème: Existence et unicité}

\begin{theorem}[Existence et unicité]
Soient $x_1, \dots, x_n$ des réels deux à deux distincts et $y_1, \dots, y_n$ des réels quelconques.
Il existe un unique polynôme $P \in \mathbb{P}_{n-1}[x]$ (i.e. de degré au plus $n-1$) tel que $P(x_i) = y_i, \forall i = 1, \dots, n$.

On dit que $P$ est le \textbf{polynôme interpolateur de Lagrange} aux points $(x_1, y_1), \dots, (x_n, y_n)$.
\end{theorem}

\textbf{Preuve:}
Soit l'application linéaire $\Phi : \mathbb{P}_{n-1}[x] \to \mathbb{R}^n$ définie par
\[
P \mapsto \begin{pmatrix} P(x_1) \\ \vdots \\ P(x_n) \end{pmatrix}
\]
Montrons que $\Phi$ est injective.
Si $\Phi(P) = 0$, alors $P(x_i) = 0$ pour tout $i=1, \dots, n$. Donc $P$ a $n$ racines distinctes $x_1, \dots, x_n$.
Comme $P$ est un polynôme de degré au plus $n-1$ avec $n$ racines, il s'ensuit que $P \equiv 0$.
Donc $\Phi$ est injective.

Comme $\mathbb{P}_{n-1}[x]$ et $\mathbb{R}^n$ sont deux espaces vectoriels de même dimension $n$, une application linéaire injective est aussi bijective, donc un isomorphisme d'espaces vectoriels. La bijectivité de $\Phi$ assure l'existence et l'unicité du polynôme interpolateur.

\begin{definition}
Si $f$ est une fonction continue sur $[a,b] \to \mathbb{R}$, et $x_1, \dots, x_n \in [a,b]$ sont $n$ points deux à deux distincts, alors l'unique polynôme $P \in \mathbb{P}_{n-1}[x]$ tel que $P(x_i) = f(x_i)$, pour $i=1, \dots, n$ est appelé \textbf{polynôme d'interpolation de Lagrange} de $f$ aux points $x_1, \dots, x_n$.
\end{definition}

\subsection{Estimation de l'erreur d'interpolation}

\subsubsection{Théorème: Erreur d'interpolation}

\begin{theorem}[Erreur d'interpolation]
Soient $a < b$, $f: [a,b] \to \mathbb{R}$ une fonction continue, et $x_1, \dots, x_n$ $n$ points deux à deux distincts dans $[a,b]$.
Soit $P_n$ le polynôme d'interpolation de Lagrange de $f$ aux points $x_i$.
Si $f$ est de classe $\mathcal{C}^n$ sur $[a,b]$, alors pour tout $x \in [a,b]$, il existe $\xi \in [a,b]$ tel que :
\[
f(x) - P_n(x) = \frac{f^{(n)}(\xi)}{n!} \underbrace{\omega_n(x)}_{= \prod_{i=1}^n (x-x_i)}
\]
où $\omega_n(x) = (x-x_1) \cdots (x-x_n)$.
\end{theorem}

\textbf{Corollaire:}
Si $|f^{(n)}(x)|$ est bornée par $M$ sur $[a,b]$ pour tout $x \in [a,b]$, alors $\forall x \in [a,b]$,
\[
|f(x) - P_n(x)| \le \frac{M}{n!} |\omega_n(x)| \le \frac{M}{n!} (b-a)^n
\]

\textbf{Preuve: (à faire)}

\subsection{Implémentation avec Python}

\begin{lstlisting}[language=Python]
from scipy.interpolate import lagrange

x = np.array([1, 2, 3]) #à remplacer par les valeurs de x_1, x_2, x_3
y = np.array([2, 3, 1]) #à remplacer par les valeurs de y_1, y_2, y_3
p = lagrange(x,y)
print(p) # affiche le polynôme
print(p(2.5)) # évalue le polynôme en x=2.5
\end{lstlisting}

\section{Construction des polynômes d'interpolation de Lagrange}

\subsection{Interpolation dans la base canonique (Vandermonde)}

\subsubsection{Construction}

Soit $P(x) = \sum_{i=0}^{n-1} a_i x^i \in \mathbb{P}_{n-1}[x]$.
On cherche les coefficients $a_0, \dots, a_{n-1}$ tels que $P(x_k) = y_k$ pour $k=1, \dots, n$.
\[
\sum_{i=0}^{n-1} a_i x_k^i = y_k, \quad k=1, \dots, n
\]
Ce qui conduit au système linéaire matriciel suivant :
\[
\begin{pmatrix}
1 & x_1 & x_1^2 & \cdots & x_1^{n-1} \\
1 & x_2 & x_2^2 & \cdots & x_2^{n-1} \\
\vdots & \vdots & \vdots & \ddots & \vdots \\
1 & x_n & x_n^2 & \cdots & x_n^{n-1}
\end{pmatrix}
\begin{pmatrix} a_0 \\ a_1 \\ \vdots \\ a_{n-1} \end{pmatrix} = \begin{pmatrix} y_1 \\ y_2 \\ \vdots \\ y_n \end{pmatrix}
\]
\[
V(x_1, \dots, x_n) \mathbf{a} = \mathbf{y}
\]
où $V(x_1, \dots, x_n)$ est la matrice de Vandermonde.

C'est une matrice pleine, souvent mal conditionnée, mais facile à construire.

\begin{lstlisting}[language=Python]
def VDM_Mat(x):
    n = len(x)
    V = np.zeros((n, n))
    for i in range(n):
        for j in range(n):
            V[i, j] = x[i]**j
    return V

def VDM_Poly(x,y):
    M = VDM_Mat(x)
    a = np.linalg.solve(M, y)
    return a
\end{lstlisting}

\subsection{Evaluation efficace : Algorithme de Horner}

\subsubsection{Proposition: Algorithme de Horner}

\begin{proposition}
Soit $P(x) = a_0 x^n + a_1 x^{n-1} + \dots + a_{n-1} x + a_n$ un polynôme.
On définit la suite $(q_k)_{k=0}^n$ par :
\[
\begin{cases}
q_0 = a_0 \\
q_k = q_{k-1} x + a_k, \quad k=1, \dots, n
\end{cases}
\]
Alors $q_n = P(x)$.
\end{proposition}

\textbf{Exemple: } $P(x) = x^2 + 2x + 1 = (x+1)^2$

Pour évaluer $P(2)$:
$q_0 = a_0 = 1$
$q_1 = q_0 \times 2 + a_1 = 1 \times 2 + 2 = 4$
$q_2 = q_1 \times 2 + a_2 = 4 \times 2 + 1 = 9 = P(2) = 2^2 + 2 \times 2 + 1 = 9$

\begin{lstlisting}[language=Python]
def Horner(P, xx):
    y = 0
    for a in P:
        y = y*xx + a
    return y
\end{lstlisting}

\begin{lstlisting}[language=Python]
def IntVal_VDM (x, y, xx):
    a = VDM_Poly(x,y)
    YY = Horner(a[::-1], xx) # reverse a pour correspondre à l'ordre des coefficients dans Horner
    return YY
\end{lstlisting}


\subsection{Interpolation dans la base duale: Formule de Lagrange et points barycentriques}

\subsubsection{Construction}

L'idée est de prendre pour base de $\mathbb{P}_{n-1}[x]$ l'image réciproque de la base canonique de $\mathbb{R}^n$ par l'application $\Phi$ définie dans le théorème d'existence et unicité.
On cherche donc une base $\{\mathcal{L}_j\}_{j=1}^n$ de $\mathbb{P}_{n-1}[x]$ telle que
\[
\mathcal{L}_j(x_i) = \begin{cases} 1 & \text{si } i = j \\ 0 & \text{si } i \neq j \end{cases}
\]
On construit les polynômes de Lagrange $\mathcal{L}_j(x)$ comme suit :
\[
\mathcal{L}_j(x) = \prod_{\substack{i=1 \\ i \neq j}}^n \frac{x-x_i}{x_j-x_i} = \frac{\prod_{i \neq j} (x-x_i)}{\prod_{i \neq j} (x_j-x_i)}
\]
Le polynôme interpolateur de Lagrange s'écrit alors :
\[
P(x) = \sum_{j=1}^n y_j \mathcal{L}_j(x)
\]

\end{document}
```