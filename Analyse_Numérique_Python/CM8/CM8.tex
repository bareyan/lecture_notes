\begin{document}
\sloppy

\section{Introduction aux Méthodes de Newton-Cotes}
Les méthodes de Newton-Cotes sont une famille de formules de quadrature numérique (intégration numérique) basées sur l'évaluation de l'intégrande en des points équidistants. Elles constituent une généralisation des méthodes élémentaires d'intégration.

\begin{definition}[Méthode de Newton-Cotes]
On appelle méthode de Newton-Cotes d'ordre K la méthode élémentaire consistant à utiliser le polynôme d'interpolation d'ordre K, $P_K(x)$, associé aux K+1 points $x_i$ équidistants :
\[ x_i = \alpha + i \frac{\beta-\alpha}{K}, \quad i=0, \dots, K \]
L'intégrale de la fonction $f(x)$ est alors approchée par l'intégrale de ce polynôme :
\[ \int_{\alpha}^{\beta} f(x) dx \approx \int_{\alpha}^{\beta} P_K(x) dx = \sum_{i=0}^{K} \omega_i f(x_i) \]
où les poids $\omega_i$ sont donnés par :
\[ \omega_i = \int_{\alpha}^{\beta} L_i(x) dx = \int_{\alpha}^{\beta} \prod_{\substack{j=0 \\ j \neq i}}^{K} \frac{x-x_j}{x_i-x_j} dx \]
$L_i(x)$ sont les polynômes de base de Lagrange.
\end{definition}

\section{Exemples de Formules et Cas Particuliers}
L'ordre de la formule de Newton-Cotes est $p$. Cette formule est d'ordre $K$ si $K$ est impair, et d'ordre $K+1$ si $K$ est pair. On n'utilise ces méthodes que pour $K$ pair, sauf le cas $K=1$.

\begin{itemize}
    \item Si $K=1$ (2 points), on a la formule des trapèzes.
    \item Si $K=2$ (3 points), on a la formule de Simpson.
    \item Si $K=4$ (5 points), on a la formule de Boole-Villarceau. Par exemple, pour l'intervalle $[0,1]$:
    \[ \int_0^1 f(x) dx \approx \frac{1}{90} \left[ 7f(0) + 32f\left(\frac{1}{4}\right) + 12f\left(\frac{1}{2}\right) + 32f\left(\frac{3}{4}\right) + 7f(1) \right] \]
    (Note: ceci correspond à $h=1/4$, et les coefficients généraux sont $\frac{2h}{45} (7f_0 + 32f_1 + 12f_2 + 32f_3 + 7f_4)$).
    \item Pour $K=6$ (7 points), on a la formule de Hardy.
\end{itemize}
Pour $K \ge 8$, certains poids $\omega_i$ deviennent négatifs, ce qui rend les formules sensibles aux erreurs d'arrondi et peut entraîner une perte de précision.

\section{Théorème et Erreur d'Intégration}
\begin{theorem}
Soient $I(f) = \int_{\alpha}^{\beta} f(x) dx$ l'intégrale exacte et $I_K(f) = \sum_{i=0}^{K} \omega_i f(x_i)$ l'approximation par la méthode de Newton-Cotes. L'erreur d'intégration est $E(f) = I(f) - I_K(f)$.
Supposons que la méthode d'intégration soit d'ordre $p \ge K$. Posons le noyau de Peano $K(t)$ comme suit :
\[ K(t) = E_x((x-t)_+^p) = \int_{\alpha}^{\beta} (x-t)_+^p dx - \sum_{i=0}^{K} \omega_i (x_i-t)_+^p \]
où $(u)_+^p = u^p$ si $u>0$ et $0$ sinon.
Alors, pour toute fonction $f \in C^{p+1}([\alpha, \beta])$, on a :
\[ E(f) = \frac{1}{p!} \int_{\alpha}^{\beta} K(t) f^{(p+1)}(t) dt \]
\end{theorem}

Si $K(t)$ est de signe constant sur $[\alpha, \beta]$ (ce qui est le cas pour les méthodes de Newton-Cotes usuelles lorsque $K$ n'est pas trop grand), alors il existe $c \in [\alpha, \beta]$ tel que :
\[ E(f) = \frac{f^{(p+1)}(c)}{p!} \int_{\alpha}^{\beta} K(t) dt \]
On peut aussi écrire cela en utilisant l'erreur pour la fonction $x \mapsto x^{p+1}$:
\[ E(f) = \frac{f^{(p+1)}(c)}{(p+1)!} E(x \mapsto x^{p+1}) \]
En effet, $E(x \mapsto x^{p+1}) = \int_{\alpha}^{\beta} K(t) \frac{d^{p+1}(t^{p+1})}{dt^{p+1}} \frac{1}{p!} dt = \int_{\alpha}^{\beta} K(t) \frac{(p+1)!}{p!} dt = (p+1) \int_{\alpha}^{\beta} K(t) dt$.
Dans les méthodes de Newton-Cotes (pour K petit), le noyau de Peano $K(t)$ a un signe constant.

\section{Construction de Formules de Quadrature à Points Non Équidistants : Formule de Gauss-Legendre}
On cherche s'il existe un meilleur choix des points $x_1, \dots, x_N$ (non nécessairement équidistants) dans $[\alpha, \beta]$ et des poids $\omega_1, \dots, \omega_N$ pour que la formule de quadrature $\sum_{i=1}^N \omega_i f(x_i)$ soit exacte pour les polynômes de $\mathbb{R}_n[X]$ avec $n$ le plus grand possible. Typiquement, pour $N$ points, on peut atteindre un degré d'exactitude $2N-1$.

\begin{example}[Formule de Gauss-Legendre à 2 points]
Cherchons une formule de la forme $\int_{-1}^{1} f(x) dx \approx \omega_1 f(x_1) + \omega_2 f(x_2)$.
Nous avons 4 inconnues ($x_1, x_2, \omega_1, \omega_2$). Nous imposons donc que la formule soit exacte pour les polynômes $1, x, x^2, x^3$.
\begin{align*} \int_{-1}^{1} 1 \, dx = 2 &= \omega_1 + \omega_2 \\ \int_{-1}^{1} x \, dx = 0 &= \omega_1 x_1 + \omega_2 x_2 \\ \int_{-1}^{1} x^2 \, dx = \frac{2}{3} &= \omega_1 x_1^2 + \omega_2 x_2^2 \\ \int_{-1}^{1} x^3 \, dx = 0 &= \omega_1 x_1^3 + \omega_2 x_2^3 \end{align*}
La résolution de ce système donne :
$x_1 = -1/\sqrt{3}$, $x_2 = 1/\sqrt{3}$ (racines du polynôme de Legendre $P_2(x) = \frac{1}{2}(3x^2-1)$).
$\omega_1 = 1$, $\omega_2 = 1$.
La formule est donc :
\[ \int_{-1}^{1} f(x) dx \approx f(-1/\sqrt{3}) + f(1/\sqrt{3}) \]
Cette formule est exacte pour les polynômes de degré $\le 3$ (car $2N-1 = 2 \times 2 - 1 = 3$).
\end{example}

\begin{example}[Formule de Gauss-Legendre à 3 points]
On cherche $\int_{-1}^{1} f(x) dx \approx \omega_1 f(x_1) + \omega_2 f(x_2) + \omega_3 f(x_3)$.
On impose l'exactitude pour les polynômes $1, x, x^2, x^3, x^4, x^5$ (degré $2N-1 = 2 \times 3 - 1 = 5$).
On obtient que $x_1, x_2, x_3$ sont les racines du polynôme de Legendre de degré 3, $P_3(x) = \frac{1}{2}(5x^3-3x)$:
\[ x_1 = -\sqrt{3/5}, \quad x_2 = 0, \quad x_3 = \sqrt{3/5} \]
Les poids sont :
\[ \omega_1 = 5/9, \quad \omega_2 = 8/9, \quad \omega_3 = 5/9 \]
La formule est :
\[ \int_{-1}^{1} f(x) dx \approx \frac{5}{9} f(-\sqrt{3/5}) + \frac{8}{9} f(0) + \frac{5}{9} f(\sqrt{3/5}) \]
\end{example}

\begin{proposition}
Considérons la formule à $N$ points $\int_{-1}^{1} P(x) dx \approx \sum_{i=1}^{N} \omega_i P(x_i)$ qui est exacte pour les polynômes de degré $\le 2N-1$.
Alors les abscisses $x_1, \dots, x_N$ sont les $N$ racines du polynôme de Legendre de degré $N$, noté $L_N(x)$ (ou $P_N(x)$), défini par la relation de récurrence :
$L_0(x) = 1$, $L_1(x) = x$
\[ L_n(x) = \frac{1}{n} \left[ (2n-1)x L_{n-1}(x) - (n-1)L_{n-2}(x) \right] \quad \text{pour } n \ge 2 \]
Les poids $\omega_i$ sont donnés par :
\[ \omega_i = \int_{-1}^{1} \prod_{\substack{j=1 \\ j \neq i}}^{N} \frac{x-x_j}{x_i-x_j} dx, \quad i=1, \dots, N \]
La formule de quadrature ainsi obtenue est appelée Formule de Gauss-Legendre.
\end{proposition}

\subsection{Méthode de résolution pratique (pour 2 points)}
Pour retrouver la formule à 2 points $\int_{-1}^{1} P(x) dx \approx \omega_1 f(x_1) + \omega_2 f(x_2)$ :
On considère le polynôme $\Pi_2(x) = (x-x_1)(x-x_2)$. Pour que la formule soit de Gauss, ce polynôme doit être orthogonal à tous les polynômes de degré inférieur à 2 sur $[-1,1]$ avec un poids $w(x)=1$.
\begin{enumerate}
    \item Orthogonalité à $P_0(x)=1$:
    \begin{align*} \int_{-1}^{1} (x-x_1)(x-x_2) dx &= 0 \\ \int_{-1}^{1} (x^2 - (x_1+x_2)x + x_1x_2) dx &= 0 \\ \left[ \frac{x^3}{3} - (x_1+x_2)\frac{x^2}{2} + x_1x_2 x \right]_{-1}^{1} &= 0 \\ \frac{2}{3} + 2x_1x_2 &= 0 \Rightarrow x_1x_2 = -1/3 \end{align*}
    \item Orthogonalité à $P_1(x)=x$:
    \begin{align*} \int_{-1}^{1} x(x-x_1)(x-x_2) dx &= 0 \\ \int_{-1}^{1} (x^3 - (x_1+x_2)x^2 + x_1x_2 x) dx &= 0 \\ \left[ \frac{x^4}{4} - (x_1+x_2)\frac{x^3}{3} + x_1x_2 \frac{x^2}{2} \right]_{-1}^{1} &= 0 \\ -\frac{2}{3}(x_1+x_2) &= 0 \Rightarrow x_1+x_2 = 0 \end{align*}
\end{enumerate}
Comme $x_1+x_2=0$ et $x_1x_2=-1/3$, le polynôme $(x-x_1)(x-x_2)$ est $x^2 - (x_1+x_2)x + x_1x_2 = x^2 - 1/3$.
Les racines de $x^2-1/3=0$ sont $x^2=1/3 \Rightarrow x_1 = -1/\sqrt{3}, x_2 = 1/\sqrt{3}$.

Pour trouver les poids $\omega_1, \omega_2$, on utilise l'exactitude de la formule pour des polynômes simples :
\begin{enumerate}
    \item Pour $f(x)=1$:
    \[ \int_{-1}^{1} 1 dx = 2 = \omega_1 + \omega_2 \]
    \item Pour $f(x)=x$:
    \[ \int_{-1}^{1} x dx = 0 = \omega_1(-1/\sqrt{3}) + \omega_2(1/\sqrt{3}) \]
    De la deuxième équation, $0 = (- \omega_1 + \omega_2)/\sqrt{3} \Rightarrow \omega_1 = \omega_2$.
    En substituant dans la première, $2 = \omega_1 + \omega_1 = 2\omega_1 \Rightarrow \omega_1 = 1$.
    Donc $\omega_1 = \omega_2 = 1$.
\end{enumerate}
On retrouve bien la formule de Gauss-Legendre à 2 points.

\end{document}