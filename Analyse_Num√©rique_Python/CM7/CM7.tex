```latex
\documentclass{article}
\usepackage{amssymb,amsmath,amsthm}
\usepackage{graphicx}
\usepackage{color}
\usepackage{float}
\usepackage{fancyhdr}
\usepackage{array}
\usepackage{listings}

\newtheorem{theorem}{Theorem}
\newtheorem{lemma}{Lemma}
\newtheorem{proposition}[theorem]{Proposition}
\newtheorem{definition}{Definition}
\newtheorem{remark}{Remark}
\newtheorem{solution}{Solution}
\newtheorem{example}{Example}

\usepackage[margin=1in]{geometry}

\begin{document}
\sloppy

\section{Introduction}

L'intégration numérique est essentielle pour approximer les intégrales définies, en particulier lorsque le calcul analytique est impossible ou complexe. Les formules de quadrature offrent une approche pour estimer ces intégrales en utilisant une somme pondérée de valeurs de la fonction en des points spécifiques. Ce chapitre introduit les formules de quadrature composites, qui consistent à appliquer des formules de quadrature élémentaires sur des subdivisions de l'intervalle d'intégration pour améliorer la précision de l'approximation. Nous explorerons les concepts de formules de quadrature élémentaires et composites, et nous détaillerons les méthodes classiques telles que les méthodes des rectangles, des trapèzes, du point milieu et de Simpson.

\section{Formules de Quadrature Élémentaires}

\begin{definition}[Formule de Quadrature Élémentaire]
On appelle formule de quadrature élémentaire sur l'intervalle $I_e = [-1, 1]$, associée aux points $x_i \in [-1, 1]$ et aux poids $w_i$, la formule :
\[ I_e(f) = \sum_{i=1}^{N} w_i f(x_i) \]
qui approche l'intégrale $\int_{-1}^{1} f(t) dt$ pour une fonction $f \in C^0([-1, 1])$.
\end{definition}

Plus généralement, pour un intervalle $[a, b]$, on peut définir une formule de quadrature élémentaire.

\begin{definition}
Étant donné un intervalle $[a, b]$, une formule de quadrature élémentaire $I_e(f)$ est une approximation de l'intégrale $\int_{a}^{b} f(t) dt$ de la forme :
\[ I_e(f) = \sum_{i=1}^{N} \omega_i f(\xi_i) \]
où $\xi_i \in [a, b]$ sont les points de quadrature et $\omega_i$ sont les poids associés.
\end{definition}

\section{Formules de Quadrature Composites}

\subsection{Principe de construction}

Pour améliorer la précision de l'approximation, on utilise des formules de quadrature composites. L'idée est de subdiviser l'intervalle d'intégration $[a, b]$ en $n$ sous-intervalles $[x_i, x_{i+1}]$ où $a = x_0 < x_1 < \dots < x_n = b$. Sur chaque sous-intervalle $[x_i, x_{i+1}]$, on applique une formule de quadrature élémentaire. La formule de quadrature composite $I_c(f)$ sur $[a, b]$ est la somme des approximations sur chaque sous-intervalle :
\[ I_c(f) = \sum_{i=0}^{n-1} I_e(f|_{[x_i, x_{i+1}]}) \]
où $I_e(f|_{[x_i, x_{i+1}]})$ est la formule de quadrature élémentaire appliquée à la fonction $f$ sur l'intervalle $[x_i, x_{i+1}]$.

Dans le cas d'une subdivision uniforme de l'intervalle $[a, b]$ en $n$ sous-intervalles, on pose $h = \frac{b-a}{n}$ et $x_i = a + ih$ pour $i = 0, 1, \dots, n$.

\subsection{Visualisation}

La visualisation des formules de quadrature composites permet de comprendre géométriquement l'approximation de l'intégrale par des sommes d'aires de rectangles ou de trapèzes.

\begin{verbatim}

#save_to: quadrature_composite_visualization.png
import matplotlib.pyplot as plt
import matplotlib.patches as patches
import numpy as np

def f(x):
    return np.sin(x) + 1

a = 0
b = 3
n_subintervals = 5
h = (b - a) / n_subintervals
x = np.linspace(a, b, 100)
y = f(x)

fig, ax = plt.subplots(figsize=(8, 5))
ax.plot(x, y, label='$f(x) = \sin(x) + 1$')

# Shade the area under the curve
x_fill = np.linspace(a, b, 100)
y_fill = f(x_fill)
ax.fill_between(x_fill, y_fill, color='lightgray', alpha=0.5, label='Area under curve')

# Composite Quadrature Visualization (Rectangles for example)
for i in range(n_subintervals):
    x_start = a + i * h
    x_end = a + (i + 1) * h
    rect_height = f(x_start) # Left rectangle rule
    rect = patches.Rectangle((x_start, 0), h, rect_height, linewidth=1, edgecolor='blue', facecolor='lightblue', alpha=0.7)
    ax.add_patch(rect)

ax.set_xlim([a - 0.1, b + 0.1])
ax.set_ylim([0, max(y) + 0.5])
ax.set_xlabel('x')
ax.set_ylabel('f(x)')
ax.set_title('Visualisation de la Quadrature Composite')
ax.legend()
ax.grid(True)
plt.savefig('quadrature_composite_visualization.png')

\end{verbatim}

\begin{figure}[h]
    \centering
    \includegraphics[ max width=\textwidth,
     max height=0.4\textheight,
     keepaspectratio]{quadrature_composite_visualization.png}
    \caption{Visualisation d'une formule de quadrature composite (méthode des rectangles gauches).}
    \label{fig:quadrature_composite_visualization}
\end{figure}


Considérons une subdivision de l'intervalle $[a, b]$ en $n$ intervalles $[x_j, x_{j+1}]$ de longueur $h = \frac{b-a}{n}$. La formule de quadrature composite s'écrit alors comme la somme des contributions sur chaque intervalle.

\section{Méthodes de Quadrature Composites Classiques}

\subsection{Méthode des Rectangles}

\subsubsection{Définition}

La méthode des rectangles approche l'intégrale sur chaque sous-intervalle $[x_j, x_{j+1}]$ par l'aire du rectangle de hauteur $f(x_j)$ (rectangle gauche) ou $f(x_{j+1})$ (rectangle droit), ou $f(\frac{x_j+x_{j+1}}{2})$ (point milieu). Pour la méthode des rectangles gauches élémentaire sur un intervalle $[\alpha, \beta]$, on a:
\[ \int_{\alpha}^{\beta} f(t) dt \approx (\beta - \alpha) f(\alpha) \]

\begin{verbatim}

#save_to: methode_rectangles.png
import matplotlib.pyplot as plt
import matplotlib.patches as patches
import numpy as np

def f(x):
    return 0.1 * x**3 - 0.5 * x**2 + x + 2

alpha = 1
beta = 4
x_curve = np.linspace(alpha, beta, 100)
y_curve = f(x_curve)

fig, ax = plt.subplots(figsize=(6, 4))
ax.plot(x_curve, y_curve, label='$f(x)$')

# Rectangle for rectangle method
rect_height = f(alpha)
rect = patches.Rectangle((alpha, 0), (beta - alpha), rect_height, linewidth=1, edgecolor='r', facecolor='red', alpha=0.5, label='Rectangle Gauche')
ax.add_patch(rect)

# Shade the area under the curve
x_fill = np.linspace(alpha, beta, 100)
y_fill = f(x_fill)
ax.fill_between(x_fill, y_fill, color='lightgray', alpha=0.5, label='Area under curve')


ax.set_xlim([alpha - 0.2, beta + 0.2])
ax.set_ylim([0, max(y_curve) + 0.5])
ax.set_xlabel('x')
ax.set_ylabel('f(x)')
ax.set_title('Méthode des Rectangles')
ax.legend()
ax.grid(True)
plt.savefig('methode_rectangles.png')

\end{verbatim}

\begin{figure}[h]
    \centering
    \includegraphics[ max width=\textwidth,
     max height=0.4\textheight,
     keepaspectratio]{methode_rectangles.png}
    \caption{Méthode des Rectangles (gauche).}
    \label{fig:methode_rectangles}
\end{figure}

La formule de quadrature composite des rectangles gauches sur $[a, b]$ avec $n$ intervalles est donnée par:
\[ I_c(f) = h \sum_{j=0}^{n-1} f(x_j) = h \sum_{j=0}^{n-1} f(a + jh) \]
où $h = \frac{b-a}{n}$ et $x_j = a + jh$.

\subsubsection{Erreur et Degré d'exactitude}

L'erreur de la méthode des rectangles est d'ordre $O(h)$. Pour une fonction $f \in C^1([a, b])$, l'erreur $E_c(f) = \int_{a}^{b} f(t) dt - I_c(f)$ de la méthode des rectangles composite est majorée par :
\[ |E_c(f)| \leq h \|f'\|_{\infty,[a,b]} (b-a) = \frac{(b-a)^2}{n} \|f'\|_{\infty,[a,b]} \]
Le degré d'exactitude de la méthode des rectangles élémentaire est 0, car elle est exacte pour les polynômes constants (de degré 0).

\subsection{Méthode des Trapèzes}

\subsubsection{Définition}

La méthode des trapèzes approche l'intégrale sur chaque sous-intervalle $[x_j, x_{j+1}]$ par l'aire du trapèze formé par les points $(x_j, 0), (x_{j+1}, 0), (x_{j+1}, f(x_{j+1})), (x_j, f(x_j))$. Pour la méthode des trapèzes élémentaire sur un intervalle $[\alpha, \beta]$, on a:
\[ \int_{\alpha}^{\beta} f(t) dt \approx \frac{(\beta - \alpha)}{2} [f(\alpha) + f(\beta)] \]

\begin{verbatim}

#save_to: methode_trapezes.png
import matplotlib.pyplot as plt
import matplotlib.patches as patches
import numpy as np

def f(x):
    return 0.1 * x**3 - 0.5 * x**2 + x + 2

alpha = 1
beta = 4
x_curve = np.linspace(alpha, beta, 100)
y_curve = f(x_curve)

fig, ax = plt.subplots(figsize=(6, 4))
ax.plot(x_curve, y_curve, label='$f(x)$')

# Trapezoid for trapezoidal method
trapez_points_x = [alpha, beta, beta, alpha]
trapez_points_y = [0, 0, f(beta), f(alpha)]
trapez = patches.Polygon(np.column_stack([trapez_points_x, trapez_points_y]), closed=True, linewidth=1, edgecolor='g', facecolor='lightgreen', alpha=0.6, label='Trapèze')
ax.add_patch(trapez)


# Shade the area under the curve
x_fill = np.linspace(alpha, beta, 100)
y_fill = f(x_fill)
ax.fill_between(x_fill, y_fill, color='lightgray', alpha=0.5, label='Area under curve')

ax.set_xlim([alpha - 0.2, beta + 0.2])
ax.set_ylim([0, max(y_curve) + 0.5])
ax.set_xlabel('x')
ax.set_ylabel('f(x)')
ax.set_title('Méthode des Trapèzes')
ax.legend()
ax.grid(True)
plt.savefig('methode_trapezes.png')

\end{verbatim}

\begin{figure}[h]
    \centering
    \includegraphics[ max width=\textwidth,
     max height=0.4\textheight,
     keepaspectratio]{methode_trapezes.png}
    \caption{Méthode des Trapèzes.}
    \label{fig:methode_trapezes}
\end{figure}

La formule de quadrature composite des trapèzes sur $[a, b]$ avec $n$ intervalles est donnée par:
\[ I_c(f) = \frac{h}{2} [f(x_0) + 2 \sum_{j=1}^{n-1} f(x_j) + f(x_n)] = \frac{h}{2} [f(a) + 2 \sum_{j=1}^{n-1} f(a + jh) + f(b)] \]
où $h = \frac{b-a}{n}$ et $x_j = a + jh$.

\subsubsection{Erreur et Degré d'exactitude}

L'erreur de la méthode des trapèzes est d'ordre $O(h^2)$. Pour une fonction $f \in C^2([a, b])$, l'erreur $E_c(f) = \int_{a}^{b} f(t) dt - I_c(f)$ de la méthode des trapèzes composite est majorée par :
\[ |E_c(f)| \leq \frac{h^2}{12} \|f''\|_{\infty,[a,b]} (b-a) = \frac{(b-a)^3}{12n^2} \|f''\|_{\infty,[a,b]} \]
Le degré d'exactitude de la méthode des trapèzes élémentaire est 1, car elle est exacte pour les polynômes de degré au plus 1.

\subsection{Méthode du Point Milieu}

\subsubsection{Définition}

La méthode du point milieu approche l'intégrale sur chaque sous-intervalle $[x_j, x_{j+1}]$ par l'aire du rectangle de hauteur $f(\frac{x_j+x_{j+1}}{2})$ au milieu de l'intervalle. Pour la méthode du point milieu élémentaire sur un intervalle $[\alpha, \beta]$, on a:
\[ \int_{\alpha}^{\beta} f(t) dt \approx (\beta - \alpha) f\left(\frac{\alpha + \beta}{2}\right) \]

\begin{verbatim}

#save_to: methode_point_milieu.png
import matplotlib.pyplot as plt
import matplotlib.patches as patches
import numpy as np

def f(x):
    return 0.1 * x**3 - 0.5 * x**2 + x + 2

alpha = 1
beta = 4
x_curve = np.linspace(alpha, beta, 100)
y_curve = f(x_curve)
midpoint = (alpha + beta) / 2

fig, ax = plt.subplots(figsize=(6, 4))
ax.plot(x_curve, y_curve, label='$f(x)$')

# Rectangle for midpoint method
rect_height = f(midpoint)
rect = patches.Rectangle((alpha, 0), (beta - alpha), rect_height, linewidth=1, edgecolor='purple', facecolor='violet', alpha=0.6, label='Rectangle Point Milieu')
ax.add_patch(rect)

# Vertical line at midpoint
plt.plot([midpoint, midpoint], [0, rect_height], color='purple', linestyle='--', linewidth=1, label='Point Milieu')


# Shade the area under the curve
x_fill = np.linspace(alpha, beta, 100)
y_fill = f(x_fill)
ax.fill_between(x_fill, y_fill, color='lightgray', alpha=0.5, label='Area under curve')


ax.set_xlim([alpha - 0.2, beta + 0.2])
ax.set_ylim([0, max(y_curve) + 0.5])
ax.set_xlabel('x')
ax.set_ylabel('f(x)')
ax.set_title('Méthode du Point Milieu')
ax.legend()
ax.grid(True)
plt.savefig('methode_point_milieu.png')

\end{verbatim}

\begin{figure}[h]
    \centering
    \includegraphics[ max width=\textwidth,
     max height=0.4\textheight,
     keepaspectratio]{methode_point_milieu.png}
    \caption{Méthode du Point Milieu.}
    \label{fig:methode_point_milieu}
\end{figure}

La formule de quadrature composite du point milieu sur $[a, b]$ avec $n$ intervalles est donnée par:
\[ I_c(f) = h \sum_{j=0}^{n-1} f\left(\frac{x_j + x_{j+1}}{2}\right) = h \sum_{j=0}^{n-1} f\left(a + (j + \frac{1}{2})h\right) \]
où $h = \frac{b-a}{n}$ et $x_j = a + jh$.

\subsubsection{Exercice : Degré d'exactitude}

Déterminons le degré d'exactitude de la formule du point milieu élémentaire sur $[-1, 1]$. On teste pour les polynômes $f(x) = 1, x, x^2, \dots$.

\begin{itemize}
    \item Pour $f(x) = 1$:
    \[ \int_{-1}^{1} 1 dx = [x]_{-1}^{1} = 2 \]
    \[ I_e(f) = (1 - (-1)) f\left(\frac{-1+1}{2}\right) = 2 \cdot f(0) = 2 \cdot 1 = 2 \]
    Donc, $I_e(f) = \int_{-1}^{1} f(x) dx$.

    \item Pour $f(x) = x$:
    \[ \int_{-1}^{1} x dx = \left[\frac{x^2}{2}\right]_{-1}^{1} = 0 \]
    \[ I_e(f) = 2 \cdot f(0) = 2 \cdot 0 = 0 \]
    Donc, $I_e(f) = \int_{-1}^{1} f(x) dx$.

    \item Pour $f(x) = x^2$:
    \[ \int_{-1}^{1} x^2 dx = \left[\frac{x^3}{3}\right]_{-1}^{1} = \frac{2}{3} \]
    \[ I_e(f) = 2 \cdot f(0) = 2 \cdot 0^2 = 0 \]
    Donc, $I_e(f) \neq \int_{-1}^{1} f(x) dx$.

    \item Pour $f(x) = x^3$:
    \[ \int_{-1}^{1} x^3 dx = \left[\frac{x^4}{4}\right]_{-1}^{1} = 0 \]
    \[ I_e(f) = 2 \cdot f(0) = 2 \cdot 0^3 = 0 \]
    Donc, $I_e(f) = \int_{-1}^{1} f(x) dx$.

    \item Pour $f(x) = x^4$:
    \[ \int_{-1}^{1} x^4 dx = \left[\frac{x^5}{5}\right]_{-1}^{1} = \frac{2}{5} \]
    \[ I_e(f) = 2 \cdot f(0) = 2 \cdot 0^4 = 0 \]
    Donc, $I_e(f) \neq \int_{-1}^{1} f(x) dx$.
\end{itemize}

La formule élémentaire du point milieu est exacte pour les polynômes de degré au moins 1. En fait, elle est exacte pour les polynômes de degré au plus 1. Testons pour un polynôme de degré 2 pour être sûr. On a vu que pour $f(x) = x^2$, la formule n'est pas exacte. Testons pour $f(x) = s^2$:
\[ \int_{-1}^{1} s^2 ds = \frac{2}{3} \]
\[ 2 \cdot f(0) = 2 \cdot 0^2 = 0 \neq \frac{2}{3} \]
Donc la formule élémentaire n'est pas exacte pour les polynômes de degré 2.

**Conclusion :** La formule du point milieu élémentaire est exacte pour les polynômes de degré 1. En fait, par un calcul plus précis et en considérant l'erreur, on montre que la formule du point milieu élémentaire est exacte pour les polynômes de degré au plus 1. On dit que le degré d'exactitude est 1. En réalité, le degré d'exactitude de la méthode du point milieu élémentaire est en fait 1, car elle est exacte pour les polynômes de degré au plus 1. La formule est d'ordre 2, donc l'erreur est en $O(h^2)$ pour les fonctions suffisamment régulières.

En réalité, la formule du point milieu élémentaire est exacte pour les polynômes de degré au plus 1.  La formule du point milieu élémentaire est en fait exacte pour les polynômes de degré au plus 1.  Donc le degré d'exactitude est 1. On dit que la formule du point milieu est d'ordre 2 car l'erreur est en $O(h^2)$.

\subsection{Méthode de Simpson}

\subsubsection{Définition}

La méthode de Simpson approche l'intégrale sur chaque intervalle double $[x_j, x_{j+2}]$ en utilisant un polynôme de degré 2 qui interpole les points $(x_j, f(x_j)), (x_{j+1}, f(x_{j+1})), (x_{j+2}, f(x_{j+2}))$.  Sur un intervalle $[\alpha, \beta]$, en posant $\gamma = \frac{\alpha + \beta}{2}$, la formule de Simpson élémentaire est:
\[ \int_{\alpha}^{\beta} f(x) dx \approx \frac{(\beta - \alpha)}{6} \left[f(\alpha) + 4f\left(\frac{\alpha + \beta}{2}\right) + f(\beta)\right] \]

\begin{verbatim}

#save_to: methode_simpson.png
import matplotlib.pyplot as plt
import matplotlib.patches as patches
import numpy as np

def f(x):
    return 0.1 * x**3 - 0.5 * x**2 + x + 2

alpha = 1
beta = 4
gamma = (alpha + beta) / 2
x_curve = np.linspace(alpha, beta, 100)
y_curve = f(x_curve)

fig, ax = plt.subplots(figsize=(6, 4))
ax.plot(x_curve, y_curve, label='$f(x)$')

# Simpson approximation - using a parabola (not directly drawn as patch but conceptually represented)
x_poly = [alpha, gamma, beta]
y_poly = [f(alpha), f(gamma), f(beta)]
x_interp = np.linspace(alpha, beta, 100)

# Shade under the curve - conceptually representing Simpson's area
x_fill = np.linspace(alpha, beta, 100)
y_fill = f(x_fill)
ax.fill_between(x_fill, y_fill, color='lightgray', alpha=0.5, label='Area under curve')
# Plot interpolation points for Simpson's rule
plt.plot(x_poly, y_poly, marker='o', linestyle='none', color='orange', label='Points for Simpson')


ax.set_xlim([alpha - 0.2, beta + 0.2])
ax.set_ylim([0, max(y_curve) + 0.5])
ax.set_xlabel('x')
ax.set_ylabel('f(x)')
ax.set_title('Méthode de Simpson (Visualisation Conceptuelle)')
ax.legend()
ax.grid(True)
plt.savefig('methode_simpson.png')

\end{verbatim}

\begin{figure}[h]
    \centering
    \includegraphics[ max width=\textwidth,
     max height=0.4\textheight,
     keepaspectratio]{methode_simpson.png}
    \caption{Méthode de Simpson (visualisation conceptuelle montrant l'aire sous la courbe).}
    \label{fig:methode_simpson}
\end{figure}

La formule de quadrature composite de Simpson sur $[a, b]$ avec $n$ intervalles (n pair) est donnée par:
\[
I_c(f) = \frac{h}{3} \left[ f(x_0) + 4 \sum_{j=1}^{n/2} f(x_{2j-1}) + 2 \sum_{j=1}^{n/2-1} f(x_{2j}) + f(x_n) \right]
\]
où $h = \frac{b-a}{n}$ et $x_j = a + jh$.

\subsubsection{Erreur et Degré d'exactitude}

L'erreur de la méthode de Simpson est d'ordre $O(h^4)$. Pour une fonction $f \in C^4([a, b])$, l'erreur $E_c(f) = \int_{a}^{b} f(t) dt - I_c(f)$ de la méthode de Simpson composite est majorée par :
\[ |E_c(f)| \leq \frac{h^4}{2880} \|f^{(4)}\|_{\infty,[a,b]} (b-a) = \frac{(b-a)^5}{2880n^4} \|f^{(4)}\|_{\infty,[a,b]} \]
Le degré d'exactitude de la méthode de Simpson élémentaire est 3, car elle est exacte pour les polynômes de degré au plus 3.


\section{Comparaison des Méthodes}

\begin{proposition}
Si $f \in C^2([a, b])$, l'erreur de quadrature de la méthode composite associée à une subdivision uniforme de pas $h$ est majorée par :
\[ |E_c(f)| = |\int_{a}^{b} f(t) dt - I_c(f)| \leq h^2 \frac{(b-a)}{12} \|f''\|_{\infty,[a,b]} \]
pour la méthode des trapèzes.
\end{proposition}

\begin{proposition}
Si $f \in C^4([a, b])$, l'erreur de quadrature de la méthode composite de Simpson associée à une subdivision uniforme de pas $h$ est majorée par :
\[ |E_c(f)| = |\int_{a}^{b} f(t) dt - I_c(f)| \leq h^4 \frac{(b-a)}{2880} \|f^{(4)}\|_{\infty,[a,b]} \]
\end{proposition}

\begin{table}[H]
    \centering
    \begin{tabular}{|c|c|c|c|}
        \hline
        Méthode & Formule élémentaire (sur $[\alpha, \beta]$) & Ordre de convergence & Degré d'exactitude \\
        \hline
        Rectangles (gauche) & $(\beta - \alpha) f(\alpha)$ & $O(h)$ & 0 \\
        Trapèzes & $\frac{(\beta - \alpha)}{2} [f(\alpha) + f(\beta)]$ & $O(h^2)$ & 1 \\
        Point Milieu & $(\beta - \alpha) f\left(\frac{\alpha + \beta}{2}\right)$ & $O(h^2)$ & 1 \\
        Simpson & $\frac{(\beta - \alpha)}{6} \left[f(\alpha) + 4f\left(\frac{\alpha + \beta}{2}\right) + f(\beta)\right]$ & $O(h^4)$ & 3 \\
        \hline
    \end{tabular}
    \caption{Comparaison des méthodes de quadrature composites classiques.}
    \label{tab:comparaison_methodes}
\end{table}

\section{Conclusion}

Les formules de quadrature composites offrent des outils efficaces pour l'approximation numérique d'intégrales définies. Le choix de la méthode dépend de la régularité de la fonction à intégrer et de la précision souhaitée. La méthode de Simpson, avec son ordre de convergence élevé, est généralement préférée pour les fonctions régulières lorsque la précision est primordiale. Cependant, les méthodes des rectangles et des trapèzes peuvent être suffisantes dans des contextes où une précision moindre est acceptable ou pour des fonctions moins régulières.

\section{Exercices}

\begin{enumerate}
    \item Calculer l'intégrale $\int_{0}^{1} x^2 dx$ en utilisant les méthodes des rectangles gauches, des trapèzes et du point milieu avec $n=4$ sous-intervalles. Comparer les résultats avec la valeur exacte de l'intégrale.
    \item Estimer l'erreur pour chaque méthode utilisée dans l'exercice précédent en utilisant les bornes d'erreur théoriques.
    \item Déterminer le nombre de sous-intervalles nécessaires pour que la méthode des trapèzes approche l'intégrale $\int_{0}^{2} e^x dx$ avec une précision de $10^{-3}$.
    \item Calculer le degré d'exactitude de la formule de quadrature élémentaire des trapèzes sur $[-1, 1]$.
\end{enumerate}

\end{document}
```