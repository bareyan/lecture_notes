```latex
\documentclass{article}
\usepackage{amssymb,amsmath,amsthm}
\usepackage{graphicx}
\usepackage{color}
\usepackage{float}
\usepackage{fancyhdr}
\usepackage{array}
\usepackage{listings}
\usepackage{amsfonts} % Pour \mathbb

\newtheorem{theorem}{Théorème}
\newtheorem{lemma}{Lemme}
\newtheorem{proposition}[theorem]{Proposition}
\newtheorem{definition}{Définition}
\newtheorem{remark}{Remarque}
\newtheorem{solution}{Solution}
\newtheorem{example}{Exemple}

\usepackage[margin=1in]{geometry}
\usepackage[utf8]{inputenc}
\usepackage[T1]{fontenc}
\usepackage{lmodern} % Police vectorielle

% Définition de la commande \K pour le corps des scalaires
\newcommand{\K}{\mathbb{K}}
\newcommand{\R}{\mathbb{R}}
\newcommand{\C}{\mathbb{C}}
\newcommand{\N}{\mathbb{N}}

\title{Espaces vectoriels normés}
\date{\today}

\begin{document}
\maketitle
% \tableofcontents % Removed as requested
\sloppy

\section{Espaces vectoriels normés}

\subsection{Rappels sur les espaces vectoriels}
Dans cette section, $E$ désigne un espace vectoriel sur $\K = \R$ ou $\C$, dont les éléments seront notés $x, y$, etc. Le vecteur nul dans $E$ sera noté $0_E$, ou parfois simplement $0$. Les scalaires seront notés $\lambda, \mu$, etc.

\subsection{Norme sur un espace vectoriel}

\begin{definition}
Soit $E$ un espace vectoriel sur $\K$, où $\K = \R$ ou $\C$. Une \textbf{norme} sur $E$ est une application $N : E \to \R^+$ telle que pour tous $x, y \in E$ et $\lambda \in \K$ on a :
\begin{enumerate}
    \item $N(\lambda x) = |\lambda| N(x)$ (homogénéité positive)
    \item $N(x + y) \leq N(x) + N(y)$ (inégalité triangulaire)
    \item $N(x) = 0 \implies x = 0_E$ (séparation)
\end{enumerate}
Un espace vectoriel $E$ muni d'une norme $N$ se note $(E, N)$ et s'appelle un \textbf{espace vectoriel normé} (evn).

Il est habituel de noter une norme par $\|x\|$, plutôt que par $N(x)$.

Une fonction $N : E \to \R^+$ vérifiant seulement les propriétés (1) et (2) est appelée une \textbf{semi-norme}.

Comme pour la fonction valeur absolue, on déduit facilement de (2) que
\begin{equation}
|N(x) - N(y)| \leq N(x - y), \quad x, y \in E.
\end{equation}
\end{definition}

\subsubsection{Norme induite}
Si $(E, \|\cdot\|)$ est un evn et $F \subset E$ est un sous-espace vectoriel, la restriction de $\|\cdot\|$ à $F$ fait de $(F, \|\cdot\|)$ un espace vectoriel normé.

\subsubsection{Exemples}
\begin{itemize}
    \item Soit $E = \K^n$ dont les éléments sont notés $x = (x_1, \dots, x_n)$, $x_i \in \K$. $E$ est un espace vectoriel de dimension $n$ sur $\K$. On pose
    \begin{itemize}
        \item $\|x\|_1 = \sum_{i=1}^n |x_i|$ (norme 1)
        \item $\|x\|_2 = \left(\sum_{i=1}^n |x_i|^2\right)^{1/2}$ (norme euclidienne)
        \item $\|x\|_\infty = \max_{1 \leq i \leq n} |x_i|$ (norme infinie ou norme sup)
        \item et plus généralement $\|x\|_p = \left(\sum_{i=1}^n |x_i|^p\right)^{1/p}$, $1 \leq p < \infty$ (norme p)
    \end{itemize}
    Les fonctions $\|\cdot\|_p$ pour $1 \leq p \leq \infty$ sont des normes sur $\K^n$. C'est facile pour $p = 1, \infty$. Pour $p=2$, l'inégalité triangulaire découle de Cauchy-Schwarz. Pour $p$ quelconque, l'inégalité triangulaire s'appelle l'inégalité de Minkowski.

    \item Soit $X$ un ensemble et soit $\mathcal{B}(X, \K)$ l'ensemble des fonctions $f : X \to \K$ bornées, c'est à dire telles que $\|f\|_\infty := \sup_{x \in X} |f(x)| < \infty$. $(\mathcal{B}(X, \K), \|\cdot\|_\infty)$ est un espace vectoriel normé.
    $\mathcal{B}(X, \K)$ est de dimension finie $n$ si et seulement si $X$ possède exactement $n$ éléments (exercice).

    \item Soit $\mathcal{R}([a, b]; \K)$ l'ensemble des fonctions à valeurs dans $\K$ intégrables au sens de Riemann sur $[a, b]$. $\mathcal{R}([a, b]; \K)$ est de dimension infinie. En effet si on note par $1_{\{c\}}(x)$ la fonction qui vaut 1 si $x = c$ et 0 sinon, alors pour toute famille finie $\{c_1, \dots, c_N\}$ avec $c_i \in [a, b]$ les vecteurs $1_{\{c_1\}}, \dots, 1_{\{c_N\}}$ forment une famille libre de $\mathcal{R}([a, b]; \K)$ (exercice).
    La fonction
    \[ \|f\|_p = \left(\int_a^b |f(x)|^p dx\right)^{1/p}, \quad 1 \leq p < \infty \]
    est une \textbf{semi-norme} sur $\mathcal{R}([a, b]; \K)$. À nouveau l'inégalité triangulaire pour $\|\cdot\|_p$ s'appelle l'inégalité de Minkowski (pour les intégrales). Ce n'est pas une norme car la fonction $f(x) = 1_{\{c\}}(x)$ pour $c \in [a, b]$ est non nulle mais $\|f\|_p = 0$.

    \item Considérons maintenant l'espace $\mathcal{C}([a, b]; \K)$ des fonctions continues sur $[a, b]$. $\mathcal{C}([a, b]; \K)$ est de dimension infinie car les familles $(1, x, x^2, \dots, x^n)$ sont libres pour tout $n \in \N$ (exercice).
    $\|\cdot\|_p$ est une norme sur $\mathcal{C}([a, b]; \K)$. En effet si $\int_a^b |u(x)|^p dx = 0$ et $u$ continue, alors $u(x) = 0$ sur $[a, b]$.

    \item Soit $\K^\N = \{a = (a_n)_{n \in \N}\}$ l'espace vectoriel des suites à valeurs dans $\K$.
    Pour $1 \leq p < \infty$ on pose
    \[ \ell^p(\N; \K) = \left\{(a_n) \in \K^\N : \sum_{n \in \N} |a_n|^p \text{ converge}\right\} \]
    On vérifie que $\ell^p(\N; \K)$ est un espace vectoriel et que
    \[ \|a\|_p = \left(\sum_{n=0}^\infty |a_n|^p\right)^{1/p} \]
    est une norme sur $\ell^p(\N; \K)$.
    De même
    \[ \ell^\infty(\N; \K) = \left\{(a_n) \in \K^\N : \sup_{n \geq 0} |a_n| < \infty\right\} \]
    est un espace vectoriel et
    \[ \|a\|_\infty = \sup_{n \geq 0} |a_n| \]
    est une norme sur $\ell^\infty(\N; \K)$. $\K^\N$ ainsi que $\ell^p(\N; \K)$ sont de dimension infinie. En effet en notant par $\delta_p$ la suite définie par $\delta_{p,n} = 1$ si $n = p$ et 0 sinon, $(\delta_1, \dots, \delta_N)$ est une famille libre pour tout $N \in \N$.
\end{itemize}

\subsection{Topologie des espaces vectoriels normés}

\subsubsection{Distance induite par une norme}

\begin{definition}
Soit $(E, \|\cdot\|)$ un evn. La fonction $E \times E \ni (x, y) \mapsto d(x, y) = \|x - y\|$ est une distance sur $E$, appelée \textbf{distance induite par la norme} $\|\cdot\|$.
\end{definition}

On peut alors définir les boules de centre $a \in E$ de rayon $r$ comme
\[ B(a, r) = \{x \in E : \|x - a\| < r\} \quad \text{(boule ouverte)} \]
\[ B_f(a, r) = \{x \in E : \|x - a\| \leq r\} \quad \text{(boule fermée)} \]
les ensembles ouverts, fermés, compacts etc de $E$.

\begin{definition}
Un espace vectoriel normé complet s'appelle un \textbf{espace de Banach}.
\end{definition}

\begin{remark}
Les espaces vectoriels normés de dimension finie sont complets (quelque soit la norme dont ils sont équipés).
Par contre un espace vectoriel normé de dimension infinie n'est pas nécessairement complet.
Les espaces $(\mathcal{B}(X; \K), \|\cdot\|_\infty)$, $(\mathcal{C}([a, b]; \K), \|\cdot\|_\infty)$, $(\ell^p(\N; \K), \|\cdot\|_p)$ pour $1 \leq p \leq \infty$ sont complets.
L'espace $\mathcal{C}([a, b]; \|\cdot\|_p)$ pour $1 \leq p < \infty$ n'est pas complet, voir Proposition~\ref{prop:C01_non_complet}.
Dans un espace de Banach les ensembles fermés et bornés ne sont pas toujours compacts, voir la Proposition~\ref{prop:l1_non_compact}. En fait on peut montrer que la boule unité fermée $B_f(x_0, r)$ est compacte dans $(E, \|\cdot\|)$ si et seulement si $E$ est de dimension finie.
\end{remark}

\subsubsection{Exemples}

\begin{proposition} \label{prop:C01_non_complet}
L'espace $E = \mathcal{C}([0, 1]; \R)$ muni de la norme $\|f\|_1 = \int_0^1 |f(x)| dx$ n'est pas complet.
\end{proposition}

\begin{proof}
Soit la suite de fonctions $(f_n)_{n \in \N}$ définie par
\[
f_n(x) =
\begin{cases}
0 & \text{si } 0 \leq x \leq \frac{1}{2} - \frac{1}{n} \\
\frac{n}{2} (x - (\frac{1}{2} - \frac{1}{n})) & \text{si } \frac{1}{2} - \frac{1}{n} \leq x \leq \frac{1}{2} + \frac{1}{n} \\
1 & \text{si } \frac{1}{2} + \frac{1}{n} \leq x \leq 1
\end{cases}
\]
pour $n \geq 2$. Chaque $f_n$ est continue sur $[0, 1]$.
Soit la fonction $f$ définie par
\[
f(x) =
\begin{cases}
0 & \text{si } 0 \leq x \leq \frac{1}{2} \\
1 & \text{si } \frac{1}{2} < x \leq 1
\end{cases}
\]
$f$ n'est pas continue en $x=1/2$.

\begin{verbatim}
#save_to: fig_fn_prop64.png
import matplotlib.pyplot as plt
import matplotlib.patches as patches
import numpy as np

n = 5 # Example value for n
x1 = np.linspace(0, 1/2 - 1/n, 10)
y1 = np.zeros_like(x1)
x2 = np.linspace(1/2 - 1/n, 1/2 + 1/n, 10)
y2 = (n/2) * (x2 - (1/2 - 1/n))
x3 = np.linspace(1/2 + 1/n, 1, 10)
y3 = np.ones_like(x3)

x = np.concatenate((x1, x2, x3))
y = np.concatenate((y1, y2, y3))

fig, ax = plt.subplots()
ax.plot(x, y, label=f'$f_{n}(x)$ (n={n})')
ax.set_xlabel('x')
ax.set_ylabel('$f_n(x)$')
ax.set_title('Fonction $f_n(x)$')
ax.grid(True)
ax.axvline(1/2 - 1/n, color='r', linestyle='--', lw=1, label='$1/2 - 1/n$')
ax.axvline(1/2 + 1/n, color='g', linestyle='--', lw=1, label='$1/2 + 1/n$')
ax.legend()
plt.ylim(-0.1, 1.1)
plt.savefig('fig_fn_prop64.png')
\end{verbatim}

\begin{figure}[H]
    \centering
    \includegraphics[max width=0.6\textwidth, keepaspectratio]{fig_fn_prop64.png}
    \caption{Illustration de la fonction $f_n(x)$ pour $n=5$.}
    \label{fig:fn_prop64}
\end{figure}

On vérifie directement que $\|f - f_n\|_1 \to 0$. Calculons la norme $\|f - f_n\|_1$:
\begin{align*}
\|f - f_n\|_1 &= \int_0^1 |f(x) - f_n(x)| dx \\
&= \int_{1/2 - 1/n}^{1/2} \left|0 - \frac{n}{2}(x - (\frac{1}{2} - \frac{1}{n}))\right| dx + \int_{1/2}^{1/2 + 1/n} \left|1 - \frac{n}{2}(x - (\frac{1}{2} - \frac{1}{n}))\right| dx \\
&= \int_{1/2 - 1/n}^{1/2} \frac{n}{2}(x - \frac{1}{2} + \frac{1}{n}) dx + \int_{1/2}^{1/2 + 1/n} \left(1 - \frac{n}{2}x + \frac{n}{4} - \frac{1}{2}\right) dx \\
&= \int_{1/2 - 1/n}^{1/2} \frac{n}{2}(x - \frac{1}{2} + \frac{1}{n}) dx + \int_{1/2}^{1/2 + 1/n} \left(\frac{1}{2} + \frac{n}{4} - \frac{n}{2}x\right) dx
\end{align*}
L'aire correspond à deux triangles de base $1/n$ et de hauteur $1/2$.
\[ \|f - f_n\|_1 = \text{Aire}(\text{triangle sous } f_n \text{ entre } 1/2-1/n \text{ et } 1/2) + \text{Aire}(\text{triangle sur } f_n \text{ entre } 1/2 \text{ et } 1/2+1/n) \]
\[ \|f - f_n\|_1 = \frac{1}{2} \times \frac{1}{n} \times \frac{1}{2} + \frac{1}{2} \times \frac{1}{n} \times \frac{1}{2} = \frac{1}{2n} \]
Donc $\|f - f_n\|_1 \to 0$ quand $n \to \infty$.

Montrons que $(f_n)$ est une suite de Cauchy dans $(\mathcal{C}([0, 1]; \R), \|\cdot\|_1)$. Pour $m, n \in \N$,
\[ \|f_n - f_m\|_1 \leq \|f_n - f\|_1 + \|f - f_m\|_1 = \frac{1}{2n} + \frac{1}{2m} \]
Pour $\epsilon > 0$, il existe $N \in \N$ tel que pour $n, m \geq N$, $\frac{1}{2n} < \epsilon/2$ et $\frac{1}{2m} < \epsilon/2$. Donc $\|f_n - f_m\|_1 < \epsilon$. La suite $(f_n)$ est bien de Cauchy.

Supposons qu'il existe $g \in \mathcal{C}([0, 1]; \R)$ tel que $\|f_n - g\|_1 \to 0$. Alors par unicité de la limite dans l'espace des fonctions intégrables $L^1([0,1])$ (qui contient $\mathcal{C}([0, 1])$), on aurait $\|f - g\|_1 = 0$.
\[ \|f - g\|_1 = \int_0^1 |f(x) - g(x)| dx = 0 \]
Comme $f$ et $g$ sont continues sur $[0, 1/2[$ et sur $]1/2, 1]$ (puisque $g$ est continue sur $[0,1]$ et $f$ est constante sur ces intervalles), et que l'intégrale de la valeur absolue de leur différence est nulle, cela implique $f(x) = g(x)$ pour $x \in [0, 1/2[$ et $x \in ]1/2, 1]$.
Donc $g(x) = 0$ pour $x \in [0, 1/2[$ et $g(x) = 1$ pour $x \in ]1/2, 1]$.
Mais $g$ doit être continue en $x=1/2$. On aurait $\lim_{x \to (1/2)^-} g(x) = 0$ et $\lim_{x \to (1/2)^+} g(x) = 1$. Ceci contredit la continuité de $g$ au point $1/2$.
La suite de Cauchy $(f_n)$ ne converge donc pas dans $(\mathcal{C}([0, 1]; \R), \|\cdot\|_1)$.
\end{proof}


\begin{proposition} \label{prop:l1_non_compact}
Dans l'espace $E = (\ell^1(\N; \R), \|\cdot\|_1)$, la boule unité fermée $B_f(0, 1)$ n'est pas compacte.
\end{proposition}

\begin{proof}
Pour éviter des complications de notation, on note un élément de $E$ (c'est à dire une suite réelle) par $u$, et le $n$-ième terme de $u$ par $u(n)$. On note que pour $n \in \N$ fixé on a
\begin{equation} \label{eq:un_leq_u1}
|u(n)| \leq \|u\|_1, \quad u \in E.
\end{equation}
Soit $(u_p)_{p \in \N}$ la suite d'éléments de $E$ définie par $u_p(n) = \delta_{np}$ (symbole de Kronecker, $u_p$ est la suite qui a 1 en position $p$ et 0 ailleurs). On a $\|u_p\|_1 = \sum_{n=0}^\infty |\delta_{np}| = 1$. Donc $(u_p)_{p \in \N}$ est une suite dans $B_f(0, 1)$. $B_f(0,1)$ est un ensemble fermé et borné dans $E$.

Supposons qu'il existe une sous-suite $(u_{\varphi(p)})_{p \in \N}$ qui converge vers $u \in E$. Par l'inégalité triangulaire inversée (issue de (6.1)), on a
\[ | \|u\|_1 - \|u_{\varphi(p)}\|_1 | \leq \|u - u_{\varphi(p)}\|_1 \]
Comme $\|u_{\varphi(p)}\|_1 = 1$ et $\|u - u_{\varphi(p)}\|_1 \to 0$ quand $p \to \infty$, on a $\lim_{p \to \infty} \|u_{\varphi(p)}\|_1 = \|u\|_1$. Donc $\|u\|_1 = 1$.

D'après \eqref{eq:un_leq_u1} appliqué à la différence $u_{\varphi(p)} - u$, on a
\[ |u_{\varphi(p)}(n) - u(n)| \leq \|u_{\varphi(p)} - u\|_1 \]
Pour $n$ fixé, comme $\|u_{\varphi(p)} - u\|_1 \to 0$, on a $\lim_{p \to \infty} u_{\varphi(p)}(n) = u(n)$.
Or $u_{\varphi(p)}(n) = \delta_{n, \varphi(p)}$. Comme $\varphi: \N \to \N$ est strictement croissante, $\varphi(p) \to \infty$ quand $p \to \infty$. Donc pour $p$ assez grand, $\varphi(p) > n$, ce qui implique $\delta_{n, \varphi(p)} = 0$.
Ainsi, pour tout $n$ fixé, $\lim_{p \to \infty} u_{\varphi(p)}(n) = 0$.
On en déduit que $u(n) = 0$ pour tout $n \in \N$. Donc $u$ est la suite nulle.
Alors $\|u\|_1 = 0$. Ceci contredit le fait que $\|u\|_1 = 1$.

Donc la suite $(u_p)$ n'admet aucune sous-suite convergente dans $E$. La boule unité fermée $B_f(0, 1)$ dans $E = (\ell^1(\N; \R), \|\cdot\|_1)$ est donc fermée et bornée mais non compacte.
\end{proof}

\subsection{Normes équivalentes}
Un point important à garder à l'esprit est que la notion d'ensembles ouverts, d'applications continues, d'ensembles compacts etc dans un espace vectoriel $E$ dépend a priori du choix d'une norme sur $E$. Deux normes différentes sur $E$ conduisent en général à deux topologies différentes.

\begin{definition}
Soit $N_1, N_2$ deux normes sur un espace vectoriel $E$.
\begin{enumerate}
    \item On dit que $N_1$ et $N_2$ sont \textbf{topologiquement équivalentes} si $(E, N_1)$ et $(E, N_2)$ ont les mêmes ensembles ouverts.
    \item On dit que $N_1$ et $N_2$ sont \textbf{équivalentes} (on écrit parfois $N_1 \sim N_2$) s'il existe $C_1, C_2 > 0$ telles que
    \[ N_1(x) \leq C_1 N_2(x), \quad N_2(x) \leq C_2 N_1(x), \quad \forall x \in E. \]
\end{enumerate}
On peut réécrire (2) de manière plus élégante comme : il existe $C > 0$ tel que
\[ C^{-1} N_2(x) \leq N_1(x) \leq C N_2(x), \quad \forall x \in E, \]
(prendre $C = \max(C_1, C_2^{-1})$). La taille des constantes $C_1, C_2$ ou $C$ n'a que peu d'importance.
Il est facile de voir que si $N_1 \sim N_2$ et $N_2 \sim N_3$ alors $N_1 \sim N_3$ (la relation $\sim$, comme la relation d'équivalence topologique, sont des relations d'équivalence).
\end{definition}

\begin{theorem}
Deux normes sur $E$ sont topologiquement équivalentes si et seulement si elles sont équivalentes.
\end{theorem}
\begin{proof}
La démonstration sera faite dans 6.6.3 (non fournie dans ces notes). Esquisse: L'identité $Id: (E, N_1) \to (E, N_2)$ et son inverse $Id: (E, N_2) \to (E, N_1)$ doivent être continues. La continuité d'une application linéaire est équivalente à être bornée (Théorème 6.14), ce qui donne les inégalités de la définition d'équivalence.
\end{proof}

\begin{remark}
Il est facile de voir que les normes $\|\cdot\|_p$ sur $\K^n$ sont équivalentes entre elles. En effet on vérifie (exercice) que
\[ \|x\|_\infty \leq \|x\|_p \leq n^{1/p} \|x\|_\infty, \quad x \in \K^n. \]
(Indication: $\|x\|_p^p = \sum |x_i|^p \leq \sum (\|x\|_\infty)^p = n \|x\|_\infty^p$. Et $|x_k| = (|x_k|^p)^{1/p} \leq (\sum |x_i|^p)^{1/p} = \|x\|_p$, donc $\|x\|_\infty = \max |x_k| \leq \|x\|_p$.)

On verra dans la suite un résultat a priori surprenant qui affirme que toutes les normes sur $\K^n$ sont équivalentes.

Par contre les normes $\|\cdot\|_p$ sur $\mathcal{C}([a, b]; \K)$ ne sont pas équivalentes entre elles. Par exemple si $[a, b] = [-1, 1]$ et
\[
f_n(x) =
\begin{cases}
1 - n|x| & \text{si } |x| \leq n^{-1} \\
0 & \text{sinon}
\end{cases}
\]
(tracer son graphe)
\begin{verbatim}
#save_to: fig_fn_rem68.png
import matplotlib.pyplot as plt
import matplotlib.patches as patches
import numpy as np

n = 5 # Example value for n
x = np.linspace(-1, 1, 400)
y = np.maximum(0, 1 - n * np.abs(x))

fig, ax = plt.subplots()
ax.plot(x, y, label=f'$f_n(x)$ (n={n})')
ax.set_xlabel('x')
ax.set_ylabel('$f_n(x)$')
ax.set_title('Fonction $f_n(x) = \max(0, 1-n|x|)$ sur $[-1, 1]$')
ax.grid(True)
ax.axvline(-1/n, color='r', linestyle='--', lw=1, label='$-1/n$')
ax.axvline(1/n, color='g', linestyle='--', lw=1, label='$1/n$')
ax.legend()
plt.ylim(-0.1, 1.1)
plt.savefig('fig_fn_rem68.png')
\end{verbatim}

\begin{figure}[H]
    \centering
    \includegraphics[max width=0.6\textwidth, keepaspectratio]{fig_fn_rem68.png}
    \caption{Graphe de la fonction $f_n(x)$ pour $n=5$.}
    \label{fig:fn_rem68}
\end{figure}

On a $\|f_n\|_\infty = 1$ et $\|f_n\|_1 = \int_{-1}^1 f_n(x) dx = \int_{-1/n}^{1/n} (1 - n|x|) dx = 2 \int_0^{1/n} (1 - nx) dx = 2 [x - \frac{nx^2}{2}]_0^{1/n} = 2 (\frac{1}{n} - \frac{n(1/n)^2}{2}) = 2 (\frac{1}{n} - \frac{1}{2n}) = \frac{1}{n}$.
Ceci entraîne qu'il n'existe pas de constante $C$ telle que $\|f\|_\infty \leq C \|f\|_1$ pour tout $f \in \mathcal{C}([-1, 1]; \K)$ (prendre $f = f_n$ et faire $n \to \infty$, on aurait $1 \leq C/n \to 0$, contradiction).
Donc $\|\cdot\|_\infty$ et $\|\cdot\|_1$ ne sont pas équivalentes sur $\mathcal{C}([-1, 1])$.
\end{remark}

\subsubsection{Équivalence des normes en dimension finie}

\begin{theorem} \label{thm:equiv_dim_finie}
Soit $E$ un espace vectoriel sur $\K$ de dimension finie. Alors $E$ possède au moins une norme et toutes les normes sur $E$ sont équivalentes.
\end{theorem}

\begin{proof}
On peut supposer que $\K = \R$ (en identifiant $\C$ à $\R^2$).
Soit $n = \dim(E)$. On fixe alors une base $\mathcal{B} = (e_1, \dots, e_n)$ de $E$. On peut identifier $E$ à $\R^n$ via l'isomorphisme $\Phi: \R^n \to E$ défini par $\Phi(x_1, \dots, x_n) = \sum_{i=1}^n x_i e_i$.
Considérons la norme $\|\cdot\|_\infty$ sur $\R^n$ définie par $\|x\|_\infty = \max_{1 \leq i \leq n} |x_i|$ pour $x = (x_1, \dots, x_n)$.

Soit $N$ une norme quelconque sur $E$. On définit une norme $N'$ sur $\R^n$ par $N'(x) = N(\Phi(x)) = N(\sum_{i=1}^n x_i e_i)$.
Montrons que $N'$ est équivalente à $\|\cdot\|_\infty$ sur $\R^n$.
Par l'inégalité triangulaire et l'homogénéité de $N$,
\begin{align*}
N'(x) = N\left(\sum_{i=1}^n x_i e_i\right) &\leq \sum_{i=1}^n N(x_i e_i) \\
&= \sum_{i=1}^n |x_i| N(e_i) \\
&\leq \left(\sum_{i=1}^n N(e_i)\right) \max_{1 \leq j \leq n} |x_j| \\
&= C \|x\|_\infty
\end{align*}
où $C = \sum_{i=1}^n N(e_i)$ est une constante positive (car les $e_i$ ne sont pas nuls).

Maintenant, montrons que $N': \R^n \to \R^+$ est une fonction continue pour la topologie induite par $\|\cdot\|_\infty$. Pour $x, y \in \R^n$,
\[ |N'(x) - N'(y)| \leq N'(x - y) \quad \text{(par (6.1))} \]
\[ N'(x - y) \leq C \|x - y\|_\infty \]
Donc $|N'(x) - N'(y)| \leq C \|x - y\|_\infty$, ce qui prouve que $N'$ est $C$-lipschitzienne, donc continue sur $(\R^n, \|\cdot\|_\infty)$.

Considérons la sphère unité pour la norme infinie : $S_\infty(0, 1) = \{x \in \R^n : \|x\|_\infty = 1\}$.
$S_\infty(0, 1)$ est un ensemble fermé (car l'application $x \mapsto \|x\|_\infty$ est continue) et borné dans $\R^n$. Par le Théorème de Borel-Lebesgue (Théorème 3.36), $S_\infty(0, 1)$ est compact.

Comme $N'$ est continue sur l'espace $(\R^n, \|\cdot\|_\infty)$ et $S_\infty(0, 1)$ est compact, $N'$ est bornée sur $S_\infty(0, 1)$ et atteint ses bornes (par le Corollaire 4.8).
Soit $m = \min_{x \in S_\infty(0, 1)} N'(x)$ et $M = \max_{x \in S_\infty(0, 1)} N'(x)$.
Comme $N'(x) = 0 \iff x = 0$, et que $0 \notin S_\infty(0, 1)$, on a $N'(x) > 0$ pour tout $x \in S_\infty(0, 1)$. Donc $m > 0$.
Ainsi, il existe des constantes $0 < m \leq M$ telles que
\[ m \leq N'(x) \leq M, \quad \forall x \in S_\infty(0, 1). \]

Soit maintenant $y \in \R^n$, $y \neq 0$. Alors $x = y / \|y\|_\infty \in S_\infty(0, 1)$.
On applique l'inégalité précédente à $x$:
\[ m \leq N'(y / \|y\|_\infty) \leq M \]
Par homogénéité de $N'$, $N'(y / \|y\|_\infty) = N'(y) / \|y\|_\infty$.
\[ m \leq \frac{N'(y)}{\|y\|_\infty} \leq M \]
\[ m \|y\|_\infty \leq N'(y) \leq M \|y\|_\infty, \quad \forall y \in \R^n, y \neq 0. \]
L'inégalité est aussi vraie pour $y = 0$.
Donc $N'$ est équivalente à $\|\cdot\|_\infty$.

Comme $N$ était une norme arbitraire sur $E$, et que $N'$ est la norme correspondante sur $\R^n$, toutes les normes sur $E$ induisent des normes sur $\R^n$ qui sont équivalentes à $\|\cdot\|_\infty$. Par transitivité, toutes les normes sur $E$ sont équivalentes entre elles.
\end{proof}

\subsection{Compléments sur les espaces vectoriels normés}
\subsubsection{Norme $L^\infty$ et convergence uniforme}
Soit $X$ un ensemble (par exemple un intervalle de $\R$) et $(f_n)_{n \in \N}$ une suite de fonctions $f_n : X \to \R$.
On rappelle que la suite de fonctions $(f_n)$ converge \textbf{simplement} vers une fonction $f : X \to \R$ si pour tout $x \in X$ la suite réelle $(f_n(x))$ converge vers $f(x)$. La convergence simple est la notion la plus faible de convergence pour des suites de fonctions. Exprimée à l'aide de quantificateurs, elle s'écrit :
\begin{equation} \label{eq:conv_simple}
\forall x \in X, \forall \epsilon > 0, \exists N \in \N \text{ tel que } |f_n(x) - f(x)| \leq \epsilon, \forall n \geq N,
\end{equation}
l'entier $N$ dépendant a priori de $\epsilon$ et de $x$.

La suite de fonctions $(f_n)$ converge \textbf{uniformément} vers une fonction $f : X \to \R$ si
\begin{equation} \label{eq:conv_unif}
\forall \epsilon > 0, \exists N \in \N \text{ tel que } |f_n(x) - f(x)| \leq \epsilon, \forall n \geq N, \forall x \in X.
\end{equation}
L'entier $N$ ne dépend maintenant que de $\epsilon$. On peut réécrire (6.4) comme
\begin{equation} \label{eq:conv_unif_sup}
\forall \epsilon > 0, \exists N \in \N \text{ tel que } \sup_{x \in X} |f_n(x) - f(x)| \leq \epsilon, \forall n \geq N.
\end{equation}
On voit donc que sur l'espace vectoriel $\mathcal{B}(X, \R)$ des fonctions bornées sur $X$, une suite $(f_n)$ converge uniformément vers $f$ si et seulement si $(f_n)$ converge vers $f$ pour la norme $\|\cdot\|_\infty$.


\end{document}
```