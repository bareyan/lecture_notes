```latex
\documentclass{article}
\usepackage[T1]{fontenc}
\usepackage[utf8]{inputenc}
\usepackage{amsmath}
\usepackage{amsfonts}
\usepackage{amssymb}
\usepackage{graphicx}
\usepackage{geometry}
\usepackage{verbatim}
\usepackage{listings}
\usepackage{amsthm}

\geometry{a4paper, margin=1in}

\newtheorem{theorem}{Théorème}
\newtheorem{lemma}{Lemme}
\newtheorem{proposition}[theorem]{Proposition}
\newtheorem{definition}{Définition}
\newtheorem{remark}{Remarque}
\newtheorem{solution}{Solution}
\newtheorem{example}{Exemple}

\begin{document}
\sloppy

\section{Espaces Vectoriels Normés}

\subsection{Espaces Vectoriel Normés}

\begin{definition}[Espace Vectoriel Normé]
\label{def:evn}
Soit $E$ un espace vectoriel sur $\mathbb{K}$, où $\mathbb{K} = \mathbb{R}$ ou $\mathbb{C}$. Une norme sur $E$ est une application $N: E \rightarrow \mathbb{R}^+$ telle que pour tous $x, y \in E$ et $\lambda \in \mathbb{K}$ on a :
\begin{enumerate}
    \item $N(\lambda x) = |\lambda| N(x)$.
    \item $N(x+y) \leq N(x) + N(y)$ (inégalité triangulaire).
    \item $N(x) = 0 \implies x = 0_E$.
\end{enumerate}
Un espace vectoriel $E$ muni d'une norme $N$ se note $(E, N)$ et s'appelle un espace vectoriel normé.
\end{definition}

\begin{remark}
Il est habituel de noter une norme par $\|x\|$ plutôt que par $N(x)$. On rencontre parfois dans la littérature le symbole $|||A|||$ quand $A$ est une application linéaire ou une matrice, qui est à éviter.
\end{remark}

\begin{definition}[Semi-norme]
Une fonction $N: E \rightarrow \mathbb{R}^+$ vérifiant seulement les propriétés (1) et (2) de la Définition~\ref{def:evn} est appelée une \textbf{semi-norme}.
\end{definition}

Comme pour la valeur absolue, on déduit facilement de la propriété (2) de la Définition~\ref{def:evn} que
\begin{equation}
    |N(x) - N(y)| \leq N(x-y), \forall x, y \in E.
\end{equation}

\subsubsection{Norme induite}
Si $(E, \|.\|)$ est un evn et $F \subset E$ est un sous-espace vectoriel, la restriction de $\|.\|$ à $F$ fait de $(F, \|.\|)$ un espace vectoriel normé.

\subsubsection{Exemples}
Soit $E = \mathbb{K}^n$ dont les éléments sont notés $x = (x_1, ..., x_n)$, $x_i \in \mathbb{K}$. $E$ est un espace vectoriel de dimension $n$ sur $\mathbb{K}$. On pose:
\begin{align*}
    \|x\|_1 &= \sum_{i=1}^n |x_i|, \\
    \|x\|_2 &= \left( \sum_{i=1}^n |x_i|^2 \right)^{1/2}, \\
    \|x\|_\infty &= \max_{1 \leq i \leq n} |x_i|,
\end{align*}
et plus généralement
\begin{equation}
    \|x\|_p = \left( \sum_{i=1}^n |x_i|^p \right)^{1/p}, \quad 1 \leq p < \infty.
\end{equation}
Les fonctions $\|.\|_p$ pour $1 \leq p \leq \infty$ sont des normes sur $\mathbb{K}^n$. C'est facile à vérifier pour $p=1, \infty$, et déjà vu pour $p=2$. Pour $p$ quelconque, l'inégalité triangulaire s'appelle \textit{l'inégalité de Minkowski}.

Soit $X$ un ensemble et soit $\mathcal{B}(X, \mathbb{K})$ l'ensemble des fonctions $f: X \rightarrow \mathbb{K}$ bornées, c'est à dire telles que $\|f\|_X = \sup_{x \in X} |f(x)| < \infty$. $(\mathcal{B}(X, \mathbb{K}), \|.\|_\infty)$ est un espace vectoriel normé.

$\mathcal{B}(X, \mathbb{K})$ est de dimension finie $n$ si et seulement si $X$ possède exactement $n$ éléments (exercice).

Soit $R([a, b]; \mathbb{K})$ l'ensemble des fonctions à valeurs dans $\mathbb{K}$ intégrables au sens de Riemann sur $[a, b]$. $R([a, b]; \mathbb{K})$ est de dimension infinie. En effet si on note par $1_{\{c\}}(x)$ la fonction qui vaut $1$ si $x=c$ et $0$ sinon, alors pour toute famille finie $\{c_1, ..., c_N\}$ avec $c_i \in [a, b]$ les vecteurs $1_{\{c_1\}}, ..., 1_{\{c_N\}}$ forment une famille libre de $R([a, b]; \mathbb{K})$ (exercice).

La fonction
\begin{equation}
    \|f\|_p = \left( \int_a^b |f(x)|^p dx \right)^{1/p}, \quad 1 \leq p < \infty
\end{equation}
est une semi-norme sur $R([a, b]; \mathbb{K})$. À nouveau l'inégalité triangulaire pour $\|.\|_p$ s'appelle l'\textit{inégalité de Minkowski (pour les intégrales)}. Ce n'est pas une norme car la fonction $f(x) = 1_{\{c\}}(x)$ pour $c \in [a, b]$ est non nulle mais $\|f\|_p = 0$.

\subsection{Considérons maintenant l'espace $C([a, b]; \mathbb{K})$ des fonctions continues sur $[a, b]$}
$C([a, b]; \mathbb{K})$ est de dimension infinie car les familles $(1, x, x^2, ..., x^n)$ sont libres pour tout $n \in \mathbb{N}$ (exercice).

$\|.\|_p$ est une norme sur $C([a, b]; \mathbb{K})$. En effet si $\int_a^b |u(x)|^p dx = 0$ et $u$ continue, alors $u(x) = 0$ sur $[a, b]$.

Soit $\mathbb{K}^\mathbb{N} = \{a = (a_n)_{n \in \mathbb{N}}\}$ l'espace vectoriel des suites à valeurs dans $\mathbb{K}$.

Pour $1 \leq p < \infty$ on pose
\begin{equation}
    l^p(\mathbb{N}; \mathbb{K}) = \left\{ (a_n) : \sum_{n \in \mathbb{N}} |a_n|^p \text{ converge} \right\}.
\end{equation}
On vérifie que $l^p(\mathbb{N}; \mathbb{K})$ est un espace vectoriel et que
\begin{equation}
    \|a\|_p = \left( \sum_{n=0}^\infty |a_n|^p \right)^{1/p}
\end{equation}
est une norme sur $l^p(\mathbb{N}; \mathbb{K})$.

De même
\begin{equation}
    l^\infty(\mathbb{N}; \mathbb{K}) = \left\{ (a_n) : \sup_{n \geq 0} |a_n| < \infty \right\},
\end{equation}
est un espace vectoriel et
\begin{equation}
    \|a\|_\infty = \sup_{n \geq 0} |a_n|
\end{equation}
est une norme sur $l^\infty(\mathbb{N}; \mathbb{K})$. $\mathbb{K}^\mathbb{N}$ ainsi que $l^p(\mathbb{N}; \mathbb{K})$ sont de dimension infinie. En effet en notant par $\delta_p$ la suite définie par $\delta_{p,n} = 1$ si $n=p$ et $0$ sinon, $(\delta_1, ..., \delta_N)$ est une famille libre pour tout $N \in \mathbb{N}$.

\subsection{Topologie des espaces vectoriels normés}

\subsubsection{Distance induite par une norme}
\begin{definition}[Distance induite par une norme]
\label{def:distance_induite}
Soit $(E, \|.\|)$ un evn. La fonction $E \times E \ni (x, y) \mapsto d(x, y) = \|x-y\|$ est une distance sur $E$, appelée distance induite par la norme $\|.\|$.
\end{definition}

On peut alors définir les boules de centre $a \in E$ de rayon $r$ comme
\begin{align*}
    B(a, r) &= \{x \in E : \|x-a\| < r\}, \\
    B_f(a, r) &= \{x \in E : \|x-a\| \leq r\},
\end{align*}
les ensembles ouverts, fermés, compacts etc. de $E$.

\begin{definition}[Espace de Banach]
\label{def:banach}
Un espace vectoriel normé complet s'appelle un espace de \textbf{Banach}.
\end{definition}

Les espaces vectoriels normés de dimension finie sont complets (quelque soit la norme dont ils sont équipés).

Par contre un espace vectoriel normé de dimension \textit{infinie} n'est pas nécessairement complet.

Les espaces $(\mathcal{B}(X; \mathbb{K}), \|.\|_\infty)$, $(C([a, b]; \mathbb{K}), \|.\|_\infty)$, $(l^p(\mathbb{N}; \mathbb{K}), \|.\|_p)$ pour $1 \leq p \leq \infty$ sont complets.

L'espace $(C([a, b]; \mathbb{K}), \|.\|_p)$ pour $1 \leq p < \infty$ n'est pas complet, voir la Proposition~\ref{prop:C_not_banach}.

Dans un espace de Banach les ensembles fermés et bornés ne sont pas toujours compacts, voir la Proposition~\ref{prop:unit_ball_not_compact}. En fait on peut montrer que la boule unité fermée $B_f(x_0, r)$ est compacte dans $(E, \|.\|)$ si et seulement si $E$ est de dimension finie.

\subsubsection{Exemples}
Donnons un exemple d'espace vectoriel normé de dimension infinie qui n'est pas complet.

\begin{proposition}
\label{prop:C_not_banach}
L'espace $E = C([0, 1]; \mathbb{R})$ muni de la norme $\|f\|_1 = \int_0^1 |f(x)| dx$ n'est pas complet.
\end{proposition}

\begin{proof}
Soit
\begin{equation}
    f_n(x) =
    \begin{cases}
        0 & \text{si } 0 \leq x \leq \frac{1}{2} - n^{-1}, \\
        \frac{n}{2}(x - (\frac{1}{2} - n^{-1})) & \text{si } \frac{1}{2} - n^{-1} \leq x \leq \frac{1}{2} + n^{-1}, \\
        1 & \text{si } \frac{1}{2} + n^{-1} \leq x \leq 1,
    \end{cases}
\end{equation}
et
\begin{equation}
    f(x) =
    \begin{cases}
        0 & \text{si } 0 \leq x \leq \frac{1}{2}, \\
        1 & \text{si } \frac{1}{2} < x \leq 1.
    \end{cases}
\end{equation}

\begin{verbatim}
#save_to: plot_fn_f.png
import matplotlib.pyplot as plt
import numpy as np

def fn(x, n):
    if 0 <= x <= 1/2 - 1/n:
        return 0
    elif 1/2 - 1/n < x <= 1/2 + 1/n:
        return (n/2) * (x - (1/2 - 1/n))
    elif 1/2 + 1/n < x <= 1:
        return 1
    else:
        return 0

def f(x):
    if 0 <= x <= 1/2:
        return 0
    elif 1/2 < x <= 1:
        return 1
    else:
        return 0

x_vals = np.linspace(0, 1, 1000)
n_val = 20 # Example n value, adjust as needed

f_n_vals = [fn(x, n_val) for x in x_vals]
f_vals = [f(x) for x in x_vals]

plt.figure(figsize=(8, 6))
plt.plot(x_vals, f_n_vals, label='$f_n(x)$', color='red')
plt.plot(x_vals, f_vals, label='$f(x)$', color='blue', linestyle='--')

plt.axvline(x=1/2 - 1/n_val, color='gray', linestyle=':')
plt.axvline(x=1/2 + 1/n_val, color='gray', linestyle=':')
plt.axvline(x=1/2, color='black', linestyle=':')

plt.xlabel('x')
plt.ylabel('y')
plt.title('$f_n(x)$ and $f(x)$ functions')
plt.legend()
plt.grid(True)
plt.savefig('plot_fn_f.png')
plt.close()

\end{verbatim}

\begin{figure}[h]
    \centering
    \includegraphics[ max width=\textwidth,
     max height=0.4\textheight,
     keepaspectratio]{plot_fn_f.png}
    \caption{Fonctions $f_n(x)$ et $f(x)$}
    \label{fig:plot_fn_f}
\end{figure}
On vérifie directement que $\|f - f_n\|_1 \rightarrow 0$ et donc $\forall \epsilon > 0$, $\exists N \in \mathbb{N}$ tel que $\|f - f_n\|_1 \leq \epsilon$ si $n > N$. Si $n, p > N$ on a donc $\|f_n - f_p\|_1 \leq \|f_n - f\|_1 + \|f_p - f\|_1 \leq 2\epsilon$, donc $(f_n)$ est de Cauchy dans $(C([0, 1]; \mathbb{R}), \|.\|_1)$.

Supposons qu'il existe $g \in C([0, 1]; \mathbb{R})$ tel que $\|f_n - g\|_1 \rightarrow 0$. On a donc $\|f - g\|_1 = 0$, donc
\begin{equation}
    \int_0^1 |f - g|(x) dx = \int_{0}^{1/2} |f - g|(x) dx + \int_{1/2}^{1} |f - g|(x) dx = 0.
\end{equation}
Puisque $f(x) = 0$ pour $x \in [0, 1/2]$ et $f(x) = 1$ pour $x \in (1/2, 1]$, si $g$ était continue et $\|f-g\|_1 = 0$, alors $g(x) = f(x)$ presque partout. Cependant, $f$ est discontinue en $x=1/2$, donc $g$ ne peut pas être continue et égale à $f$ partout. Plus précisément, si $\int_0^{1/2} |0 - g(x)| dx = 0$ et $g$ est continue, alors $g(x) = 0$ sur $[0, 1/2]$. De même, si $\int_{1/2}^{1} |1 - g(x)| dx = 0$ et $g$ est continue, alors $g(x) = 1$ sur $[1/2, 1]$. C'est une contradiction car une fonction continue ne peut pas être à la fois $0$ sur $[0, 1/2]$ et $1$ sur $[1/2, 1]$ à moins d'être constante et égale à la fois à 0 et à 1, ce qui est impossible. Ainsi, $g$ ne peut pas être continue.
La suite de Cauchy $(f_n)$ ne converge pas dans $(C([0, 1]; \mathbb{R}), \|.\|_1)$.
\end{proof}

Voici maintenant un exemple d'espace de Banach de dimension infinie dans lequel les ensembles fermés bornés ne sont pas nécessairement compacts.

\begin{proposition}
\label{prop:unit_ball_not_compact}
Dans l'espace $E = (l^1(\mathbb{N}; \mathbb{R}), \|.\|_1)$ la boule unité fermée $B_f(0, 1)$ n'est pas compacte.
\end{proposition}

\begin{proof}
Pour éviter des complications de notation, on note un élément de $E$ (c'est à dire une suite réelle) par $u$, et le $n$-ième terme de $u$ par $u(n)$. On note que pour $n \in \mathbb{N}$ fixé on a
\begin{equation}
    |u(n)| \leq \|u\|_1, \quad u \in E.
\end{equation}
Soit $(u_p)_{p \in \mathbb{N}}$ la suite d'éléments de $E$ définie par $u_p(n) = \delta_{np}$. On a $\|u_p\|_1 = 1$, donc $(u_p)_{p \in \mathbb{N}}$ est une suite dans $B_f(0, 1)$ qui est un ensemble fermé et borné dans $E$. Supposons qu'il existe une sous-suite $(u_{\varphi(p)})$ qui converge vers $u \in E$. Par l'inégalité triangulaire (Définition~\ref{def:evn}, propriété 2) on a
\begin{equation}
    |\|u\|_1 - \|u_{\varphi(p)}\|_1| \leq \|u - u_{\varphi(p)}\|_1,
\end{equation}
donc en faisant $p \rightarrow \infty$ on a $\|u\|_1 = 1$. D'après l'inégalité (\ref{eq:6.2}) appliqué à $u_{\varphi(p)} - u$ on a
\begin{equation}
    |u_{\varphi(p)}(n) - u(n)| \leq \|u_{\varphi(p)} - u\|_1,
\end{equation}
donc pour $n$ fixé $\lim_{p \rightarrow \infty} |u_{\varphi(p)}(n) - u(n)| = 0$. Comme $u_{\varphi(p)}(n) = 0$ dès que $\varphi(p) > n$ et pour $p$ suffisamment grand, $\varphi(p) > n$, donc on déduit que $u(n) = 0$. Comme $n$ est arbitraire, $u$ est la suite nulle ce qui contredit le fait que $\|u\|_1 = 1$. La boule unité fermée dans $E$ est donc fermée et bornée mais non compacte.
\end{proof}

\subsection{Normes équivalentes}

Un point important à garder à l'esprit est que la notion d'ensembles ouverts, d'applications continues, d'ensembles compacts etc. dans un espace vectoriel $E$ dépend a priori du choix d'une norme sur $E$. Deux normes différentes sur $E$ conduisent en général à deux topologies différentes.

\begin{definition}[Normes topologiquement équivalentes et équivalentes]
\label{def:norm_equivalence}
Soit $N_1, N_2$ deux normes sur un espace vectoriel $E$.
\begin{enumerate}
    \item On dit que $N_1$ et $N_2$ sont topologiquement équivalentes si $(E, N_1)$ et $(E, N_2)$ ont les mêmes ensembles ouverts.
    \item On dit que $N_1$ et $N_2$ sont équivalentes (on écrit parfois $N_1 \sim N_2$) si il existe $C_1, C_2 > 0$ telles que
    \begin{align*}
        N_1(x) &\leq C_1 N_2(x), \\
        N_2(x) &\leq C_2 N_1(x), \quad \forall x \in E.
    \end{align*}
\end{enumerate}
\end{definition}
On peut réécrire la condition (2) de la Définition~\ref{def:norm_equivalence} de manière plus élégante comme: il existe $C > 0$ tel que
\begin{equation}
    C^{-1} N_2(x) \leq N_1(x) \leq C N_2(x), \forall x \in E,
\end{equation}
(prendre $C = \max(C_1, C_2)$). La taille des constantes $C_1, C_2$ ou $C$ n'a que peu d'importance.

Il est facile de voir que si $N_1 \sim N_2$ et $N_2 \sim N_3$ alors $N_1 \sim N_3$ (la relation $\sim$ comme la relation d'équivalence topologique sont des relations d'équivalence).

\begin{theorem}
\label{thm:topo_equiv_iff_equiv}
Deux normes sur $E$ sont topologiquement équivalentes si et seulement si elles sont équivalentes.
\end{theorem}

La démonstration sera faite dans la section 6.6.3.

\begin{remark}
\label{rem:norms_equiv_in_Rn}
Il est facile de voir que les normes $\|.\|_p$ sur $\mathbb{K}^n$ sont équivalentes entre elles. En effet on vérifie (exercice) que
\begin{equation}
    \|x\|_\infty \leq \|x\|_p \leq n^{1/p} \|x\|_\infty, \quad x \in \mathbb{K}^n.
\end{equation}
On verra dans la suite un résultat a priori surprenant qui affirme que toutes les normes sur $\mathbb{K}^n$ sont équivalentes.
\end{remark}

Par contre les normes $\|.\|_p$ sur $C([a, b]; \mathbb{K})$ ne sont pas équivalentes entre elles. Par exemple si $[a, b] = [-1, 1]$ et
\begin{equation}
    f_n(x) =
    \begin{cases}
        1 - n|x| & \text{si } |x| \leq n^{-1}, \\
        0 & \text{sinon}
    \end{cases}
\end{equation}
\begin{verbatim}
#save_to: plot_fn_norm.png
import matplotlib.pyplot as plt
import numpy as np

def fn_norm(x, n):
    if abs(x) <= 1/n:
        return 1 - n*abs(x)
    else:
        return 0

x_vals = np.linspace(-1, 1, 1000)
n_vals = [1, 2, 5, 10] # Example n values

plt.figure(figsize=(8, 6))

for n in n_vals:
    f_n_vals = [fn_norm(x, n) for x in x_vals]
    plt.plot(x_vals, f_n_vals, label='$f_{}(x)$'.format(n))

plt.xlabel('x')
plt.ylabel('y')
plt.title('$f_n(x)$ functions for different n values')
plt.legend()
plt.grid(True)
plt.savefig('plot_fn_norm.png')
plt.close()

\end{verbatim}

\begin{figure}[h]
    \centering
    \includegraphics[ max width=\textwidth,
     max height=0.4\textheight,
     keepaspectratio]{plot_fn_norm.png}
    \caption{Fonctions $f_n(x)$ pour différentes valeurs de n}
    \label{fig:plot_fn_norm}
\end{figure}
(tracer son graphe), on a $\|f_n\|_\infty = 1$ et $\|f_n\|_1 = n^{-1}$. Ceci entraine qu'il n'existe pas de constante $C$ telle que $\|f\|_\infty \leq C \|f\|_1$ pour tout $f \in C([-1, 1]; \mathbb{K})$ (prendre $f=f_n$ and let $n \rightarrow \infty$).

\subsubsection{Équivalence des normes en dimension finie}

\begin{theorem}
\label{thm:norm_equiv_finite_dim}
Soit $E$ un espace vectoriel sur $\mathbb{K}$ de dimension finie. Alors $E$ possède au moins une norme et toutes les normes sur $E$ sont équivalentes.
\end{theorem}

\begin{proof}
On peut supposer que $\mathbb{K} = \mathbb{R}$ (en identifiant $\mathbb{C}$ à $\mathbb{R}^2$). On fixe alors une base $B = (e_1, ..., e_n)$ de $E$ et on identifie $E$ à $\mathbb{R}^n$ à l'aide de $B$. On peut donc supposer que $E = \mathbb{R}^n$ et que $(e_1, ..., e_n)$ est la base canonique de $\mathbb{R}^n$. La norme $\|.\|_\infty$ est une norme sur $\mathbb{R}^n$, équivalente à la norme usuelle, et définissant donc les mêmes ouverts.

Soit $S_\infty(0, 1) = \{x \in \mathbb{R}^n : \|x\|_\infty = 1\}$ la sphère de rayon $1$ pour $\|.\|_\infty$. $S_\infty(0, 1)$ est fermée, car l'application $E \ni x \mapsto \|x\| \in \mathbb{R}$ est continue sur tout evn $(E, \|.\|)$ (exercice). $S_\infty(0, 1)$ est évidemment bornée, donc compacte par le Théorème de Borel-Lebesgue.

Soit maintenant $N: \mathbb{R}^n \rightarrow \mathbb{R}^+$ une norme. On a
\begin{equation}
    N(x) = N \left( \sum_{i=1}^n x_i e_i \right) \leq \sum_{i=1}^n |x_i| N(e_i) \leq \left( \sum_{i=1}^n N(e_i) \right) \max_{1 \leq i \leq n} |x_i| = C \|x\|_\infty, \quad C = \sum_{i=1}^n N(e_i).
\end{equation}
Grâce à l'inégalité triangle pour la norme $N$ (Définition~\ref{def:evn}, propriété 2), on obtient
\begin{equation}
    |N(x) - N(y)| \leq N(x-y) \leq C \|x-y\|_\infty,
\end{equation}
et donc la fonction $N: \mathbb{R}^n \rightarrow \mathbb{R}$ est continue sur $(\mathbb{R}^n, \|.\|_\infty)$. Par le Corollaire (4.8) $N$ est bornée et atteint ses bornes sur $S_\infty(0, 1)$. Comme $N(x) > 0$ pour tout $x \in S_\infty(0, 1)$ il existe donc des constantes $0 < a < b$ telles que
\begin{equation}
    a \leq N(x) \leq b, \forall x \in S_\infty(0, 1).
\end{equation}
En appliquant cette inégalité à $\frac{x}{\|x\|_\infty}$ pour $x \neq 0$, on en déduit que
\begin{equation}
    a \leq N \left( \frac{x}{\|x\|_\infty} \right) \leq b \implies a \|x\|_\infty \leq \|x\|_\infty N \left( \frac{x}{\|x\|_\infty} \right) = N(x) \leq b \|x\|_\infty, \forall x \neq 0,
\end{equation}
et l'inégalité est aussi vraie pour $x=0$. Par conséquent, $N$ et $\|.\|_\infty$ sont équivalentes. Toutes les normes sur $\mathbb{R}^n$ sont équivalentes à $\|.\|_\infty$ et donc équivalentes entre elles.
\end{proof}

\subsection{Compléments sur les espaces vectoriels normés}

\subsubsection{Norme $L^\infty$ et convergence uniforme}

Soit $X$ un ensemble (par exemple un intervalle de $\mathbb{R}$) et $(f_n)_{n \in \mathbb{N}}$ une suite de fonctions $f_n: X \rightarrow \mathbb{R}$.

On rappelle que la suite de fonctions $(f_n)$ converge simplement vers une fonction $f: X \rightarrow \mathbb{R}$ si pour tout $x \in X$ la suite réelle $(f_n(x))$ converge vers $f(x)$. La convergence simple est la notion la plus faible de convergence pour des suites de fonctions. Exprimée à l'aide de quantificateurs, elle s'écrit:
\begin{equation}
    \forall x \in X, \forall \epsilon > 0, \exists N \in \mathbb{N} \text{ tel que } |f_n(x) - f(x)| \leq \epsilon, \forall n \geq N,
\end{equation}
l'entier $N$ dépendant a priori de $\epsilon$ et de $x$.

La suite de fonctions $(f_n)$ converge \textit{uniformément} vers une fonction $f: X \rightarrow \mathbb{R}$ si
\begin{equation}
    \forall \epsilon > 0, \exists N \in \mathbb{N} \text{ tel que } |f_n(x) - f(x)| \leq \epsilon, \forall n \geq N, \forall x \in X.
\end{equation}
L'entier $N$ ne dépend maintenant que de $\epsilon$. On peut réécrire (\ref{eq:6.4}) comme
\begin{equation}
    \forall \epsilon > 0, \exists N \in \mathbb{N} \text{ tel que } \sup_{x \in X} |f_n(x) - f| \leq \epsilon.
\end{equation}
On voit donc que sur l'espace vectoriel $\mathcal{B}(X, \mathbb{R})$ des fonctions bornées sur $X$, une suite $(f_n)$ converge uniformément vers $f$ si et seulement si $(f_n)$ converge vers $f$ pour la norme $\|.\|_\infty$.

\subsubsection{Limites uniformes de fonctions continues}

Prenons $X = [a, b]$. L'espace $C([a, b]; \mathbb{R})$ des fonctions continues sur $[a, b]$ est un sous-espace vectoriel de l'espace $\mathcal{B}([a, b]; \mathbb{R})$ des fonctions bornées. On sait que la limite uniforme d'une suite de fonctions continues sur $[a, b]$ est une fonction continue sur $[a, b]$.

Avec le langage que nous avons appris, ceci signifie que $C([a, b]; \mathbb{R})$ est un sous-espace vectoriel fermé de $\mathcal{B}([a, b]; \mathbb{R})$ muni de la norme $\|.\|_\infty$.

\subsubsection{Séries à valeurs dans un espace vectoriel normé}

Soit $(E, \|.\|)$ un espace vectoriel normé et $(u_n)_{n \in \mathbb{N}}$ une suite dans $E$. On peut former la suite
\begin{equation}
    S_N = \sum_{n=0}^N u_n,
\end{equation}
appelée suite des sommes partielles de la série $\sum_{n \in \mathbb{N}} u_n$. La définition suivante est exactement la même que pour les séries numériques.

\begin{definition}[Convergence d'une série dans un EVN]
\label{def:series_convergence_evn}
La série $\sum_{n \in \mathbb{N}} u_n$ converge dans $(E, \|.\|)$ si la suite des sommes partielles $(S_N)_{N \in \mathbb{N}}$ converge dans $(E, \|.\|)$. L'élément $\lim_{N \rightarrow +\infty} S_N = \sum_{n=0}^\infty u_n$ est noté $\sum_{n=0}^\infty u_n$ et appelé somme de la série $\sum_{n \in \mathbb{N}} u_n$.
\end{definition}

\begin{proposition}
\label{prop:series_linearity}
Soit $\sum_{n \in \mathbb{N}} u_n$, $\sum_{n \in \mathbb{N}} v_n$ deux séries convergentes à valeurs dans $E$ et $\lambda \in \mathbb{K}$. Alors $\sum_{n \in \mathbb{N}} (u_n + v_n)$ et $\sum_{n \in \mathbb{N}} (\lambda u_n)$ sont convergentes et on a:
\begin{align*}
    \sum_{n=0}^\infty (u_n + v_n) &= \sum_{n=0}^\infty u_n + \sum_{n=0}^\infty v_n, \\
    \sum_{n=0}^\infty (\lambda u_n) &= \lambda \sum_{n=0}^\infty u_n.
\end{align*}
\end{proposition}

\subsubsection{Convergence normale}

\begin{definition}[Convergence normale]
\label{def:normal_convergence}
Soit $(E, \|.\|)$ un espace vectoriel normé et $(u_n)_{n \in \mathbb{N}}$ une suite dans $E$. La série $\sum_{n \in \mathbb{N}} u_n$ converge normalement si la série numérique $\sum_{n \in \mathbb{N}} \|u_n\|$ converge.
\end{definition}

La convergence normale est un outil important pour montrer la convergence d'une série à valeurs dans un espace de Banach. On peut retenir sa définition en pensant à la 'convergence de la série des normes'.

\begin{theorem}
\label{thm:normal_convergence_implies_convergence}
Soit $(E, \|.\|)$ un espace de Banach. Alors toute série $\sum_{n \in \mathbb{N}} u_n$ normalement convergente est convergente et on a
\begin{equation}
    \left\| \sum_{n=0}^\infty u_n \right\| \leq \sum_{n=0}^\infty \|u_n\|.
\end{equation}
\end{theorem}

\section{Suites d'éléments de E}
\subsection{Suite extraite}
\textbf{Exemple:} $E = (\mathcal{C}([0, 1], \mathbb{R}), \|.\|_\infty)$. On construit une suite d'éléments de $C([0, 1], \mathbb{R})$ qui n'est pas compacte.

\begin{lemma}
\label{lemma:C_complete}
Soit $E = (C([0, 1], \mathbb{R}), \|.\|_\infty)$. $E$ est complet.
\end{lemma}

\textbf{Exemple}
On construit une suite d'éléments de $B_f(0, 1)$ qui n'admet pas de sous-suite convergente.

Pour $n \in \mathbb{N}^*$, on pose $f_n : [0, 1] \rightarrow \mathbb{R}$ continue affine par morceaux :
\begin{itemize}
    \item $f_n(0) = 0$
    \item $f_n(\frac{1}{2n}) = 1$
    \item $f_n(\frac{1}{n}) = 0$
    \item $f_n(x) = 0$ pour $x \geq \frac{1}{n}$ ou $x \leq 0$.
\end{itemize}

\begin{verbatim}
#save_to: plot_fn_suite.png
import matplotlib.pyplot as plt
import numpy as np

def fn_suite(x, n):
    if x < 0 or x > 1/n:
        return 0
    elif 0 <= x <= 1/(2*n):
        return 2*n*x
    elif 1/(2*n) < x <= 1/n:
        return 2 - 2*n*x
    else:
        return 0

x_vals = np.linspace(-0.1, 0.2, 1000)
n_vals = [1, 2, 3, 4, 5] # Example n values


plt.figure(figsize=(8, 6))

for n in n_vals:
    f_n_vals = [fn_suite(x, n) for x in x_vals]
    plt.plot(x_vals, f_n_vals, label='$f_{}(x)$'.format(n))

plt.xlabel('x')
plt.ylabel('y')
plt.title('$f_n(x)$ functions for different n values')
plt.legend()
plt.grid(True)
plt.savefig('plot_fn_suite.png')
plt.close()

\end{verbatim}

\begin{figure}[h]
    \centering
    \includegraphics[ max width=\textwidth,
     max height=0.4\textheight,
     keepaspectratio]{plot_fn_suite.png}
    \caption{Fonctions $f_n(x)$ pour différentes valeurs de n}
    \label{fig:plot_fn_suite}
\end{figure}
On remarque que $\forall n$, $\|f_n\|_\infty = 1$ et que si $n \neq p$, $f_n$ et $f_p$ ont des supports disjoints.

Si $n \neq p$, $\|f_n - f_p\|_\infty = 1$.

Donc $(f_n)_{n \geq 1}$ ne peut pas avoir de sous-suite de Cauchy, car si $\varphi : \mathbb{N} \rightarrow \mathbb{N}$ est strictement croissante, $\forall p \neq q$, $\|f_{\varphi(p)} - f_{\varphi(q)}\|_\infty = 1$.

Donc $(f_n)_{n \geq 1}$ n'a pas de sous-suite convergente. $B_f(0, 1)$ n'est pas compacte.

\section{Normes Equivalentes}

\begin{definition}[Normes topologiquement équivalentes et équivalentes]
$N_1, N_2$ sont topologiquement équivalentes sur $E$ si
\begin{align*}
    Id : (E, N_1) &\rightarrow (E, N_2) \text{ est continue} \\
    Id : (E, N_2) &\rightarrow (E, N_1) \text{ est continue}
\end{align*}
$\iff \exists C_1, C_2 > 0$ telles que
\begin{align*}
    N_2(x) &\leq C_1 N_1(x) \\
    N_1(x) &\leq C_2 N_2(x)
\end{align*}
$\iff \exists C > 0$ telles que
\begin{equation}
    \frac{1}{C} N_1(x) \leq N_2(x) \leq C N_1(x).
\end{equation}
Dans ce cas, on dit que $N_1$ et $N_2$ sont des normes équivalentes.
\end{definition}

\begin{theorem}
\label{thm:norm_equiv_finite_dim_again}
En dimension finie, toutes les normes sont équivalentes.
\end{theorem}

\begin{example}
Dans $\mathbb{R}^2$, $N_1(x) = |x_1| + |x_2|$, $N_2(x) = \sqrt{x_1^2 + x_2^2}$, $N_\infty(x) = \max(|x_1|, |x_2|)$.
\end{example}

\begin{verbatim}
#save_to: plot_norm_equivalence.png
import matplotlib.pyplot as plt
import matplotlib.patches as patches
import numpy as np

fig, ax = plt.subplots()

# Square for N_inf
square = patches.Rectangle((-1, -1), 2, 2, linewidth=1, edgecolor='r', facecolor='none', label='$N_\infty(x) = 1$')
ax.add_patch(square)

# Circle for N_2
circle = patches.Circle((0, 0), radius=1, linewidth=1, edgecolor='g', facecolor='none', label='$N_2(x) = 1$')
ax.add_patch(circle)

# Diamond for N_1
diamond_points = np.array([[0, 1], [1, 0], [0, -1], [-1, 0]])
diamond = patches.Polygon(diamond_points, linewidth=1, edgecolor='b', facecolor='none', label='$N_1(x) = 1$')
ax.add_patch(diamond)


ax.set_aspect('equal', adjustable='box')
ax.set_xlim([-1.5, 1.5])
ax.set_ylim([-1.5, 1.5])
ax.set_xlabel('x')
ax.set_ylabel('y')
ax.set_title('Equivalence of Norms in $\mathbb{R}^2$')
ax.legend()
ax.grid(True)
plt.savefig('plot_norm_equivalence.png')
plt.close()


\end{verbatim}

\begin{figure}[h]
    \centering
    \includegraphics[ max width=\textwidth,
     max height=0.4\textheight,
     keepaspectratio]{plot_norm_equivalence.png}
    \caption{Equivalence des Normes dans $\mathbb{R}^2$}
    \label{fig:plot_norm_equivalence}
\end{figure}
\begin{itemize}
    \item $N_\infty(x) \leq N_2(x) \leq \sqrt{2} N_\infty(x)$
    \item $N_\infty(x) \leq N_1(x) \leq 2 N_\infty(x)$
    \item $N_2(x) \leq N_1(x) \leq \sqrt{2} N_2(x)$
\end{itemize}
Il existe un isomorphisme entre $(\mathbb{R}^2, N_1)$ et $(\mathbb{R}^2, N_2)$.

$\mathbb{R}^2 \sim \mathbb{C}$. On identifie $\mathbb{R}^2$ et $\mathbb{C}$.
$z = x_1 + ix_2 \leftrightarrow (x_1, x_2)$.
$\|z\| = \sqrt{x_1^2 + x_2^2} = N_2((x_1, x_2))$.

\textbf{Soit $N$ une norme sur $\mathbb{R}^d$.}
On veut montrer que $\exists a, b > 0$ telles que $a N_\infty(x) \leq N(x) \leq b N_\infty(x)$.
On pose $S_\infty = \{x \in \mathbb{R}^d : N_\infty(x) = 1\}$.
$S_\infty$ est compacte.
$N$ est continue sur $(\mathbb{R}^d, N_\infty)$.
$N$ est continue sur $S_\infty$.
$N$ est bornée et atteint ses bornes sur $S_\infty$.
$\exists x_0 \in S_\infty$ telle que $N(x_0) = \min_{x \in S_\infty} N(x) = a$.
$\exists x_1 \in S_\infty$ telle que $N(x_1) = \max_{x \in S_\infty} N(x) = b$.

$a = \min_{x \in S_\infty} N(x) > 0$ car si $N(x_0) = 0$ avec $N_\infty(x_0) = 1 \implies x_0 = 0$ et $N_\infty(x_0) = 0 \neq 1$. Contradiction. Donc $a > 0$.
$b = \max_{x \in S_\infty} N(x) < \infty$ car $N$ continue sur le compact $S_\infty$.

Soit $x \in \mathbb{R}^d$, $x \neq 0$. On pose $y = \frac{x}{N_\infty(x)}$. $N_\infty(y) = 1 \implies y \in S_\infty$.
$a \leq N(y) \leq b$.
$a \leq N \left( \frac{x}{N_\infty(x)} \right) \leq b$.
$a \leq \frac{N(x)}{N_\infty(x)} \leq b$.
$a N_\infty(x) \leq N(x) \leq b N_\infty(x)$.

\begin{theorem}
\label{thm:topo_equiv_iff_equiv_again}
$N_1, N_2$ sont topologiquement équivalentes si et seulement si $N_1, N_2$ sont équivalentes.
\end{theorem}

\textbf{Exercice:} $E = C([0, 1], \mathbb{R})$. Comparer $\|.\|_1$ et $\|.\|_\infty$.
$\|f\|_1 = \int_0^1 |f(t)| dt$.
$\|f\|_\infty = \sup_{t \in [0, 1]} |f(t)|$.
$\|f\|_1 \leq \|f\|_\infty$.
$\|f\|_\infty \leq C \|f\|_1$? Faux.

On prend $f_n$ (triangles). $\|f_n\|_\infty = 1$ et $\|f_n\|_1 = 1/n$.
$\|f_n\|_\infty = n \|f_n\|_1$.
On ne peut pas majorer $\|f\|_\infty$ par $C \|f\|_1$ avec $C$ constante indépendante de $f$.
Donc $\|.\|_1$ et $\|.\|_\infty$ ne sont pas équivalentes.

\end{document}
```