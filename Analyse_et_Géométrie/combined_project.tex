\documentclass[oneside]{book}
\usepackage{amssymb,amsmath,amsthm, thmtools}
\usepackage{graphicx}
\usepackage{tikz}
\usepackage{pgfplots}
\usepackage{color}
\usepackage{float}
\usepackage{fancyhdr}
\usepackage[Sonny]{fncychap}
\usetikzlibrary{arrows}
\usepackage{listings}
\usepackage[margin=1in]{geometry}
\usetikzlibrary{shapes.geometric}
\usepackage[utf8]{inputenc}
\pgfplotsset{compat=1.16}

\definecolor{GoodGreen}{rgb}{0.0667, 0.2078, 0.2157}
\definecolor{GreenBackground}{rgb}{0.5176, 0.6902, 0.5098}
\definecolor{DarkPurple}{rgb}{0.4157, 0.1804, 0.2078}
\definecolor{RedBackground}{rgb}{0.9490, 0.7333, 0.7529}
\definecolor{OxfordBlue}{rgb}{0.0, 0.1333, 0.2667}
\definecolor{BlueBackground}{rgb}{0.4471, 0.6314, 0.8980}

\definecolor{Gold}{rgb}{0.9176, 0.6667, 0.0784} % Define the color



\declaretheoremstyle[
    headfont=\bfseries\sffamily\color{GoodGreen}, bodyfont=\normalfont,
    mdframed={
        linewidth=2pt,
        rightline=false, topline=false, bottomline=false,
        linecolor=GoodGreen, backgroundcolor=GreenBackground!30!white,
    }
]{thmgreenbox}

\declaretheoremstyle[
    headfont=\bfseries\sffamily\color{OxfordBlue}, bodyfont=\normalfont,
    mdframed={
        linewidth=2pt,
        rightline=false, topline=false, bottomline=false,
        linecolor=OxfordBlue, backgroundcolor=BlueBackground!15
    }
]{thmblueline}

\declaretheoremstyle[
    headfont=\bfseries\sffamily\color{DarkPurple}, bodyfont=\normalfont,
    mdframed={
        linewidth=2pt,
        rightline=false, topline=false, bottomline=false,
        linecolor=DarkPurple, backgroundcolor=RedBackground!40,
    }
]{thmredbox}

\declaretheoremstyle[
     headfont=\bfseries\sffamily\color{Gold}, bodyfont=\normalfont,
    mdframed={
        linewidth=2pt,
        rightline=false, topline=false, bottomline=false,
        linecolor=Gold, backgroundcolor=Gold!10
    }
]{rmrk}


\declaretheoremstyle[
    headfont=\bfseries\sffamily\color{OxfordBlue}, bodyfont=\normalfont,
    % unnumbered=true,
    mdframed={
        linewidth=2pt,
        rightline=false, topline=false, bottomline=false,
        linecolor=OxfordBlue,backgroundcolor=BlueBackground!15
    },
    qed=\qedsymbol
]{thmproofbox}

% Consistent theorem styling:
\declaretheorem[numberwithin=chapter, style=thmgreenbox, name=Definition]{definition}
\declaretheorem[sibling=definition, style=thmredbox, name=Theorem]{theorem}
\declaretheorem[sibling=definition, style=thmredbox, name=Lemma]{lemma}
\declaretheorem[sibling=definition, style=thmredbox, name=Proposition]{proposition}
\declaretheorem[sibling=definition, style=rmrk, name=Remark]{remark}
\declaretheorem[sibling=definition, style=thmblueline, name=Example]{example}
\declaretheorem[unnumbered=true, style=thmproofbox, name=Solution]{solution}
\declaretheorem[unnumbered=true, style=thmproofbox, name=Preuve]{prf}


\renewcommand{\proof}{
\begin{prf}}
\renewcommand{\endproof}{\end{prf}}

\begin{document}
\sloppy
\chapter{CM1}
\sloppy

\section{Analyse}

\section{Introduction aux espaces vectoriels $\mathbb{R}^d$ et $\mathbb{C}^d$}

\subsection{Définitions et propriétés fondamentales}

Nous allons définir les espaces vectoriels $\mathbb{R}^d$ et $\mathbb{C}^d$.

\begin{definition}
L'espace $\mathbb{R}^d$, pour $d \geq 1$, est défini comme l'ensemble des $d$-uplets de nombres réels :
\[
\mathbb{R}^d := \{X = (x_1, \ldots, x_d) : x_i \in \mathbb{R}\}
\]
où $x_1, \ldots, x_d$ sont les coordonnées cartésiennes du point $X$.
\end{definition}

Pour $d=2$, on note parfois les points de $\mathbb{R}^2$ par $(x, y)$, et pour $d=3$, par $(x, y, z)$.

\begin{definition}
L'espace $\mathbb{C}^d$, pour $d \geq 1$, est défini de manière analogue comme l'ensemble des $d$-uplets de nombres complexes :
\[
\mathbb{C}^d := \{X = (x_1, \ldots, x_d) : x_i \in \mathbb{C}\}
\]
où $x_1, \ldots, x_d$ sont des nombres complexes.
\end{definition}

Tout nombre complexe $z \in \mathbb{C}$ peut être écrit sous la forme $z = a + bi$, où $a = Re(z)$ est la partie réelle de $z$ et $b = Im(z)$ est la partie imaginaire de $z$. On peut écrire pour $X \in \mathbb{C}^d$,
\[
X = Re(X) + i Im(X)
\]
où $Re(X) = (Re(x_1), \ldots, Re(x_d)) \in \mathbb{R}^d$ et $Im(X) = (Im(x_1), \ldots, Im(x_d)) \in \mathbb{R}^d$.

$\mathbb{R}^d$ et $\mathbb{C}^d$ sont des espaces vectoriels. L'addition vectorielle et la multiplication par un scalaire sont définies de la manière suivante dans $\mathbb{R}^d$ (et de même dans $\mathbb{C}^d$) :
Pour $X = (x_1, \ldots, x_d) \in \mathbb{R}^d$, $Y = (y_1, \ldots, y_d) \in \mathbb{R}^d$ et $\lambda \in \mathbb{R}$,
\begin{itemize}
    \item \textbf{Addition vectorielle} : $X + Y = (x_1 + y_1, \ldots, x_d + y_d)$
    \item \textbf{Multiplication par un scalaire} : $\lambda X = (\lambda x_1, \ldots, \lambda x_d)$
    \item \textbf{Vecteur nul} : $\overrightarrow{0} = (0, \ldots, 0)$
\end{itemize}
Pour $\mathbb{C}^d$, le corps des scalaires est $\mathbb{C}$.

\subsection{Coordonnées polaires et sphériques}

Dans $\mathbb{R}^2$, on peut utiliser les coordonnées polaires $(r, \theta) \in \mathbb{R}^+ \times [0, 2\pi[$ définies par :
\begin{align*}
    x &= r \cos \theta \\
    y &= r \sin \theta
\end{align*}
avec $r = \sqrt{x^2 + y^2}$.

Dans $\mathbb{R}^3$, on peut utiliser les coordonnées sphériques $(r, \theta, \varphi) \in \mathbb{R}^+ \times [-\pi/2, \pi/2] \times [0, 2\pi[$ définies par :
\begin{align*}
    x &= r \sin \theta \cos \varphi \\
    y &= r \sin \theta \sin \varphi \\
    z &= r \cos \theta
\end{align*}
avec $r = \sqrt{x^2 + y^2 + z^2}$.

\subsection{Représentation graphique de $\mathbb{R}^3$}

On peut représenter un point $X = (x, y, z)$ de $\mathbb{R}^3$ à l'aide des coordonnées sphériques.
Dans la figure ci-dessous, $\theta$ est l'angle entre l'axe $z$ et le vecteur $OX$, et $\varphi$ est l'angle entre l'axe $x$ et la projection de $OX$ sur le plan $(xOy)$.

\begin{verbatim}
```python
#save_to: R3_coordinates.png
import matplotlib.pyplot as plt
import matplotlib.patches as patches
import numpy as np

fig = plt.figure(figsize=(6,6))
ax = fig.add_subplot(111, projection='3d')

# Axes
ax.quiver(0, 0, 0, 1, 0, 0, color='k', arrow_length_ratio=0.05, linestyle='--')
ax.quiver(0, 0, 0, 0, 1, 0, color='k', arrow_length_ratio=0.05, linestyle='--')
ax.quiver(0, 0, 0, 0, 0, 1, color='k', arrow_length_ratio=0.05, linestyle='--')
ax.text(1.1, 0, 0, 'x')
ax.text(0, 1.1, 0, 'y')
ax.text(0, 0, 1.1, 'z')

# Point X
x, y, z = 0.8, 0.5, 0.6
ax.scatter(x, y, z, color='red', s=50)
ax.text(x*1.1, y*1.1, z*1.1, 'X')

# Vector OX
ax.quiver(0, 0, 0, x, y, z, color='blue', arrow_length_ratio=0.1)

# Projection on xy plane
ax.plot([0, x], [0, y], [0, 0], linestyle='--', color='gray')
ax.scatter(x, y, 0, color='gray', s=30)

# Angles
r = np.sqrt(x**2 + y**2 + z**2)
theta = np.arccos(z/r)
phi = np.arctan2(y, x)

# Arc for theta
arc_theta_radius = 0.5
arc_theta_start = np.arctan2(np.sqrt(x**2+y**2), z)
arc_theta = patches.Arc((0,0,0), arc_theta_radius, arc_theta_radius, 0, 0, np.degrees(theta))
ax.add_patch(arc_theta)
from mpl_toolkits.mplot3d import art3d
art3d.pathpatch_2d_to_3d(arc_theta, z=0, zdir="y")
ax.text(arc_theta_radius*0.6, -0.1, arc_theta_radius*0.6, r'$\theta$')


# Arc for phi
arc_phi_radius = 0.4
arc_phi_start = 0
arc_phi_end = np.degrees(phi)
arc_phi = patches.Arc((0,0,0), arc_phi_radius, arc_phi_radius, 0, 0, arc_phi_end)
ax.add_patch(arc_phi)
art3d.pathpatch_2d_to_3d(arc_phi, z=0, zdir="z")
ax.text(arc_phi_radius*0.7, arc_phi_radius*0.7, 0, r'$\varphi$')


ax.set_xlabel("x")
ax.set_ylabel("y")
ax.set_zlabel("z")
ax.set_xlim([-0.2, 1.2])
ax.set_ylim([-0.2, 1.2])
ax.set_zlim([-0.2, 1.2])
ax.view_init(elev=20, azim=-135)
plt.savefig('R3_coordinates.png')
```
\end{verbatim}

\begin{figure}[h]
    \centering
    \includegraphics[width=0.7\textwidth]{R3_coordinates.png}
    \caption{Représentation des coordonnées sphériques dans $\mathbb{R}^3$}
    \label{fig:R3_coordinates}
\end{figure}


\subsection{Produit scalaire dans $\mathbb{R}^d$}

\begin{definition}
Le produit scalaire de deux vecteurs $X = (x_1, \ldots, x_d) \in \mathbb{R}^d$ et $Y = (y_1, \ldots, y_d) \in \mathbb{R}^d$ est défini par :
\[
X \cdot Y = \sum_{i=1}^d x_i y_i
\]
\end{definition}

\subsection{Propriétés du produit scalaire dans $\mathbb{R}^d$}

Le produit scalaire dans $\mathbb{R}^d$ possède les propriétés suivantes :

\begin{enumerate}
    \item \textbf{Bilinéarité} :
    Pour tous vecteurs $X, Y, Z \in \mathbb{R}^d$ et scalaire $\lambda \in \mathbb{R}$,
    \begin{align*}
        (X + Y) \cdot Z &= X \cdot Z + Y \cdot Z \\
        Z \cdot (X + Y) &= Z \cdot X + Z \cdot Y \\
        (\lambda X) \cdot Y &= X \cdot (\lambda Y) = \lambda (X \cdot Y)
    \end{align*}
    \item \textbf{Symétrie} :
    Pour tous vecteurs $X, Y \in \mathbb{R}^d$,
    \[
    X \cdot Y = Y \cdot X
    \]
    \item \textbf{Positivité et caractère défini} :
    Pour tout vecteur $X \in \mathbb{R}^d$,
    \[
    X \cdot X \geq 0 \quad \text{et} \quad (X \cdot X = 0 \iff X = \overrightarrow{0})
    \]
\end{enumerate}

\begin{proposition}[Inégalité de Cauchy-Schwarz]
Pour tous vecteurs $X, Y \in \mathbb{R}^d$, on a l'inégalité de Cauchy-Schwarz :
\[
|X \cdot Y| \leq \sqrt{(X \cdot X)} \sqrt{(Y \cdot Y)}
\]
\end{proposition}

\begin{proof}
Fixons $X, Y \in \mathbb{R}^d$ et considérons la fonction polynomiale $P : \mathbb{R} \to \mathbb{R}$ définie par
\[
P(t) = (X + tY) \cdot (X + tY)
\]
En développant, on obtient
\[
P(t) = t^2 (Y \cdot Y) + 2t (X \cdot Y) + (X \cdot X)
\]
Comme $P(t) = ||X + tY||^2 \geq 0$ pour tout $t \in \mathbb{R}$, le polynôme $P(t)$ de degré 2 est toujours positif ou nul. Son discriminant $\Delta$ est donc négatif ou nul :
\[
\Delta = (2(X \cdot Y))^2 - 4 (Y \cdot Y) (X \cdot X) = 4 (X \cdot Y)^2 - 4 (X \cdot X) (Y \cdot Y) \leq 0
\]
D'où $(X \cdot Y)^2 \leq (X \cdot X) (Y \cdot Y)$, et en prenant la racine carrée, on obtient l'inégalité de Cauchy-Schwarz.
\end{proof}


\section{Normes}

\subsection{Norme associée au produit scalaire dans $\mathbb{R}^d$}

\begin{definition}
La norme euclidienne associée au produit scalaire dans $\mathbb{R}^d$ est définie par :
\[
\|X\| = \sqrt{X \cdot X} = \sqrt{\sum_{i=1}^d x_i^2}
\]
pour $X = (x_1, \ldots, x_d) \in \mathbb{R}^d$.
\end{definition}
Cette norme représente la longueur du vecteur $X$. Elle est parfois notée $\|X\|_2$.

\subsection{Propriétés des normes}

Les normes possèdent les propriétés suivantes :

\begin{enumerate}
    \item \textbf{Homogénéité} : Pour tout vecteur $X \in \mathbb{R}^d$ et scalaire $\lambda \in \mathbb{R}$,
    \[
    \|\lambda X\| = |\lambda| \|X\|
    \]
    \item \textbf{Inégalité triangulaire} : Pour tous vecteurs $X, Y \in \mathbb{R}^d$,
    \[
    \|X + Y\| \leq \|X\| + \|Y\|
    \]
    \item \textbf{Positivité et caractère défini} : Pour tout vecteur $X \in \mathbb{R}^d$,
    \[
    \|X\| \geq 0 \quad \text{et} \quad (\|X\| = 0 \iff X = \overrightarrow{0})
    \]
\end{enumerate}

\subsection{Définition générale d'une norme}

\begin{definition}
Une norme sur $\mathbb{R}^d$ est une application $N : \mathbb{R}^d \to \mathbb{R}^+$ satisfaisant les propriétés suivantes pour tous $X, Y \in \mathbb{R}^d$ et $\lambda \in \mathbb{R}$ :
\begin{enumerate}
    \item $N(\lambda X) = |\lambda| N(X)$
    \item $N(X + Y) \leq N(X) + N(Y)$
    \item $N(X) \geq 0$ et $(N(X) = 0 \iff X = \overrightarrow{0})$
\end{enumerate}
\end{definition}

\subsection{Exemples de normes sur $\mathbb{R}^d$}

Outre la norme euclidienne $\|X\| = \|X\|_2$, on peut définir d'autres normes sur $\mathbb{R}^d$ :

\begin{enumerate}
    \item \textbf{Norme 1} : $\|X\|_1 = \sum_{i=1}^d |x_i|$
    \item \textbf{Norme infini} : $\|X\|_\infty = \max_{1 \leq i \leq d} |x_i|$
\end{enumerate}


\section{Espace $\mathbb{C}^d$}

\subsection{Définition et produit scalaire hermitien}

Nous rappelons la définition de l'espace vectoriel $\mathbb{C}^d$ sur $\mathbb{C}$ :
\[
\mathbb{C}^d = \{X = (x_1, \ldots, x_d) : x_i \in \mathbb{C}\}
\]

\begin{definition}
Le produit scalaire hermitien de deux vecteurs $X = (x_1, \ldots, x_d) \in \mathbb{C}^d$ et $Y = (y_1, \ldots, y_d) \in \mathbb{C}^d$ est défini par :
\[
(X|Y) = \sum_{i=1}^d x_i \overline{y_i}
\]
\end{definition}
Notez la conjugaison complexe sur les composantes de $Y$. Certains auteurs utilisent la conjugaison sur les composantes de $X$.

\subsection{Propriétés du produit scalaire hermitien dans $\mathbb{C}^d$}

Le produit scalaire hermitien dans $\mathbb{C}^d$ possède les propriétés suivantes :

\begin{enumerate}
    \item \textbf{Sesquilinéarité} :
    Pour tous vecteurs $X, Y, Z \in \mathbb{C}^d$ et scalaires $\lambda, \mu \in \mathbb{C}$,
    \begin{align*}
        (X + Y|Z) &= (X|Z) + (Y|Z) \\
        (\lambda X|Z) &= \lambda (X|Z) \\
        (Z|X + Y) &= (Z|X) + (Z|Y) \\
        (Z|\lambda Y) &= \overline{\lambda} (Z|Y)
    \end{align*}
    On dit que le produit scalaire est linéaire par rapport au premier argument (à gauche) et antilinéaire par rapport au deuxième argument (à droite). Certains auteurs choisissent la convention inverse.

    \item \textbf{Hermiticité (ou symétrie hermitienne)} :
    Pour tous vecteurs $X, Y \in \mathbb{C}^d$,
    \[
    (X|Y) = \overline{(Y|X)}
    \]

    \item \textbf{Positivité et caractère défini} :
    Pour tout vecteur $X \in \mathbb{C}^d$,
    \[
    (X|X) = \sum_{i=1}^d |x_i|^2 \geq 0 \quad \text{et} \quad ((X|X) = 0 \iff X = \overrightarrow{0})
    \]
\end{enumerate}


\section{Norme Hilbertienne dans $\mathbb{C}^d$}

\begin{definition}
La norme Hilbertienne (ou norme euclidienne) associée au produit scalaire hermitien dans $\mathbb{C}^d$ est définie par :
\[
\|X\| = \sqrt{(X|X)} = \sqrt{\sum_{i=1}^d |x_i|^2}
\]
pour $X = (x_1, \ldots, x_d) \in \mathbb{C}^d$.
\end{definition}
On a $\|X\|^2 = \sum_{i=1}^d |x_i|^2 = \|Re(X)\|^2 + \|Im(X)\|^2$.

\begin{lemma}
Pour tout $X \in \mathbb{C}^d$, on a :
\[
\|X\| = \sup_{\|Y\| \leq 1} |(X|Y)|
\]
\end{lemma}

\begin{proof}
Par l'inégalité de Cauchy-Schwarz appliquée au produit scalaire hermitien (qui est aussi valable dans $\mathbb{C}^d$), on a
\[
|(X|Y)| \leq \sqrt{(X|X)} \sqrt{(Y|Y)} = \|X\| \|Y\|
\]
Si $\|Y\| \leq 1$, alors $|(X|Y)| \leq \|X\|$. Donc $\sup_{\|Y\| \leq 1} |(X|Y)| \leq \|X\|$.

Inversement, considérons $X \neq \overrightarrow{0}$ (si $X = \overrightarrow{0}$, l'égalité est triviale). Posons $Y = \frac{X}{\|X\|}$. Alors $\|Y\| = 1$ et
\[
(X|Y) = \left(X \Big| \frac{X}{\|X\|} \right) = \frac{1}{\|X\|} (X|X) = \frac{\|X\|^2}{\|X\|} = \|X\|
\]
Donc, il existe $Y$ avec $\|Y\| \leq 1$ tel que $|(X|Y)| = \|X\|$. D'où $\sup_{\|Y\| \leq 1} |(X|Y)| \geq \|X\|$.

En conclusion, on a bien l'égalité $\|X\| = \sup_{\|Y\| \leq 1} |(X|Y)|$.
\end{proof}

\subsection{Autres normes sur $\mathbb{C}^d$}

De même que pour $\mathbb{R}^d$, on peut définir les normes $\|\cdot\|_1$ et $\|\cdot\|_\infty$ sur $\mathbb{C}^d$ :
\begin{itemize}
    \item Norme 1 : $\|X\|_1 = \sum_{i=1}^d |x_i|$
    \item Norme infini : $\|X\|_\infty = \max_{1 \leq i \leq d} |x_i|$
\end{itemize}


\section{Distances}

\subsection{Distance induite par une norme}

\begin{definition}
Soit $\|\cdot\|$ une norme sur $\mathbb{R}^d$ (ou $\mathbb{C}^d$). La distance induite par cette norme entre deux points $X, Y$ est définie par :
\[
d(X, Y) = \|Y - X\|
\]
\end{definition}

\subsection{Propriétés d'une distance}

Une distance $d : E \times E \to \mathbb{R}^+$ sur un ensemble $E$ doit satisfaire les propriétés suivantes :

\begin{definition}
Une application $d: E \times E \to \mathbb{R}^+$ est une distance sur $E$ si elle satisfait pour tous $x, y, z \in E$:
\begin{enumerate}
    \item \textbf{Positivité} : $d(x, y) \geq 0$
    \item \textbf{Symétrie} : $d(x, y) = d(y, x)$
    \item \textbf{Inégalité triangulaire} : $d(x, y) \leq d(x, z) + d(z, y)$
    \item \textbf{Séparation} : $d(x, y) = 0 \iff x = y$
\end{enumerate}
\end{definition}

\subsection{Exemples de distances}

\begin{enumerate}
    \item \textbf{Distance euclidienne} (ou distance $d_2$) : induite par la norme euclidienne $\|\cdot\|_2$.
    \[
    d_2(X, Y) = \|X - Y\|_2 = \sqrt{\sum_{i=1}^d (x_i - y_i)^2}
    \]
    \item \textbf{Distance $d_1$} : induite par la norme $\|\cdot\|_1$.
    \[
    d_1(X, Y) = \|X - Y\|_1 = \sum_{i=1}^d |x_i - y_i|
    \]
    \item \textbf{Distance $d_\infty$} : induite par la norme $\|\cdot\|_\infty$.
    \[
    d_\infty(X, Y) = \|X - Y\|_\infty = \max_{1 \leq i \leq d} |x_i - y_i|
    \]
    \item \textbf{Distance logarithmique} sur $\mathbb{R}^+ = ]0, +\infty[$ :
    \[
    d_{log}(a, b) = |\ln(b/a)| = |\ln(b) - \ln(a)|
    \]
    En base 10 : $d_{log_{10}}(x, y) = |\log_{10}(y/x)|$.

    \item \textbf{Distance SNCF} sur un ensemble de points. Si on considère trois points $0, X, Y$ non alignés :

\begin{verbatim}
```python
#save_to: SNCF_distance.png
import matplotlib.pyplot as plt

fig, ax = plt.subplots()

points = {'O': (0, 0), 'X': (2, 0), 'Y': (2, 2)}

# Draw points
for name, coord in points.items():
    ax.plot(*coord, 'o', markersize=8, label=name)
    ax.annotate(name, coord, textcoords="offset points", xytext=(0,5), ha='center')

# Connect O-X and O-Y
ax.plot([points['O'][0], points['X'][0]], [points['O'][1], points['X'][1]], 'k-')
ax.plot([points['O'][0], points['Y'][0]], [points['O'][1], points['Y'][1]], 'k-')

ax.set_xlim([-1, 4])
ax.set_ylim([-1, 3])
ax.set_aspect('equal', adjustable='box')
ax.axis('off')
plt.savefig('SNCF_distance.png')
```
\end{verbatim}

\begin{figure}[H]
    \centering
    \includegraphics[width=0.3\textwidth]{SNCF_distance.png}
    \caption{Distance SNCF}
    \label{fig:sncf_distance}
\end{figure}
    La distance SNCF entre $X$ et $Y$ est définie par
    \[
    d_{SNCF}(X, Y) = d(0, X) + d(0, Y)
    \]

    \item \textbf{Distance usuelle dans $\mathbb{R}^2$} :
    \[
    \delta(X, Y) =
    \begin{cases}
        d(X, Y) & \text{si } 0, X, Y \text{ alignés} \\
        d(0, X) + d(0, Y) & \text{sinon}
    \end{cases}
    \]
\end{enumerate}


\section{Espaces métriques}

\begin{definition}
Un espace métrique est un couple $(E, d)$ où $E$ est un ensemble et $d : E \times E \to \mathbb{R}^+$ est une distance sur $E$.
\end{definition}

\subsection{Inégalité triangulaire inverse}
Dans un espace métrique $(E, d)$, on a l'inégalité triangulaire inverse :
Pour tous $x, y, z \in E$,
\[
|d(x, z) - d(y, z)| \leq d(x, y)
\]

\subsection{Distance induite sur un sous-ensemble}
Si $(E, d)$ est un espace métrique et $U \subset E$ est un sous-ensemble de $E$, la restriction de $d$ à $U \times U$ fait de $(U, d)$ un espace métrique.

\section{Boules dans les espaces métriques}

Soit $(E, d)$ un espace métrique, $x_0 \in E$ et $r \geq 0$.

\begin{definition}
\begin{enumerate}
    \item La \textbf{boule ouverte} de centre $x_0$ et de rayon $r$ est l'ensemble
    \[
    B(x_0, r) = \{x \in E : d(x_0, x) < r\}
    \]
    \item La \textbf{boule fermée} de centre $x_0$ et de rayon $r$ est l'ensemble
    \[
    B_f(x_0, r) = \{x \in E : d(x_0, x) \leq r\}
    \]
    \item La \textbf{sphère} de centre $x_0$ et de rayon $r$ est l'ensemble
    \[
    S(x_0, r) = \{x \in E : d(x_0, x) = r\}
    \]
\end{enumerate}
\end{definition}

\subsection{Propriétés des boules}

\begin{lemma}
\begin{enumerate}
    \item $B(x_0, 0) = \emptyset$ et $B_f(x_0, 0) = \{x_0\}$
    \item Pour $0 \leq r_1 < r_2$, on a $B(x_0, r_1) \subset B_f(x_0, r_1) \subset B(x_0, r_2)$
    \item Si $d(x_0, x_1) + r_1 \leq r$, alors $B(x_1, r_1) \subset B(x_0, r)$
\end{enumerate}
\end{lemma}

\begin{proof}
\begin{enumerate}
    \item Les propriétés (1) et (2) sont évidentes d'après la définition des boules.

    \item Pour montrer (3), supposons $x \in B(x_1, r_1)$. Alors $d(x_1, x) < r_1$.
    Par l'inégalité triangulaire, on a
    \[
    d(x_0, x) \leq d(x_0, x_1) + d(x_1, x) < d(x_0, x_1) + r_1 \leq r
    \]
    Donc $d(x_0, x) < r$, ce qui signifie que $x \in B(x_0, r)$. D'où $B(x_1, r_1) \subset B(x_0, r)$.
\end{enumerate}
\end{proof}\chapter{CM2}
\sloppy

\section{Espaces métriques}

\begin{definition}
Soit $E$ un ensemble. Une application $d : E \times E \to \mathbb{R}^+$ est appelée distance sur $E$ si :
\begin{enumerate}
    \item $d(x, y) \geq 0$ (positivité)
    \item $d(x, y) = d(y, x)$ (symétrie)
    \item $d(x, y) \leq d(x, z) + d(z, y)$ (inégalité triangulaire)
    \item $d(x, y) = 0 \Leftrightarrow x = y$ (axiome de séparation)
\end{enumerate}
$(E, d)$ est appelé espace métrique.
\end{definition}

\begin{proposition}
[Inégalité triangulaire]
Dans un espace métrique $(E, d)$, on a aussi l'inégalité suivante:
\[
|d(x, y) - d(x, z)| \leq d(y, z)
\]
\end{proposition}

\subsection{Exemples}

\begin{example}
\begin{enumerate}
    \item $E = \mathbb{R}$. On définit $d(x, y) = |x - y|$. \\
    Boule $B(x_0, r) = \{x \in \mathbb{R} : d(x, x_0) < r \} = ]x_0 - r, x_0 + r[$.

    \item $E = \mathbb{R}^d$, $d = 2, 3, \dots$. On a différentes normes :
    \begin{itemize}
        \item Norme euclidienne: $||x||_2 = (\sum_{i=1}^d x_i^2)^{1/2}$
        \item Norme $1$: $||x||_1 = \sum_{i=1}^d |x_i|$
        \item Norme $\infty$: $||x||_\infty = \max_{1 \leq i \leq d} |x_i|$
    \end{itemize}
    Pour $E = \mathbb{R}^d$, on définit la distance $d_2(x, y) = ||y - x||_2 = ||\overrightarrow{xy}||_2$. De même, on peut définir $d_1(x, y) = ||y - x||_1$ et $d_\infty(x, y) = ||y - x||_\infty$.

    Boule $B_2(0, r)$ pour $d_2$ dans $\mathbb{R}^2$:

    \begin{verbatim}
    ```python
#save_to: boule_d2.png
import matplotlib.pyplot as plt
import matplotlib.patches as patches
import numpy as np

r=1
fig, ax = plt.subplots()
circle = patches.Circle((0, 0), r, edgecolor='black', facecolor='lightgray', fill=True, linewidth=1)
ax.add_patch(circle)
ax.axhline(0, color='black',linewidth=0.5)
ax.axvline(0, color='black',linewidth=0.5)
ax.set_aspect('equal', adjustable='box')
ax.set_xlim([-1.5*r, 1.5*r])
ax.set_ylim([-1.5*r, 1.5*r])
ax.set_xticks([-r, 0, r])
ax.set_yticks([-r, 0, r])
ax.set_xlabel('$x$')
ax.set_ylabel('$y$')
ax.set_title('$B_2(0, r)$')
plt.savefig('boule_d2.png')
    ```
    \end{verbatim}

    \begin{figure}[H]
        \centering
        \includegraphics[width=0.5\textwidth]{boule_d2.png}
        \caption{Boule $B_2(0, r)$ dans $\mathbb{R}^2$}
        \label{fig:boule_d2}
    \end{figure}

    Boule $B_\infty(0, r)$ pour $d_\infty$ dans $\mathbb{R}^2$:

    \begin{verbatim}
    ```python
#save_to: boule_d_infini.png
import matplotlib.pyplot as plt
import matplotlib.patches as patches
import numpy as np

r=1
fig, ax = plt.subplots()
square = patches.Rectangle((-r, -r), 2*r, 2*r, edgecolor='black', facecolor='lightgray', fill=True, linewidth=1)
ax.add_patch(square)
ax.axhline(0, color='black',linewidth=0.5)
ax.axvline(0, color='black',linewidth=0.5)
ax.set_aspect('equal', adjustable='box')
ax.set_xlim([-1.5*r, 1.5*r])
ax.set_ylim([-1.5*r, 1.5*r])
ax.set_xticks([-r, 0, r])
ax.set_yticks([-r, 0, r])
ax.set_xlabel('$x$')
ax.set_ylabel('$y$')
ax.set_title('$B_\infty(0, r)$')
plt.savefig('boule_d_infini.png')
    ```
    \end{verbatim}

    \begin{figure}[H]
        \centering
        \includegraphics[width=0.5\textwidth]{boule_d_infini.png}
        \caption{Boule $B_\infty(0, r)$ dans $\mathbb{R}^2$}
        \label{fig:boule_d_infini}
    \end{figure}

        Boule $B_1(0, r)$ pour $d_1$ dans $\mathbb{R}^2$:

    \begin{verbatim}
    ```python
#save_to: boule_d1.png
import matplotlib.pyplot as plt
import matplotlib.patches as patches
import numpy as np

r=1
fig, ax = plt.subplots()
diamond = patches.Polygon([[r, 0], [0, r], [-r, 0], [0, -r]], closed=True, edgecolor='black', facecolor='lightgray', fill=True, linewidth=1)
ax.add_patch(diamond)
ax.axhline(0, color='black',linewidth=0.5)
ax.axvline(0, color='black',linewidth=0.5)
ax.set_aspect('equal', adjustable='box')
ax.set_xlim([-1.5*r, 1.5*r])
ax.set_ylim([-1.5*r, 1.5*r])
ax.set_xticks([-r, 0, r])
ax.set_yticks([-r, 0, r])
ax.set_xlabel('$x$')
ax.set_ylabel('$y$')
ax.set_title('$B_1(0, r)$')
plt.savefig('boule_d1.png')
    ```
    \end{verbatim}

    \begin{figure}[H]
        \centering
        \includegraphics[width=0.5\textwidth]{boule_d1.png}
        \caption{Boule $B_1(0, r)$ dans $\mathbb{R}^2$}
        \label{fig:boule_d1}
    \end{figure}

    \begin{remark}
    \textbf{Important:} notion de proximité, pas la forme.
    \end{remark}

    Dans $\mathbb{R}^n$, on a les relations entre les distances:
    \begin{align*}
        d_\infty(x, y) &\leq d_1(x, y) \leq n d_\infty(x, y) \\
        d_\infty(x, y) &\leq d_2(x, y) \leq \sqrt{n} d_\infty(x, y)
    \end{align*}
\end{enumerate}
\end{example}

\section{Parties bornées}

\begin{definition}
Soit $(E, d)$ un espace métrique et $A \subset E$. $A$ est dite \textbf{bornée} si
\[
\exists R > 0 \text{ et } \exists x_0 \in E \text{ tel que } A \subset B(x_0, R).
\]
\end{definition}

\begin{lemma}
Les propriétés suivantes sont équivalentes:
\begin{enumerate}
    \item $A$ est bornée.
    \item $\forall x_0 \in E, \exists r > 0$ tel que $A \subset B(x_0, r)$.
    \item $\exists r > 0$ tel que $\forall x, y \in A$, on a $d(x, y) < r$.
\end{enumerate}
\end{lemma}

\begin{solution}
[Démonstration:]
\textbf{1) $\Rightarrow$ 2) $\Rightarrow$ 3) $\Rightarrow$ 1)}

\textbf{Preuve que 1) $\Rightarrow$ 2)}

Hyp: $\exists x_1 \in E, \exists r_1 > 0$ tel que $A \subset B(x_1, r_1)$.

Soit $x_0 \in E$. But: trouver $r$ tel que $A \subset B(x_0, r)$.

Si $x \in A$, alors $x \in B(x_1, r_1)$, on a $d(x, x_1) < r_1$.

On veut $d(x_0, x) < r$.

\begin{align*}
    d(x_0, x) &\leq d(x_0, x_1) + d(x_1, x) \\
              &\leq d(x_0, x_1) + r_1 \\
              &< r \text{ si } r > d(x_0, x_1) + r_1
\end{align*}
Il suffit de prendre $r = d(x_0, x_1) + r_1$.

\textbf{2) $\Rightarrow$ 3)}

On fixe $x_0 \in E$. D'après 2), $\exists r_0 > 0$ tel que $A \subset B(x_0, r_0)$.
Alors $\forall x, y \in A$,
\begin{align*}
    d(x, y) &\leq d(x, x_0) + d(x_0, y) \\
              &< r_0 + r_0 = 2r_0
\end{align*}
On prend $r = 2r_0$.

\textbf{3) $\Rightarrow$ 1)}

On fixe $x_0 \in E$ (n'importe lequel). D'après 3), $\exists r > 0$ tel que $\forall x, y \in A$, $d(x, y) < r$.
Alors $\forall x \in A$, $d(x_0, x) \leq d(x_0, y) + d(y, x) < d(x_0, y) + r$.
On fixe $y \in A$. Alors $d(x_0, y) < \infty$ est fixe. On prend $R = d(x_0, y) + r$.
Alors $\forall x \in A$, $d(x_0, x) < R$, donc $A \subset B(x_0, R)$.
\end{solution}

\begin{proposition}
[Propriétés élémentaires]
\begin{enumerate}
    \item Toute partie finie est bornée.
    \item Si $A$ bornée et $B \subset A$ alors $B$ bornée.
    \item L'union d'un nb fini de bornées est bornée.
\end{enumerate}
\end{proposition}

\begin{solution}
[Démonstration:]
\textbf{1) Tout partie finie est bornée}

Soit $A = \{a_1, \dots, a_n\}$ une partie finie de $E$. On fixe $x_0 \in E$.
Pour chaque $a_i$, $d(x_0, a_i) < \infty$. Soit $r_i = d(x_0, a_i) + 1$. Alors $a_i \in B(x_0, r_i)$.
On prend $R = \max_{1 \leq i \leq n} r_i$. Alors $a_i \in B(x_0, R)$ pour tout $i$.
Donc $A \subset B(x_0, R)$.

\textbf{2) Si $A$ bornée et $B \subset A$ alors $B$ bornée}

Si $A$ est bornée, $\exists x_0, R$ tq $A \subset B(x_0, R)$. Comme $B \subset A$, on a $B \subset B(x_0, R)$. Donc $B$ est bornée.

\textbf{3) L'union d'un nb fini de bornées est bornée (Partie)}

Soient $A_1, \dots, A_n$ sont bornées. Je fixe $x_0 \in E$.
$A_i$ bornée ($1 \leq i \leq n$) donc $\exists r_i > 0$ tel que $A_i \subset B(x_0, r_i)$.
Soit $r = \max_{1 \leq i \leq n} r_i$.
Si $x \in \bigcup_{i=1}^n A_i$, alors $x \in A_i$ pour un $i$. Donc $x \in B(x_0, r_i) \subset B(x_0, r)$.
Donc $\bigcup_{i=1}^n A_i \subset B(x_0, r)$.
\end{solution}

\section{Fonctions bornées}

\begin{definition}
Soit $B$ un ensemble. Une fonction $F: B \to E$ est \textbf{bornée} si
\[
F(B) = \{F(b) : b \in B\} \subset E
\]
est bornée.
\end{definition}

\section{Distances entre ensembles}

\begin{definition}
Soit $A, B$ deux parties de $E$. On pose
\[
d(A, B) = \inf_{\substack{x \in A \\ y \in B}} d(x, y).
\]
\end{definition}

\begin{remark}
$\forall x \in A, y \in B$, $d(A, B) \leq d(x, y)$.
$\forall \epsilon > 0$, $\exists x \in A, y \in B$ tq $d(x, y) \leq d(A, B) + \epsilon$.
\end{remark}

\begin{proposition}
[Notation Proposition]
\[
d(x, A) = \inf_{y \in A} d(x, y) = d(\{x\}, A).
\]
\end{proposition}

\section{Topologie des espaces métriques}

\textbf{Concepts importants:} distance $\to$ boules $B(x_0, r)$ $\to$ ensembles ouverts.

\begin{definition}
Soit $(E, d)$ espace métrique.
\begin{enumerate}
    \item $U \subset E$ est \textbf{ouvert} si
    \[
    \forall x_0 \in U, \exists r > 0 \text{ tel que } B(x_0, r) \subset U.
    \]
    \item $F \subset E$ est \textbf{fermé} si $E \setminus F$ est ouvert.
\end{enumerate}
\end{definition}

\begin{remark}
Dans $\mathbb{R}$ les intervalles ouverts sont des ouverts.
\end{remark}

\textbf{Comment montrer que l'ensemble est ouvert ou fermé.} Dans le poly.

\begin{itemize}
    \item $\emptyset$ est ouvert (par définition).
    \item $E$ est ouvert.
    \item $E$ est fermé, $\emptyset$ est fermé (comme complémentaires d'ouverts).
\end{itemize}

\section{Lemmes et théorèmes}

\begin{lemma}
$1) B(x_0, r)$ est ouvert.

$2) B_f(x_0, r)$ est fermé.
\end{lemma}

\begin{solution}
[Démo dans le poly]
\end{solution}

\begin{example}
$E = \mathbb{R}$, $d(x, y) = |y - x|$.
$A = ]0, 1[$ ouvert dans $\mathbb{R}$.
$A = ]0, 1[$ pas fermé dans $A$.
$R \setminus A = ]-\infty, 0] \cup [1, \infty[$ fermé dans $\mathbb{R}$.

Je regarde $A$ comme partie de $(A, d)$.
$A$ est fermé dans $A$.
\end{example}

\begin{theorem}
\begin{enumerate}
    \item Soit $U_i, i \in I$ une collection d'ouverts. Alors $\bigcup_{i \in I} U_i$ est ouvert (l'union quelconque d'ouverts est un ouvert).
    \item Si $U_1, \dots, U_n$ sont ouverts, alors $\bigcap_{i=1}^n U_i$ est ouvert (l'intersection d'une famille finie d'ouverts est ouvert).
    \item Si $F_i, i \in I$ sont fermés, alors $\bigcap_{i \in I} F_i$ est fermé (l'intersection quelconque de fermés est fermé).
    \item $F_1, \dots, F_n$ fermés, alors $\bigcup_{i=1}^n F_i$ est fermé.
\end{enumerate}
\end{theorem}

\begin{example}
[Examples et remarques]
$U_i = ]-\frac{1}{i}, \frac{1}{i}[$, $i \geq 0$.
$\bigcap_{i \in \mathbb{N}} U_i = \{0\}$ pas ouvert dans $\mathbb{R}$.

$F_i = [0, 1 - \frac{1}{i}]$ fermé dans $\mathbb{R}$.
$\bigcup_{i \in \mathbb{N}} F_i = [0, 1[$ pas fermé dans $\mathbb{R}$.
\end{example}

\begin{solution}
[Dem:]
\textbf{1) Soit $x \in \bigcup_{i \in I} U_i = U$}. Il existe un $i$ noté $i_0$ tel que $x \in U_{i_0}$.
$U_{i_0}$ est ouvert donc $\exists r > 0$ tel que $B(x, r) \subset U_{i_0} \subset \bigcup_{i \in I} U_i = U$. Donc $U$ est ouvert.

\textbf{2) Soit $x \in \bigcap_{i=1}^n U_i = U$}. $x \in U_i$ pour $1 \leq i \leq n$. $U_i$ ouvert donc $\exists r_i > 0$ tel que $B(x, r_i) \subset U_i$.
Soit $r = \min_{1 \leq i \leq n} r_i > 0$.
$B(x, r) \subset B(x, r_i) \subset U_i$ $1 \leq i \leq n$.
Donc $B(x, r) \subset \bigcap_{i=1}^n U_i = U$.
Donc $B(x, r) \subset U$.
\end{solution}\chapter{CM3}
\sloppy

\section{Intérieur, Adhérence, Frontière}

\subsection{Intérieur}

\begin{definition}
Soit $A \subseteq E$.
\begin{enumerate}
    \item Un point $x_0 \in E$ est \textbf{intérieur} à $A$ s'il existe $\delta > 0$ tel que $B(x_0, \delta) \subseteq A$.
    \item \textbf{Int$(A)$} (l'intérieur de $A$) : ensemble de tous les points intérieurs de $A$.
\end{enumerate}
Autre notation : $\overset{\circ}{A}$.
\end{definition}

\begin{proposition}
Int$(A)$ est le plus grand ouvert inclus dans $A$. De manière équivalente, Int$(A)$ est la réunion de tous les ouverts inclus dans $A$.
\end{proposition}

\begin{proof}
\begin{enumerate}
    \item \textbf{Int$(A) \subseteq A$ : évident}. Par définition, tous les points de Int$(A)$ sont dans $A$.
    \item \textbf{Int$(A)$ est ouvert}. Soit $x_0 \in \text{Int}(A)$. Il existe $\delta_0 > 0$ tel que $B(x_0, \delta_0) \subseteq A$. Pour montrer que Int$(A)$ est ouvert, il faut montrer que pour tout $x \in \text{Int}(A)$, il existe $\delta > 0$ tel que $B(x, \delta) \subseteq \text{Int}(A)$.

    Soit $x \in \text{Int}(A)$. Puisque $x \in \text{Int}(A)$, il existe $\delta_0 > 0$ tel que $B(x, \delta_0) \subseteq A$.
    Pour montrer que $x$ est un point intérieur de $\text{Int}(A)$, nous devons trouver un $\delta > 0$ tel que $B(x, \delta) \subseteq \text{Int}(A)$.
    Choisissons $\delta = \delta_0/2$. Considérons $y \in B(x, \delta)$. Alors $d(y, x) < \delta_0/2$.
    Pour montrer que $y \in \text{Int}(A)$, nous devons trouver $\delta' > 0$ tel que $B(y, \delta') \subseteq A$.
    Prenons $\delta' = \delta_0/2$. Si $z \in B(y, \delta')$, alors $d(z, y) < \delta_0/2$.
    Par l'inégalité triangulaire, $d(z, x) \leq d(z, y) + d(y, x) < \delta_0/2 + \delta_0/2 = \delta_0$.
    Donc $z \in B(x, \delta_0) \subseteq A$. Ainsi, $B(y, \delta') \subseteq A$, ce qui signifie que $y \in \text{Int}(A)$.
    Par conséquent, $B(x, \delta) \subseteq \text{Int}(A)$. Donc Int$(A)$ est ouvert.

    \item Si $U$ est ouvert et $U \subseteq A$ alors $U \subseteq \text{Int}(A)$ ?

    Soit $U$ un ouvert tel que $U \subseteq A$. Pour tout $x_0 \in U$, puisque $U$ est ouvert, il existe $\delta > 0$ tel que $B(x_0, \delta) \subseteq U$. Comme $U \subseteq A$, on a $B(x_0, \delta) \subseteq A$. Par définition, cela signifie que $x_0$ est un point intérieur de $A$, donc $x_0 \in \text{Int}(A)$. Par conséquent, $U \subseteq \text{Int}(A)$.


\end{enumerate}
\end{proof}


\subsection{Adhérence}

\begin{definition}
Soit $A \subseteq E$, $x_0 \in E$. $x_0$ est \textbf{adhérent} à $A$ si $\forall \delta > 0$, $B(x_0, \delta) \cap A \neq \emptyset$. (Équivalent à $d(x_0, A) = 0$).

\textbf{Adh$(A)$} (adhérence ou fermeture de $A$) = ensemble des points adhérents à $A$.

Notée aussi $\overline{A}$.
\end{definition}

$d(x_0, A) = \inf_{x \in A} d(x_0, x)$.

$d(x_0, A) = 0 \iff x_0 \in \text{Adh}(A)$.

$\forall \delta > 0$, $\exists x \in A$ t.q. $d(x_0, x) < \delta$.
$\forall \delta > 0$, $\exists x \in A$ t.q. $d(x_0, x) \leq \delta$.
$\forall \delta > 0$, donc $d(x_0, A) = 0$.

\begin{proposition}
Adh$(A)$ est le plus petit fermé qui contient $A$ (l'intersection de tous les fermés qui contiennent $A$).
\end{proposition}

\begin{proof}
\begin{enumerate}
    \item $A \subseteq \text{Adh}(A)$ : clair. Si $x \in A$, alors pour tout $\delta > 0$, $B(x, \delta) \cap A \neq \emptyset$ car $x \in B(x, \delta) \cap A$. Donc $x \in \text{Adh}(A)$.
    \item Adh$(A)$ est fermé. Il faut montrer que $E \setminus \text{Adh}(A)$ est ouvert.

    $x_0 \in \text{Adh}(A) \iff \forall \delta > 0$, $B(x_0, \delta) \cap A \neq \emptyset$.
    $x_0 \notin \text{Adh}(A) \iff \exists \delta_0 > 0$ t.q. $B(x_0, \delta_0) \cap A = \emptyset$.
    $\iff \exists \delta_0 > 0$ t.q. $B(x_0, \delta_0) \subseteq E \setminus A$.
    $\implies x_0 \in \text{Int}(E \setminus A)$.

    Donc $E \setminus \text{Adh}(A) \subseteq \text{Int}(E \setminus A)$.

    Réciproquement, si $x_0 \in \text{Int}(E \setminus A)$, alors il existe $\delta_0 > 0$ tel que $B(x_0, \delta_0) \subseteq E \setminus A$. Donc $B(x_0, \delta_0) \cap A = \emptyset$. Ainsi $x_0 \notin \text{Adh}(A)$, et donc $x_0 \in E \setminus \text{Adh}(A)$.

    $E \setminus \text{Adh}(A) = \text{Int}(E \setminus A)$.

    Comme $\text{Int}(E \setminus A)$ est ouvert, son complémentaire $E \setminus \text{Int}(E \setminus A) = \text{Adh}(A)$ est fermé.

    Adh$(A) = E \setminus \text{Int}(E \setminus A)$.
\end{enumerate}
\end{proof}


\subsection{Frontière}

\begin{definition}
Soit $A \subseteq E$, la \textbf{frontière} de $A$ (ou bord de $A$) notée Fr$(A)$ ou $\partial A$, c'est $\text{Adh}(A) \cap \text{Adh}(E \setminus A)$.

$x_0 \in \text{Fr}(A) \iff d(x_0, A) = 0$ et $d(x_0, E \setminus A) = 0$.

$\forall \delta > 0$, $B(x_0, \delta)$ intersecte $A$ et aussi $E \setminus A$.
\end{definition}

\begin{example}
\textbf{Exemples dans $\mathbb{R}$}
Int$(\mathbb{Q}) = \emptyset$.
Int$(\mathbb{R} \setminus \mathbb{Q}) = \emptyset$.

Adh$(\mathbb{Q}) = \mathbb{R}$.
Adh$(\mathbb{R} \setminus \mathbb{Q}) = \mathbb{R}$.

Fr$(\mathbb{Q}) = \mathbb{R}$.
Fr$(\mathbb{R} \setminus \mathbb{Q}) = \mathbb{R}$.
\end{example}

Parfois $B_f(x_0, r)$ notée $\overline{B}(x_0, r)$.

\begin{example}
$E = \{a, b, c\}$.
On pose $d(a, a) = d(b, b) = d(c, c) = 0$.
$d(a, b) = d(b, a) = d(b, c) = d(c, b) = 1$.
$d(a, c) = d(c, a) = 2$.

$B(a, 2) = \{a, b\} = \text{Adh}(B(a, 2))$. No it should be $B(a,2) = \{y \in E : d(a,y) < 2\} = \{a, b\}$.  Adh$(B(a, 2)) = \text{Adh}(\{a,b\})$. Points adherent to $\{a,b\}$ are points $x$ such that for any $\delta>0$, $B(x,\delta) \cap \{a,b\} \ne \emptyset$.
For $a$, $B(a, \delta) \cap \{a,b\} \ne \emptyset$ for any $\delta > 0$. Same for $b$. For $c$, $B(c, 1) = \{c\}$, $B(c, 1) \cap \{a,b\} = \emptyset$. So $\text{Adh}(\{a,b\}) = \{a,b\}$.
$B_f(a, 2) = \{y \in E : d(a,y) \leq 2\} = \{a, b, c\} = E$.
\end{example}


\begin{proposition}
\begin{enumerate}
    \item Int$(A) \subseteq A \subseteq \text{Adh}(A)$.
    \item $E = \text{Int}(A) \cup \text{Fr}(A) \cup \text{Int}(E \setminus A)$ (union disjointe).
    \item $E \setminus \text{Int}(A) = \text{Adh}(E \setminus A)$.
    \item $E \setminus \text{Adh}(A) = \text{Int}(E \setminus A)$.
    \item Fr$(A) = \text{Adh}(A) \setminus \text{Int}(A)$.
\end{enumerate}
\end{proposition}

\begin{proposition}
\begin{enumerate}
    \item $A$ ouvert $\iff A = \text{Int}(A)$.
    \item $A$ fermé $\iff A = \text{Adh}(A)$.
    \item $x \in \text{Adh}(A) \iff d(x, A) = 0$.
    \item $x \in \text{Int}(A) \iff d(x, E \setminus A) > 0$.
\end{enumerate}
\end{proposition}

\section{Ensembles Denses}

\begin{definition}
Soit $A \subseteq B \subseteq E$. On dit que $A$ est \textbf{dense} dans $B$ si $B \subseteq \text{Adh}(A)$.
\end{definition}

Soit $x_0 \in B$, $\forall \epsilon > 0$, $\exists x \in A$ t.q. $d(x_0, x) < \epsilon$.

\begin{example}
$\mathbb{Q}^2 = \{(x, y) : x, y \in \mathbb{Q}\}$ dense dans $\mathbb{R}^2$.
\end{example}


\section{Suites dans un Espace Métrique}

\begin{definition}
Soit $E$ un ensemble. Une \textbf{suite} dans $E$ (notée $(u_n)_{n \in \mathbb{N}}$) c'est une fonction $u : \mathbb{N} \to E$ où $n \mapsto u(n)$.
On note $u_n$ le $n$-ième terme de la suite $(u_n)_{n \in \mathbb{N}}$.

Si $E = \mathbb{R}^d$.
$X_n = (x_{1,n}, \dots, x_{d,n})$ où $(x_{i,n})_{n \in \mathbb{N}}$ suites dans $\mathbb{R}$.
\end{definition}

\begin{definition}
Soit $(X_n)_{n \in \mathbb{N}}$ une suite dans $E$ et $x \in E$. On dit que $\lim_{n \to \infty} X_n = x$ si :
$(\forall \epsilon > 0), (\exists N \in \mathbb{N})$ t.q. si $n \geq N \implies d(X_n, x) < \epsilon$.
\end{definition}

\textbf{Suite bornée} : $(X_n)_{n \in \mathbb{N}}$ est \textbf{bornée} si $\{X_n : n \in \mathbb{N}\} \subseteq E$ est un ensemble borné.

\begin{remark}
Dans $\mathbb{R}^d$ muni de $d_2$.
$X_n = (x_{1, n}, \dots, x_{d, n})$.
$X = (x_1, \dots, x_d)$.
$\lim_{n \to \infty} X_n = X \iff \lim_{n \to \infty} x_{i, n} = x_i$, $1 \leq i \leq d$.
\end{remark}

\begin{proposition}
La limite d'une suite convergente est unique.
\end{proposition}

\begin{proof}
Soit $X_n \xrightarrow[n \to \infty]{} x$ et $X_n \xrightarrow[n \to \infty]{} x'$.
$d(x, x') \leq d(x, X_n) + d(X_n, x')$.
$\xrightarrow[n \to \infty]{} 0$.

$\implies d(x, x') = 0 \implies x = x'$.
\end{proof}

\begin{proposition}[Lien avec l'adhérence]
\begin{enumerate}
    \item $x \in \text{Adh}(A)$ ssi il existe une suite $(X_n)$ d'éléments de $A$ t.q. $\lim_{n \to \infty} X_n = x$.
    \item $A$ est fermé ssi pour toute suite $(X_n)$ d'éléments de $A$ qui converge vers $x \in E$, on a $x \in A$.
\end{enumerate}
\end{proposition}

\begin{proof}
\begin{enumerate}
    \item "$\implies$" : Soit $x \in \text{Adh}(A)$.

    Avec $(X_n)$, $X_n \in A$ et $\lim_{n \to \infty} X_n = x$.

    J'ai $\forall \epsilon > 0$, $\exists x_\epsilon \in A$ t.q. $d(x, x_\epsilon) < \epsilon$.
    donc $\inf_{y \in A} d(x, y) = 0 = d(x, A)$.
    $d(x, A) = 0 \implies x \in \text{Adh}(A)$.

    "$\implies$" soit $x \in \text{Adh}(A)$.
    $\implies d(x, A) = 0$.
    $\implies \forall \epsilon > 0$, $\exists x_\epsilon \in A$ t.q. $d(x, x_\epsilon) < \epsilon$.
    Prendre $\epsilon = 1/n$. Je pose $u_n = x_{1/n}$.
    $u_n \in A$. $d(x, u_n) \leq 1/n$. Donc $\lim_{n \to \infty} u_n = x$.

    \item "$\implies$" soit $A$ fermé donc $A = \text{Adh}(A)$.

    Soit $(X_n)$ suite dans $A$ qui converge vers $x$.
    $x \in \text{Adh}(A) = A$.
    $x \in \text{Adh}(A) \implies x \in A$.

    "$\Longleftarrow$" Réciproquement.
    Si toute suite dans $A$ qui converge vers $x$, $x \in A$ (donc $A$ fermé).
    $A \subseteq \text{Adh}(A)$, j'ai $A = \text{Adh}(A)$ (donc $A$ fermé).
    Suites de Cauchy.
\end{enumerate}
\end{proof}


\subsection{Suites de Cauchy}
\begin{definition}
Une suite $(X_n)$ est de \textbf{Cauchy} si :
$\forall \epsilon > 0$, $\exists N \in \mathbb{N}$ tel que $d(X_p, X_n) < \epsilon$ pour tous $n, p \geq N$.
\end{definition}\chapter{CM4}
\sloppy

\section{Suites de Cauchy et Complétude}

\begin{definition}[Suite de Cauchy]
Une suite $(x_n)_{n \in \mathbb{N}}$ dans un espace métrique $E$ est dite suite de Cauchy si pour tout $\varepsilon > 0$, il existe $N \in \mathbb{N}$ tel que pour tous $n, p \geq N$, on a $d(x_p, x_n) \leq \varepsilon$.
\end{definition}

\begin{proposition}
Toute suite convergente est une suite de Cauchy.
\end{proposition}
\begin{proof}
Supposons que $\lim_{n \to \infty} x_n = x$. Pour $\varepsilon > 0$, on peut trouver $N \in \mathbb{N}$ tel que $d(x, x_n) \leq \varepsilon/2$ pour tout $n \geq N$. Alors pour tous $n, p \geq N$, on a
\begin{align*}
d(x_n, x_p) &\leq d(x_n, x) + d(x, x_p) \\
&\leq \varepsilon/2 + \varepsilon/2 = \varepsilon.
\end{align*}
Ainsi $(x_n)_{n \in \mathbb{N}}$ est une suite de Cauchy.
\end{proof}

\begin{proposition}
Toute suite de Cauchy est bornée.
\end{proposition}
\begin{proof}
Soit $(x_n)_{n \in \mathbb{N}}$ une suite de Cauchy. Par définition (en prenant $\varepsilon = 1$), il existe $N$ tel que $d(x_n, x_p) \leq 1$ pour $n, p \geq N$. En particulier $d(x_n, x_N) \leq 1$ pour $n \geq N$. On a donc pour tout $n \in \mathbb{N}$,
$$d(x_n, x_N) \leq \max( \{d(x_1, x_N), \dots, d(x_{N-1}, x_N)\} \cup \{1\}) =: r_0.$$
Ainsi $x_n \in B(x_N, r_0)$ pour tout $n \in \mathbb{N}$.
\end{proof}

\begin{definition}[Espace complet]
Un espace métrique $(E, d)$ est dit complet si toute suite de Cauchy dans $E$ est convergente.
\end{definition}

\begin{theorem}
$\mathbb{R}^d$ muni de la distance canonique est complet.
\end{theorem}

\section{Intérieur et Adhérence}

\begin{definition}[Intérieur]
Soit $A \subset E$. Un point $x \in E$ est intérieur à $A$ s'il existe $\delta > 0$ tel que $B(x, \delta) \subset A$. L'ensemble des points intérieurs à $A$ se note $\mathrm{Int}(A)$ et s'appelle l'intérieur de $A$.
\end{definition}

\begin{proposition}
$\mathrm{Int}(A)$ est le plus grand ouvert inclus dans $A$, ou de manière équivalente la réunion de tous les ouverts inclus dans $A$.
\end{proposition}

\begin{definition}[Adhérence]
Soit $A \subset E$. Un point $x \in E$ est adhérent à $A$ si $B(x, r) \cap A \neq \emptyset$ pour tout $r > 0$. L'ensemble des points adhérents à $A$ se note $\mathrm{Adh}(A)$ et s'appelle l'adhérence ou la fermeture de $A$.
\end{definition}

\begin{proposition}
$\mathrm{Adh}(A)$ est le plus petit fermé contenant $A$, ou de manière équivalente l'intersection de tous les fermés contenant $A$.
\end{proposition}

\begin{proposition}
$x \in \mathrm{Adh}(A)$ si et seulement s'il existe une suite $(x_n)_{n \in \mathbb{N}}$ d'éléments de $A$ telle que $x = \lim_{n \to \infty} x_n$.
\end{proposition}

\begin{example}
Soit $A = \{(x,y) \in \mathbb{R}^2 : 2x+3y < 4\}$. Déterminer $\mathrm{Int}(A)$ et $\mathrm{Adh}(A)$.
\begin{itemize}
    \item $\mathrm{Int}(A) = A = \{(x,y) \in \mathbb{R}^2 : 2x+3y < 4\}$. $A$ est ouvert.
    \item $\mathrm{Adh}(A) = C = \{(x,y) \in \mathbb{R}^2 : 2x+3y \leq 4\}$. $C$ est fermé et contient $A$.
\end{itemize}
\end{example}

\begin{example}
Soit $A = \{(x,y) \in \mathbb{R}^2 : x > 0, y = \sin(1/x)\}$. Déterminer $\mathrm{Adh}(A)$ et $\mathrm{Int}(A)$.
\begin{itemize}
    \item $\mathrm{Int}(A) = \emptyset$. Car $\mathrm{Int}(A)$ est un ouvert inclus dans $A$. Or $A$ ne contient aucune boule ouverte.
    \item $\mathrm{Adh}(A) = A \cup \{(0,y) : y \in [-1, 1]\}$.
\end{itemize}
\end{example}

\section{Exercices Résolus}

\begin{example}
Soit $A = \{(x,y) \in \mathbb{R}^2 : |x| < 1, |y| < 1\}$. Déterminer $\mathrm{Int}(A)$ et $\mathrm{Adh}(A)$.
\end{example}

\begin{solution}
\begin{enumerate}
    \item On dessine $A$, c'est un carré ouvert.
    \item On pense que $\mathrm{Int}(A) = B = A = \{(x,y) \in \mathbb{R}^2 : |x| < 1, |y| < 1\}$ et $\mathrm{Adh}(A) = C = \{(x,y) \in \mathbb{R}^2 : |x| \leq 1, |y| \leq 1\}$.
    \item Montrons que $B = \mathrm{Int}(A)$.
    \begin{itemize}
        \item $B$ est ouvert et $B \subset A$. Vrai par définition de $B$.
        \item Soit $X \in A \setminus B = \emptyset$. Donc il n'y a pas de points de $A$ qui ne sont pas dans $B$. Ainsi $B = \mathrm{Int}(A)$.
    \end{itemize}
    \item Montrons que $C = \mathrm{Adh}(A)$.
    \begin{itemize}
        \item $C$ est fermé et $A \subset C$. Vrai par définition de $C$.
        \item Montrons que $C \subset \mathrm{Adh}(A)$. Pour chaque $X \in C$, on cherche une suite $(X_n)$ avec $X_n \in A$ et $\lim X_n = X$. Soit $X = (x, y) \in C$, i.e., $|x| \leq 1, |y| \leq 1$. On prend $X_n = (x - 1/n, y - 1/n)$ (si $x = 1$, on prend $x - 1/n$, similarly for $y$). Plus précisément, soit $X_n = (x_n, y_n)$ avec $x_n = x - \frac{1}{n} \text{sign}(x)$ si $x \neq 0$ et $x_n = -1/n$ si $x = 0$, et $y_n = y - \frac{1}{n} \text{sign}(y)$ si $y \neq 0$ et $y_n = -1/n$ si $y = 0$. Alors $X_n \in A$ et $\lim X_n = X$.
    \end{itemize}
\end{enumerate}
\end{solution}

\begin{example}
Soit $A = \{(x,y) \in \mathbb{R}^2 : x > 0, y = \sin(1/x)\}$. Déterminer $\mathrm{Adh}(A)$ et $\mathrm{Int}(A)$.
\end{example}

\begin{solution}
\begin{enumerate}
    \item On dessine $A$. C'est le graphe de $\sin(1/x)$ pour $x > 0$.
    \item On pense que $\mathrm{Int}(A) = \emptyset$ et $\mathrm{Adh}(A) = A \cup \{(0,y) : y \in [-1, 1]\}$. Soit $C = A \cup \{(0,y) : y \in [-1, 1]\}$.
    \item Montrons que $\mathrm{Int}(A) = \emptyset$. Si $\mathrm{Int}(A) \neq \emptyset$, alors $\mathrm{Int}(A)$ est un ouvert non vide inclus dans $A$. Donc $\mathrm{Int}(A)$ contient une boule $B(X_0, r) \subset A$. Mais $A$ est le graphe d'une fonction, il n'y a pas de boule dans $A$. Donc $\mathrm{Int}(A) = \emptyset$.
    \item Montrons que $\mathrm{Adh}(A) = C = A \cup \{(0,y) : y \in [-1, 1]\}$.
    \begin{itemize}
        \item $C$ est fermé et $A \subset C$. $A$ n'est pas fermé. $C$ est fermé, car si $(X_n) \in C$ et $X_n \to X$, alors $X \in C$. Si $X_n = (x_n, y_n) \in A$, alors $x_n > 0, y_n = \sin(1/x_n)$. Si $X_n \to X = (x, y)$, alors $x_n \to x, y_n \to y$. Si $x > 0$, alors $X \in A \subset C$. Si $x = 0$, on ne peut pas dire que $y = \sin(1/x)$. Mais on sait que $-1 \leq \sin(1/x_n) \leq 1$, donc $-1 \leq y_n \leq 1$, donc $-1 \leq y \leq 1$. Donc si $x = 0$, $X = (0, y)$ avec $y \in [-1, 1]$, donc $X \in C$.
        \item Montrons que $C \subset \mathrm{Adh}(A)$. Pour $X \in C$, si $X \in A$, alors $X \in \mathrm{Adh}(A)$. Si $X \in C \setminus A = \{(0,y) : y \in [-1, 1]\}$, i.e., $X = (0, y)$ avec $y \in [-1, 1]$. On doit montrer que $X \in \mathrm{Adh}(A)$. On cherche une suite $X_n \in A$ avec $X_n \to X$. On prend $X_n = (\frac{1}{n \pi + \arcsin(y)}, \sin(n \pi + \arcsin(y))) = (\frac{1}{n \pi + \arcsin(y)}, y)$. Alors $X_n \in A$ et $X_n \to (0, y) = X$. Donc $X \in \mathrm{Adh}(A)$.
    \end{itemize}
\end{enumerate}
\end{solution}\chapter{CM5}
\sloppy

\section{Compacité}

\subsection{Définitions clés}

\begin{definition}[Recouvrement ouvert]
Soit $F \subset E$. Un recouvrement ouvert de $F$ est une collection $(U_i)_{i \in I}$ où $U_i$ sont des ouverts de $E$ et $F \subset \bigcup_{i \in I} U_i$.
\end{definition}

\begin{definition}[Ensemble compact]
$K \subset E$ est compact si de tout recouvrement ouvert $(U_i)_{i \in I}$ de $K$, on peut extraire un sous-recouvrement fini, c'est-à-dire qu'il existe un sous-ensemble fini $J \subset I$ tel que $K \subset \bigcup_{i \in J} U_i$.
\end{definition}

\begin{theorem}[Caractérisation séquentielle de la compacité]
$K \subset E$ est compact si et seulement si toute suite d'éléments de $K$ admet une sous-suite qui converge vers un élément de $K$.
\end{theorem}

\subsection{Exemples et contre-exemples}

\begin{example}
$F = \mathbb{R}^2$ n'est pas compact. Considérons le recouvrement ouvert $U_x = B(x, 1/2)$ pour $x \in \mathbb{R}^2$. Alors $\mathbb{R}^2 = \bigcup_{x \in \mathbb{R}^2} U_x$. Cependant, on ne peut pas extraire un sous-recouvrement fini.
\end{example}

\begin{example}
Soit $F = \{(x, y) \in \mathbb{R}^2 \mid x > 0, 0 \leq -\frac{1}{x} \leq y \leq \frac{1}{x} \}$. $F$ n'est pas compact. Considérons la suite $u_n = (n, 0) \in F$. Toute sous-suite de $(u_n)$ est non bornée, donc sans sous-suite convergente dans $F$.
\end{example}

\section{Propriétés des ensembles compacts}

\begin{proposition}
Tout compact $K \subset E$ est borné et fermé.
\end{proposition}

\begin{proposition}
Si $K$ est compact et $F$ est fermé, alors $K \cap F$ est compact.
\end{proposition}

\begin{proposition}
Si $K$ est compact, toute suite de Cauchy dans $K$ converge dans $K$.
\end{proposition}

\subsection{Preuves des propriétés}

\begin{proof}[Preuve qu'un compact est borné]
Soit $K$ compact. Pour $x \in K$, considérons $U_x = B(x, 1)$. Alors $(U_x)_{x \in K}$ est un recouvrement ouvert de $K$. Puisque $K$ est compact, il existe un sous-recouvrement fini $U_{x_1}, \dots, U_{x_n}$ tel que $K \subset \bigcup_{i=1}^n U_{x_i}$. Soit $R = \max_{1 \leq i \leq n} \|x_i\| + 1$. Alors pour tout $x \in K$, il existe $i$ tel que $x \in U_{x_i} = B(x_i, 1)$, donc $d(x, x_i) < 1$. Par l'inégalité triangulaire, $\|x\| \leq \|x - x_i\| + \|x_i\| < 1 + \|x_i\| \leq R$. Ainsi, $K \subset B(0, R)$, et $K$ est borné.
\end{proof}

\begin{proof}[Preuve qu'un compact est fermé]
Soit $K$ compact et montrons que $K$ est fermé. Montrons que $E \setminus K$ est ouvert. Soit $x \notin K$. Pour tout $y \in K$, il existe $r_y > 0$ tel que $B(x, r_y) \cap B(y, r_y) = \emptyset$. Considérons le recouvrement ouvert de $K$ donné par $(B(y, r_y))_{y \in K}$. Il existe un sous-recouvrement fini $B(y_1, r_{y_1}), \dots, B(y_n, r_{y_n})$ tel que $K \subset \bigcup_{i=1}^n B(y_i, r_{y_i})$. Soit $r = \min_{1 \leq i \leq n} r_{y_i} > 0$. Considérons $B(x, r)$. Pour tout $z \in B(x, r)$, et pour tout $i$, $B(z, r) \cap B(y_i, r_{y_i}) = \emptyset$. Donc $B(x, r) \cap K = \emptyset$, et $B(x, r) \subset E \setminus K$. Ainsi $E \setminus K$ est ouvert, et $K$ est fermé.
\end{proof}

\begin{proof}[Preuve que K compact et F fermé $\implies$ K $\cap$ F compact]
Soit $K$ compact et $F$ fermé. Considérons un recouvrement ouvert $(U_i)_{i \in I}$ de $K \cap F$. Alors $(U_i)_{i \in I} \cup (E \setminus F)$ est un recouvrement ouvert de $K$. Puisque $K$ est compact, il existe un sous-recouvrement fini $U_{i_1}, \dots, U_{i_n}, E \setminus F$ tel que $K \subset U_{i_1} \cup \dots \cup U_{i_n} \cup (E \setminus F)$. Alors $K \cap F \subset (U_{i_1} \cup \dots \cup U_{i_n} \cup (E \setminus F)) \cap F = (U_{i_1} \cap F) \cup \dots \cup (U_{i_n} \cap F) \cup ((E \setminus F) \cap F) = (U_{i_1} \cap F) \cup \dots \cup (U_{i_n} \cap F) \subset U_{i_1} \cup \dots \cup U_{i_n}$. Ainsi, $K \cap F$ est compact.
\end{proof}

\begin{proof}[Preuve que si K est compact, toute suite de Cauchy dans K converge dans K]
Soit $(u_n)$ une suite de Cauchy dans $K$ compact. Puisque $K$ est compact, il existe une sous-suite $(u_{\phi(n)})$ qui converge vers une limite $l \in K$. Puisque $(u_n)$ est de Cauchy et qu'une sous-suite converge vers $l$, la suite $(u_n)$ converge vers $l$. Donc toute suite de Cauchy dans $K$ converge dans $K$.
\end{proof}

\begin{proof}[Preuve par contradiction qu'un compact est borné]
Supposons que $K$ n'est pas borné. On fixe $a \in E$. Pour tout $n \in \mathbb{N}$, comme $K$ n'est pas borné, il existe $x_n \in K$ tel que $d(a, x_n) > n$. La suite $(x_n)$ n'est pas bornée (car $d(a, x_n) \to +\infty$), donc $(x_n)$ ne possède pas de sous-suite convergente. Ceci contredit le fait que $K$ est compact (par caractérisation séquentielle). Donc $K$ est borné.
\end{proof}

\section{Compacts de $\mathbb{R}^n$}

\subsection{Théorème de Borel-Lebesgue}

\begin{theorem}[Théorème de Borel-Lebesgue]
Dans $\mathbb{R}^n$ avec la distance usuelle, $K \subset \mathbb{R}^n$ est compact si et seulement si $K$ est fermé et borné.
\end{theorem}

\subsection{Compacité des boules fermées}

\begin{proposition}
Dans $\mathbb{R}^n$ avec la distance usuelle, les boules fermées $B_f(x_0, r)$ sont compactes.
\end{proposition}

\subsection{Preuve de la compacité des boules fermées}

\begin{proof}
Pour $n=1$, montrons que $[a, b]$ est compact. Soit $(U_i)_{i \in I}$ un recouvrement ouvert de $[a, b]$. Soit $\mathcal{U} = (U_i)_{i \in I}$.
Soit $E = \{x \in [a, b] \mid [a, x] \text{ est recouvert par un nombre fini de } U_i \}$.
$E$ est non vide car $a \in E$. Montrons que $E$ est borné.
Soit $c = \sup E$. Supposons que $c < b$. Puisque $c \in [a, b]$, il existe $U_{i_0} \in \mathcal{U}$ tel que $c \in U_{i_0}$. Comme $U_{i_0}$ est ouvert, il existe $\delta > 0$ tel que $]c - \delta, c + \delta[ \subset U_{i_0}$.
Puisque $c = \sup E$, il existe $x \in E$ tel que $c - \delta < x \leq c$. Par définition de $E$, $[a, x]$ est recouvert par un nombre fini de $U_i$.
Donc $[a, x] \cup [x, c + \delta/2] = [a, c + \delta/2]$ est recouvert par un nombre fini de $U_i$ (en ajoutant $U_{i_0}$). Donc $c + \delta/2 \in E$, ce qui contredit $c = \sup E$. Donc $c = b$.
Montrons que $b \in E$. On choisit $U_{i_1}$ tel que $b \in U_{i_1}$ et $\delta > 0$ tel que $]b - \delta, b + \delta[ \subset U_{i_1}$. On choisit $x \in ]b - \delta, b] \cap E$. Alors $[a, x]$ est recouvert par un nombre fini de $U_i$. Donc $[a, x] \cup [x, b] = [a, b]$ est recouvert par un nombre fini de $U_i$ (en ajoutant $U_{i_1}$). Donc $[a, b]$ est compact.
\end{proof}

\section{Limites et continuité}

\subsection{Définition des limites dans les espaces métriques}

\begin{definition}[Limite]
Soient $(E_1, d_1)$ et $(E_2, d_2)$ deux espaces métriques, $x_0 \in E_1$, $l \in E_2$ et $F : E_1 \to E_2$ une application. On dit que $\lim_{x \to x_0} F(x) = l$ si pour tout $\epsilon > 0$, il existe $\delta > 0$ tel que pour tout $x \in E_1$ tel que $d_1(x, x_0) < \delta$, on a $d_2(F(x), l) < \epsilon$.
\end{definition}

\subsection{Définition de la continuité}

\begin{definition}[Continuité en un point]
On dit que $F$ est continue en $x_0$ si $\lim_{x \to x_0} F(x) = F(x_0)$.
\end{definition}

\begin{definition}[Continuité sur un ensemble]
On dit que $F$ est continue (sur $E_1$) si $F$ est continue en tout point $x_0 \in E_1$.
\end{definition}

\begin{proposition}
$F$ est continue sur $E_1$ si et seulement si elle est continue en tout point de $E_1$.
\end{proposition}

\section{Propriétés équivalentes de la continuité}

\begin{proposition}[Propriétés équivalentes de la continuité]
Soient $(E_1, d_1)$ et $(E_2, d_2)$ deux espaces métriques et $F : E_1 \to E_2$ une application. Les propriétés suivantes sont équivalentes :
\begin{enumerate}
    \item $F$ est continue.
    \item Pour tout ouvert $U \subset E_2$, $F^{-1}(U)$ est ouvert dans $E_1$.
    \item Pour tout fermé $F \subset E_2$, $F^{-1}(F)$ est fermé dans $E_1$.
    \item Pour toute suite $(x_n)_{n \in \mathbb{N}}$ de $E_1$ avec $\lim_{n \to \infty} x_n = x \in E_1$, on a $\lim_{n \to \infty} F(x_n) = F(x)$.
\end{enumerate}
\end{proposition}

\subsection{Preuves des équivalences}

\begin{proof}
\begin{enumerate}
    \item $(1) \implies (2)$ : Soit $U \subset E_2$ ouvert et $x_0 \in F^{-1}(U)$. Alors $y_0 = F(x_0) \in U$. Comme $U$ est ouvert, il existe $\epsilon > 0$ tel que $B(y_0, \epsilon) \subset U$. Comme $F$ est continue en $x_0$, il existe $\delta > 0$ tel que si $d_1(x, x_0) < \delta$, alors $d_2(F(x), y_0) < \epsilon$. Donc si $x \in B(x_0, \delta)$, alors $F(x) \in B(y_0, \epsilon) \subset U$, donc $x \in F^{-1}(U)$. Ainsi $B(x_0, \delta) \subset F^{-1}(U)$, et $F^{-1}(U)$ est ouvert.

    \item $(2) \implies (3)$ : Par passage aux complémentaires.

    \item $(3) \implies (4)$ : Soit $(x_n)$ une suite dans $E_1$ avec $\lim_{n \to \infty} x_n = x \in E_1$. Supposons que $(F(x_n))$ ne converge pas vers $F(x)$. Alors il existe $\epsilon_0 > 0$ tel que pour tout $n \in \mathbb{N}$, il existe $p(n) > n$ avec $d_2(F(x_{p(n)}), F(x)) \geq \epsilon_0$. Soit $y_n = x_{p(n)}$ et $U = E_2 \setminus B(F(x), \epsilon_0)$. $U$ est fermé, $F(y_n) \in U$, donc $y_n \in F^{-1}(U)$, qui est fermé par propriété (3). Comme $(y_n)$ est une sous-suite de $(x_n)$ qui converge vers $x$, on a $\lim_{n \to \infty} y_n = x$. Puisque $F^{-1}(U)$ est fermé, on a $x \in F^{-1}(U)$. Donc $F(x) \in U = E_2 \setminus B(F(x), \epsilon_0)$, ce qui signifie $d_2(F(x), F(x)) \geq \epsilon_0$, ce qui est faux.

    \item $(4) \implies (1)$ : Supposons que $F$ n'est pas continue en $x_0 \in E_1$. Alors il existe $\epsilon_0 > 0$ tel que pour tout $\delta > 0$, il existe $x_\delta \in E_1$ avec $d_1(x_\delta, x_0) < \delta$ et $d_2(F(x_\delta), F(x_0)) \geq \epsilon_0$. En prenant $\delta = 1/n$, on obtient une suite $(x_n)_{n \geq 1}$ telle que $\lim_{n \to \infty} x_n = x_0$ et $d_2(F(x_n), F(x_0)) \geq \epsilon_0$ pour tout $n$. Ceci contredit (4).
\end{enumerate}
\end{proof}

\subsection{Exemple de fonction continue}

\begin{example}
Considérons $F : \mathbb{R}^2 \to \mathbb{R}$, $F(x, y) = x \sin(y) - e^x$. Les fonctions coordonnées $x$ et $y$, et les fonctions $\sin(y)$ et $e^x$ sont continues. Par composition et opérations algébriques, $F(x, y) = x \sin(y) - e^x$ est continue.
\end{example}

\section{Fonctions de plusieurs variables}

\subsection{Cadre $\mathbb{R}^n \to \mathbb{R}^p$}

Considérons des fonctions de $\mathbb{R}^n$ à valeurs dans $\mathbb{R}^p$. Le cadre est $\mathbb{R}^n$ pour la variable et $\mathbb{R}^p$ pour la valeur.

Soit $D \subset \mathbb{R}^n$ le domaine de définition. On considère des applications $F : D \to \mathbb{R}^p$.

\subsection{Continuité et composantes}

\begin{proposition}
$F : D \to \mathbb{R}^p$ est continue si et seulement si chaque composante $F_i : D \to \mathbb{R}$ est continue, où $F(x_1, \dots, x_n) = (F_1(x_1, \dots, x_n), \dots, F_p(x_1, \dots, x_n))$.
\end{proposition}

\subsection{Fonctions coordonnées}

Les fonctions coordonnées $(x_i)$ sont continues. Ce sont les $F_i$ pour $F(x) = x$ (l'identité).\chapter{CM6}
\sloppy

\section{Continuité}

\subsection{Définition de la continuité}

\begin{definition}
Soit $D \subset \mathbb{R}^d$, $f: D \rightarrow \mathbb{R}$ et $x_0 \in D$.
On dit que $f$ est continue en $x_0$ si
\[
\lim_{x \rightarrow x_0} f(x) = f(x_0).
\]
Plus précisément, pour tout $\epsilon > 0$, il existe $\alpha > 0$ tel que pour tout $x \in D$ avec $\|x - x_0\| \leq \alpha$, on a $|f(x) - f(x_0)| \leq \epsilon$.
\end{definition}

\subsection{Opérations sur les fonctions continues}
Si $f, g: D \rightarrow \mathbb{R}$ sont continues sur $D$, alors:
\begin{itemize}
    \item $f + g$ est continue sur $D$.
    \item $f \cdot g$ est continue sur $D$.
    \item Si $g(x) \neq 0$ pour tout $x \in D$, alors $\frac{f}{g}$ est continue sur $D$.
    \item Si $\varphi: I \rightarrow \mathbb{R}$ est continue sur $I \subset \mathbb{R}$ et $f(D) \subset I$, alors $\varphi \circ f$ est continue sur $D$.
\end{itemize}

\subsection{Continuité et compacité}

\begin{theorem}
Soit $K \subset \mathbb{R}^d$ un compact et $f: K \rightarrow \mathbb{R}^p$ une application continue.
Alors $f(K)$ est compact dans $\mathbb{R}^p$.
\end{theorem}

\begin{proposition}
Si $f: K \rightarrow \mathbb{R}$ est continue et $K \subset \mathbb{R}^d$ est compact, alors $f$ est bornée et atteint ses bornes.
\end{proposition}

\subsection{Continuité uniforme}
\begin{definition}
Une fonction $f: D \rightarrow \mathbb{R}^p$ est uniformément continue sur $D$ si pour tout $\epsilon > 0$, il existe $\alpha > 0$ tel que pour tout $x, y \in D$ avec $\|x - y\| \leq \alpha$, on a $\|f(x) - f(y)\| \leq \epsilon$.
\end{definition}

\subsection{Lien avec la compacité}

\begin{theorem}
Soit $F: \mathbb{R}^n \rightarrow \mathbb{R}^p$ continue et $K \subset \mathbb{R}^n$ compact. Alors $F(K)$ est compact dans $\mathbb{R}^p$.
\end{theorem}

\begin{remark}
Alors $F(K)$ compact dans $\mathbb{R}^p$ donc borné et atteint ses bornes.
\end{remark}

\subsection{Continuité partielle}

\begin{definition}
Soit $D \subset \mathbb{R}^n$ ouvert et $f: D \rightarrow \mathbb{R}$. On dit que $f$ est partiellement continue en $a = (a_1, \ldots, a_n) \in D$ si les fonctions partielles $f_i(t) = f(a_1, \ldots, a_{i-1}, t, a_{i+1}, \ldots, a_n)$ sont continues en $a_i$ pour tout $1 \leq i \leq n$.
On dit que $f$ est partiellement continue sur $D$ si $f$ est partiellement continue en tout point de $D$.
\end{definition}

\paragraph{\textbf{Non continuité et continuité partielle:}}
Considérons la fonction $f: \mathbb{R}^2 \rightarrow \mathbb{R}$ définie par
\[
f(x_1, x_2) = \begin{cases}
\frac{x_1 x_2}{x_1^2 + x_2^2} & \text{si } (x_1, x_2) \neq (0, 0) \\
0 & \text{si } (x_1, x_2) = (0, 0)
\end{cases}
\]
\begin{itemize}
    \item $f$ est continue sur $\mathbb{R}^2 \setminus \{(0, 0)\}$.
    \item $f$ est partiellement continue en $(0, 0)$. En effet, les fonctions partielles sont
    \begin{itemize}
        \item $f(x_1, 0) = \frac{x_1 \cdot 0}{x_1^2 + 0^2} = 0$ si $x_1 \neq 0$ et $f(0, 0) = 0$. Donc $f(x_1, 0) = 0$ pour tout $x_1 \in \mathbb{R}$.
        \item $f(0, x_2) = \frac{0 \cdot x_2}{0^2 + x_2^2} = 0$ si $x_2 \neq 0$ et $f(0, 0) = 0$. Donc $f(0, x_2) = 0$ pour tout $x_2 \in \mathbb{R}$.
    \end{itemize}
    Les fonctions partielles sont constantes nulles, donc continues en $0$.
    \item $f$ n'est pas continue en $(0, 0)$.
    En coordonnées polaires $x_1 = r \cos \theta$, $x_2 = r \sin \theta$, pour $(x_1, x_2) \neq (0, 0)$, on a
    \[
    f(r \cos \theta, r \sin \theta) = \frac{r \cos \theta \cdot r \sin \theta}{r^2 \cos^2 \theta + r^2 \sin^2 \theta} = \frac{r^2 \cos \theta \sin \theta}{r^2} = \cos \theta \sin \theta.
    \]
    Si $\theta$ est constant, alors $\lim_{r \rightarrow 0} f(r \cos \theta, r \sin \theta) = \cos \theta \sin \theta$ dépend de $\theta$. Par exemple:
    \begin{itemize}
        \item si $\theta = 0$, $\lim_{r \rightarrow 0} f(r \cos 0, r \sin 0) = 0$.
        \item si $\theta = \pi/4$, $\lim_{r \rightarrow 0} f(r \cos (\pi/4), r \sin (\pi/4)) = \cos (\pi/4) \sin (\pi/4) = \frac{1}{2}$.
    \end{itemize}
    La limite $\lim_{(x_1, x_2) \rightarrow (0, 0)} f(x_1, x_2)$ n'existe pas.
\end{itemize}

\begin{verbatim}
```python
#save_to: discont_ex.png
import matplotlib.pyplot as plt
import matplotlib.patches as patches
import numpy as np

fig, ax = plt.subplots()

ax.set_aspect('equal')

ax.spines['left'].set_position('zero')
ax.spines['bottom'].set_position('zero')
ax.spines['right'].set_color('none')
ax.spines['top'].set_color('none')

ax.xaxis.set_ticks_position('bottom')
ax.yaxis.set_ticks_position('left')

x = np.linspace(-1, 1, 400)
y = x**2
ax.plot(x, y, 'r', linewidth=2)

circle = patches.Circle((0, 0), radius=0.1, facecolor='white', edgecolor='black', linewidth=1.5, zorder=3)
ax.add_patch(circle)


ax.set_xlabel('$x_1$', loc='right')
ax.set_ylabel('$x_2$', loc='top')
ax.set_xticks([1])
ax.set_yticks([1])
ax.set_xlim([-0.5, 1.5])
ax.set_ylim([-0.5, 1.5])
ax.text(1.05, 0, '$x_1$')
ax.text(0, 1.05, '$x_2$')
ax.text(0.5, 0.5, '$f=1$')
ax.text(0.5, -0.1, '$f=0$')


plt.savefig('discont_ex.png')
```
\end{verbatim}

\begin{figure}[h]
\centering
\includegraphics[width=0.5\textwidth]{discont_ex.png}
\caption{Discontinuité en $(0,0)$}
\label{fig:discont_ex}
\end{figure}


\begin{remark}
La continuité implique la continuité partielle. La réciproque est fausse.
\end{remark}

\section{Dérivation des fonctions de plusieurs variables}

\subsection{Dérivabilité selon une direction}

\begin{definition}
Soit $D \subset \mathbb{R}^n$ un ouvert, $f: D \rightarrow \mathbb{R}$ et $x_0 \in D$, $u \in \mathbb{R}^n$. On dit que $f$ est dérivable au point $x_0$ dans la direction $u$ si la fonction $g(t) = f(x_0 + tu)$ est dérivable en $t = 0$.
\end{definition}

\subsection{Dérivées partielles}

\begin{definition}
On dit que $f$ admet des dérivées partielles en $x_0$ si $f$ est dérivable en $x_0$ dans les directions de la base canonique $e_1, \ldots, e_n$. On pose
\[
\frac{\partial f}{\partial x_i}(x_0) = \frac{d}{dt} f(x_0 + te_i) \Big|_{t=0}.
\]
\end{definition}

\paragraph{\textbf{Notation:}}
\[
\frac{\partial f}{\partial x_i}(x_0) = \partial_i f(x_0) = D_i f(x_0).
\]

\subsection{Différentiabilité}

\begin{definition}
Soit $D \subset \mathbb{R}^n$ un ouvert et $f: D \rightarrow \mathbb{R}$. On dit que $f$ est différentiable en $x_0 \in D$ s'il existe une application linéaire $L: \mathbb{R}^n \rightarrow \mathbb{R}$ telle que
\[
f(x_0 + h) = f(x_0) + L(h) + \|h\| \epsilon(h)
\]
avec $\lim_{h \rightarrow 0} \epsilon(h) = 0$.
On note $L = df(x_0) = Df(x_0)$.
\end{definition}

\begin{remark}
L'application linéaire $L$ est unique.
\end{remark}

\paragraph{\textbf{Gradient:}} L'application linéaire $L$ est de la forme $L(h) = \nabla f(x_0) \cdot h$ où $\nabla f(x_0)$ est le gradient de $f$ en $x_0$.

\begin{lemma}
Si $f$ est différentiable en $x_0$, alors $f$ est continue en $x_0$ et $f$ est dérivable dans toutes les directions en $x_0$ et
\[
\nabla f(x_0) = \begin{pmatrix}
\frac{\partial f}{\partial x_1}(x_0) \\
\vdots \\
\frac{\partial f}{\partial x_n}(x_0)
\end{pmatrix}.
\]
\end{lemma}

\subsection{Plan tangent}
Soit $S = \{(x, y, z) \in \mathbb{R}^3 : F(x, y, z) = 0\}$ une surface dans $\mathbb{R}^3$ et $x_0 \in S$.
Le plan tangent à $S$ en $x_0$ est donné par l'équation
\[
\nabla F(x_0) \cdot (x - x_0) = 0
\]
si $\nabla F(x_0) \neq 0$.

\subsection{Fonctions de classe $C^1$}
\begin{definition}
On dit que $f$ est de classe $C^1$ sur $D$ si $f$ est différentiable en tout point de $D$ et les fonctions $x \mapsto \frac{\partial f}{\partial x_i}(x)$ sont continues sur $D$ pour tout $1 \leq i \leq n$.
\end{definition}

\begin{theorem}
Si $f$ est de classe $C^1$ sur $D$, alors $f$ est différentiable sur $D$.
\end{theorem}

\begin{remark}
La réciproque est fausse. Une fonction peut être différentiable sans être $C^1$.
\end{remark}\end{document}