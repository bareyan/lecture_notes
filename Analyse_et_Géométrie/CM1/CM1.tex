```latex
\begin{document}
\sloppy

\section{Analyse}

\section{Introduction aux espaces vectoriels $\mathbb{R}^d$ et $\mathbb{C}^d$}

\subsection{Définitions et propriétés fondamentales}

Nous allons définir les espaces vectoriels $\mathbb{R}^d$ et $\mathbb{C}^d$.

\begin{definition}
L'espace $\mathbb{R}^d$, pour $d \geq 1$, est défini comme l'ensemble des $d$-uplets de nombres réels :
\[
\mathbb{R}^d := \{X = (x_1, \ldots, x_d) : x_i \in \mathbb{R}\}
\]
où $x_1, \ldots, x_d$ sont les coordonnées cartésiennes du point $X$.
\end{definition}

Pour $d=2$, on note parfois les points de $\mathbb{R}^2$ par $(x, y)$, et pour $d=3$, par $(x, y, z)$.

\begin{definition}
L'espace $\mathbb{C}^d$, pour $d \geq 1$, est défini de manière analogue comme l'ensemble des $d$-uplets de nombres complexes :
\[
\mathbb{C}^d := \{X = (x_1, \ldots, x_d) : x_i \in \mathbb{C}\}
\]
où $x_1, \ldots, x_d$ sont des nombres complexes.
\end{definition}

Tout nombre complexe $z \in \mathbb{C}$ peut être écrit sous la forme $z = a + bi$, où $a = Re(z)$ est la partie réelle de $z$ et $b = Im(z)$ est la partie imaginaire de $z$. On peut écrire pour $X \in \mathbb{C}^d$,
\[
X = Re(X) + i Im(X)
\]
où $Re(X) = (Re(x_1), \ldots, Re(x_d)) \in \mathbb{R}^d$ et $Im(X) = (Im(x_1), \ldots, Im(x_d)) \in \mathbb{R}^d$.

$\mathbb{R}^d$ et $\mathbb{C}^d$ sont des espaces vectoriels. L'addition vectorielle et la multiplication par un scalaire sont définies de la manière suivante dans $\mathbb{R}^d$ (et de même dans $\mathbb{C}^d$) :
Pour $X = (x_1, \ldots, x_d) \in \mathbb{R}^d$, $Y = (y_1, \ldots, y_d) \in \mathbb{R}^d$ et $\lambda \in \mathbb{R}$,
\begin{itemize}
    \item \textbf{Addition vectorielle} : $X + Y = (x_1 + y_1, \ldots, x_d + y_d)$
    \item \textbf{Multiplication par un scalaire} : $\lambda X = (\lambda x_1, \ldots, \lambda x_d)$
    \item \textbf{Vecteur nul} : $\overrightarrow{0} = (0, \ldots, 0)$
\end{itemize}
Pour $\mathbb{C}^d$, le corps des scalaires est $\mathbb{C}$.

\subsection{Coordonnées polaires et sphériques}

Dans $\mathbb{R}^2$, on peut utiliser les coordonnées polaires $(r, \theta) \in \mathbb{R}^+ \times [0, 2\pi[$ définies par :
\begin{align*}
    x &= r \cos \theta \\
    y &= r \sin \theta
\end{align*}
avec $r = \sqrt{x^2 + y^2}$.

Dans $\mathbb{R}^3$, on peut utiliser les coordonnées sphériques $(r, \theta, \varphi) \in \mathbb{R}^+ \times [-\pi/2, \pi/2] \times [0, 2\pi[$ définies par :
\begin{align*}
    x &= r \sin \theta \cos \varphi \\
    y &= r \sin \theta \sin \varphi \\
    z &= r \cos \theta
\end{align*}
avec $r = \sqrt{x^2 + y^2 + z^2}$.

\subsection{Représentation graphique de $\mathbb{R}^3$}

On peut représenter un point $X = (x, y, z)$ de $\mathbb{R}^3$ à l'aide des coordonnées sphériques.
Dans la figure ci-dessous, $\theta$ est l'angle entre l'axe $z$ et le vecteur $OX$, et $\varphi$ est l'angle entre l'axe $x$ et la projection de $OX$ sur le plan $(xOy)$.

\begin{verbatim}
```python
#save_to: R3_coordinates.png
import matplotlib.pyplot as plt
import matplotlib.patches as patches
import numpy as np

fig = plt.figure(figsize=(6,6))
ax = fig.add_subplot(111, projection='3d')

# Axes
ax.quiver(0, 0, 0, 1, 0, 0, color='k', arrow_length_ratio=0.05, linestyle='--')
ax.quiver(0, 0, 0, 0, 1, 0, color='k', arrow_length_ratio=0.05, linestyle='--')
ax.quiver(0, 0, 0, 0, 0, 1, color='k', arrow_length_ratio=0.05, linestyle='--')
ax.text(1.1, 0, 0, 'x')
ax.text(0, 1.1, 0, 'y')
ax.text(0, 0, 1.1, 'z')

# Point X
x, y, z = 0.8, 0.5, 0.6
ax.scatter(x, y, z, color='red', s=50)
ax.text(x*1.1, y*1.1, z*1.1, 'X')

# Vector OX
ax.quiver(0, 0, 0, x, y, z, color='blue', arrow_length_ratio=0.1)

# Projection on xy plane
ax.plot([0, x], [0, y], [0, 0], linestyle='--', color='gray')
ax.scatter(x, y, 0, color='gray', s=30)

# Angles
r = np.sqrt(x**2 + y**2 + z**2)
theta = np.arccos(z/r)
phi = np.arctan2(y, x)

# Arc for theta
arc_theta_radius = 0.5
arc_theta_start = np.arctan2(np.sqrt(x**2+y**2), z)
arc_theta = patches.Arc((0,0,0), arc_theta_radius, arc_theta_radius, 0, 0, np.degrees(theta))
ax.add_patch(arc_theta)
from mpl_toolkits.mplot3d import art3d
art3d.pathpatch_2d_to_3d(arc_theta, z=0, zdir="y")
ax.text(arc_theta_radius*0.6, -0.1, arc_theta_radius*0.6, r'$\theta$')


# Arc for phi
arc_phi_radius = 0.4
arc_phi_start = 0
arc_phi_end = np.degrees(phi)
arc_phi = patches.Arc((0,0,0), arc_phi_radius, arc_phi_radius, 0, 0, arc_phi_end)
ax.add_patch(arc_phi)
art3d.pathpatch_2d_to_3d(arc_phi, z=0, zdir="z")
ax.text(arc_phi_radius*0.7, arc_phi_radius*0.7, 0, r'$\varphi$')


ax.set_xlabel("x")
ax.set_ylabel("y")
ax.set_zlabel("z")
ax.set_xlim([-0.2, 1.2])
ax.set_ylim([-0.2, 1.2])
ax.set_zlim([-0.2, 1.2])
ax.view_init(elev=20, azim=-135)
plt.savefig('R3_coordinates.png')
```
\end{verbatim}

\begin{figure}[h]
    \centering
    \includegraphics[width=0.7\textwidth]{R3_coordinates.png}
    \caption{Représentation des coordonnées sphériques dans $\mathbb{R}^3$}
    \label{fig:R3_coordinates}
\end{figure}


\subsection{Produit scalaire dans $\mathbb{R}^d$}

\begin{definition}
Le produit scalaire de deux vecteurs $X = (x_1, \ldots, x_d) \in \mathbb{R}^d$ et $Y = (y_1, \ldots, y_d) \in \mathbb{R}^d$ est défini par :
\[
X \cdot Y = \sum_{i=1}^d x_i y_i
\]
\end{definition}

\subsection{Propriétés du produit scalaire dans $\mathbb{R}^d$}

Le produit scalaire dans $\mathbb{R}^d$ possède les propriétés suivantes :

\begin{enumerate}
    \item \textbf{Bilinéarité} :
    Pour tous vecteurs $X, Y, Z \in \mathbb{R}^d$ et scalaire $\lambda \in \mathbb{R}$,
    \begin{align*}
        (X + Y) \cdot Z &= X \cdot Z + Y \cdot Z \\
        Z \cdot (X + Y) &= Z \cdot X + Z \cdot Y \\
        (\lambda X) \cdot Y &= X \cdot (\lambda Y) = \lambda (X \cdot Y)
    \end{align*}
    \item \textbf{Symétrie} :
    Pour tous vecteurs $X, Y \in \mathbb{R}^d$,
    \[
    X \cdot Y = Y \cdot X
    \]
    \item \textbf{Positivité et caractère défini} :
    Pour tout vecteur $X \in \mathbb{R}^d$,
    \[
    X \cdot X \geq 0 \quad \text{et} \quad (X \cdot X = 0 \iff X = \overrightarrow{0})
    \]
\end{enumerate}

\begin{proposition}[Inégalité de Cauchy-Schwarz]
Pour tous vecteurs $X, Y \in \mathbb{R}^d$, on a l'inégalité de Cauchy-Schwarz :
\[
|X \cdot Y| \leq \sqrt{(X \cdot X)} \sqrt{(Y \cdot Y)}
\]
\end{proposition}

\begin{proof}
Fixons $X, Y \in \mathbb{R}^d$ et considérons la fonction polynomiale $P : \mathbb{R} \to \mathbb{R}$ définie par
\[
P(t) = (X + tY) \cdot (X + tY)
\]
En développant, on obtient
\[
P(t) = t^2 (Y \cdot Y) + 2t (X \cdot Y) + (X \cdot X)
\]
Comme $P(t) = ||X + tY||^2 \geq 0$ pour tout $t \in \mathbb{R}$, le polynôme $P(t)$ de degré 2 est toujours positif ou nul. Son discriminant $\Delta$ est donc négatif ou nul :
\[
\Delta = (2(X \cdot Y))^2 - 4 (Y \cdot Y) (X \cdot X) = 4 (X \cdot Y)^2 - 4 (X \cdot X) (Y \cdot Y) \leq 0
\]
D'où $(X \cdot Y)^2 \leq (X \cdot X) (Y \cdot Y)$, et en prenant la racine carrée, on obtient l'inégalité de Cauchy-Schwarz.
\end{proof}


\section{Normes}

\subsection{Norme associée au produit scalaire dans $\mathbb{R}^d$}

\begin{definition}
La norme euclidienne associée au produit scalaire dans $\mathbb{R}^d$ est définie par :
\[
\|X\| = \sqrt{X \cdot X} = \sqrt{\sum_{i=1}^d x_i^2}
\]
pour $X = (x_1, \ldots, x_d) \in \mathbb{R}^d$.
\end{definition}
Cette norme représente la longueur du vecteur $X$. Elle est parfois notée $\|X\|_2$.

\subsection{Propriétés des normes}

Les normes possèdent les propriétés suivantes :

\begin{enumerate}
    \item \textbf{Homogénéité} : Pour tout vecteur $X \in \mathbb{R}^d$ et scalaire $\lambda \in \mathbb{R}$,
    \[
    \|\lambda X\| = |\lambda| \|X\|
    \]
    \item \textbf{Inégalité triangulaire} : Pour tous vecteurs $X, Y \in \mathbb{R}^d$,
    \[
    \|X + Y\| \leq \|X\| + \|Y\|
    \]
    \item \textbf{Positivité et caractère défini} : Pour tout vecteur $X \in \mathbb{R}^d$,
    \[
    \|X\| \geq 0 \quad \text{et} \quad (\|X\| = 0 \iff X = \overrightarrow{0})
    \]
\end{enumerate}

\subsection{Définition générale d'une norme}

\begin{definition}
Une norme sur $\mathbb{R}^d$ est une application $N : \mathbb{R}^d \to \mathbb{R}^+$ satisfaisant les propriétés suivantes pour tous $X, Y \in \mathbb{R}^d$ et $\lambda \in \mathbb{R}$ :
\begin{enumerate}
    \item $N(\lambda X) = |\lambda| N(X)$
    \item $N(X + Y) \leq N(X) + N(Y)$
    \item $N(X) \geq 0$ et $(N(X) = 0 \iff X = \overrightarrow{0})$
\end{enumerate}
\end{definition}

\subsection{Exemples de normes sur $\mathbb{R}^d$}

Outre la norme euclidienne $\|X\| = \|X\|_2$, on peut définir d'autres normes sur $\mathbb{R}^d$ :

\begin{enumerate}
    \item \textbf{Norme 1} : $\|X\|_1 = \sum_{i=1}^d |x_i|$
    \item \textbf{Norme infini} : $\|X\|_\infty = \max_{1 \leq i \leq d} |x_i|$
\end{enumerate}


\section{Espace $\mathbb{C}^d$}

\subsection{Définition et produit scalaire hermitien}

Nous rappelons la définition de l'espace vectoriel $\mathbb{C}^d$ sur $\mathbb{C}$ :
\[
\mathbb{C}^d = \{X = (x_1, \ldots, x_d) : x_i \in \mathbb{C}\}
\]

\begin{definition}
Le produit scalaire hermitien de deux vecteurs $X = (x_1, \ldots, x_d) \in \mathbb{C}^d$ et $Y = (y_1, \ldots, y_d) \in \mathbb{C}^d$ est défini par :
\[
(X|Y) = \sum_{i=1}^d x_i \overline{y_i}
\]
\end{definition}
Notez la conjugaison complexe sur les composantes de $Y$. Certains auteurs utilisent la conjugaison sur les composantes de $X$.

\subsection{Propriétés du produit scalaire hermitien dans $\mathbb{C}^d$}

Le produit scalaire hermitien dans $\mathbb{C}^d$ possède les propriétés suivantes :

\begin{enumerate}
    \item \textbf{Sesquilinéarité} :
    Pour tous vecteurs $X, Y, Z \in \mathbb{C}^d$ et scalaires $\lambda, \mu \in \mathbb{C}$,
    \begin{align*}
        (X + Y|Z) &= (X|Z) + (Y|Z) \\
        (\lambda X|Z) &= \lambda (X|Z) \\
        (Z|X + Y) &= (Z|X) + (Z|Y) \\
        (Z|\lambda Y) &= \overline{\lambda} (Z|Y)
    \end{align*}
    On dit que le produit scalaire est linéaire par rapport au premier argument (à gauche) et antilinéaire par rapport au deuxième argument (à droite). Certains auteurs choisissent la convention inverse.

    \item \textbf{Hermiticité (ou symétrie hermitienne)} :
    Pour tous vecteurs $X, Y \in \mathbb{C}^d$,
    \[
    (X|Y) = \overline{(Y|X)}
    \]

    \item \textbf{Positivité et caractère défini} :
    Pour tout vecteur $X \in \mathbb{C}^d$,
    \[
    (X|X) = \sum_{i=1}^d |x_i|^2 \geq 0 \quad \text{et} \quad ((X|X) = 0 \iff X = \overrightarrow{0})
    \]
\end{enumerate}


\section{Norme Hilbertienne dans $\mathbb{C}^d$}

\begin{definition}
La norme Hilbertienne (ou norme euclidienne) associée au produit scalaire hermitien dans $\mathbb{C}^d$ est définie par :
\[
\|X\| = \sqrt{(X|X)} = \sqrt{\sum_{i=1}^d |x_i|^2}
\]
pour $X = (x_1, \ldots, x_d) \in \mathbb{C}^d$.
\end{definition}
On a $\|X\|^2 = \sum_{i=1}^d |x_i|^2 = \|Re(X)\|^2 + \|Im(X)\|^2$.

\begin{lemma}
Pour tout $X \in \mathbb{C}^d$, on a :
\[
\|X\| = \sup_{\|Y\| \leq 1} |(X|Y)|
\]
\end{lemma}

\begin{proof}
Par l'inégalité de Cauchy-Schwarz appliquée au produit scalaire hermitien (qui est aussi valable dans $\mathbb{C}^d$), on a
\[
|(X|Y)| \leq \sqrt{(X|X)} \sqrt{(Y|Y)} = \|X\| \|Y\|
\]
Si $\|Y\| \leq 1$, alors $|(X|Y)| \leq \|X\|$. Donc $\sup_{\|Y\| \leq 1} |(X|Y)| \leq \|X\|$.

Inversement, considérons $X \neq \overrightarrow{0}$ (si $X = \overrightarrow{0}$, l'égalité est triviale). Posons $Y = \frac{X}{\|X\|}$. Alors $\|Y\| = 1$ et
\[
(X|Y) = \left(X \Big| \frac{X}{\|X\|} \right) = \frac{1}{\|X\|} (X|X) = \frac{\|X\|^2}{\|X\|} = \|X\|
\]
Donc, il existe $Y$ avec $\|Y\| \leq 1$ tel que $|(X|Y)| = \|X\|$. D'où $\sup_{\|Y\| \leq 1} |(X|Y)| \geq \|X\|$.

En conclusion, on a bien l'égalité $\|X\| = \sup_{\|Y\| \leq 1} |(X|Y)|$.
\end{proof}

\subsection{Autres normes sur $\mathbb{C}^d$}

De même que pour $\mathbb{R}^d$, on peut définir les normes $\|\cdot\|_1$ et $\|\cdot\|_\infty$ sur $\mathbb{C}^d$ :
\begin{itemize}
    \item Norme 1 : $\|X\|_1 = \sum_{i=1}^d |x_i|$
    \item Norme infini : $\|X\|_\infty = \max_{1 \leq i \leq d} |x_i|$
\end{itemize}


\section{Distances}

\subsection{Distance induite par une norme}

\begin{definition}
Soit $\|\cdot\|$ une norme sur $\mathbb{R}^d$ (ou $\mathbb{C}^d$). La distance induite par cette norme entre deux points $X, Y$ est définie par :
\[
d(X, Y) = \|Y - X\|
\]
\end{definition}

\subsection{Propriétés d'une distance}

Une distance $d : E \times E \to \mathbb{R}^+$ sur un ensemble $E$ doit satisfaire les propriétés suivantes :

\begin{definition}
Une application $d: E \times E \to \mathbb{R}^+$ est une distance sur $E$ si elle satisfait pour tous $x, y, z \in E$:
\begin{enumerate}
    \item \textbf{Positivité} : $d(x, y) \geq 0$
    \item \textbf{Symétrie} : $d(x, y) = d(y, x)$
    \item \textbf{Inégalité triangulaire} : $d(x, y) \leq d(x, z) + d(z, y)$
    \item \textbf{Séparation} : $d(x, y) = 0 \iff x = y$
\end{enumerate}
\end{definition}

\subsection{Exemples de distances}

\begin{enumerate}
    \item \textbf{Distance euclidienne} (ou distance $d_2$) : induite par la norme euclidienne $\|\cdot\|_2$.
    \[
    d_2(X, Y) = \|X - Y\|_2 = \sqrt{\sum_{i=1}^d (x_i - y_i)^2}
    \]
    \item \textbf{Distance $d_1$} : induite par la norme $\|\cdot\|_1$.
    \[
    d_1(X, Y) = \|X - Y\|_1 = \sum_{i=1}^d |x_i - y_i|
    \]
    \item \textbf{Distance $d_\infty$} : induite par la norme $\|\cdot\|_\infty$.
    \[
    d_\infty(X, Y) = \|X - Y\|_\infty = \max_{1 \leq i \leq d} |x_i - y_i|
    \]
    \item \textbf{Distance logarithmique} sur $\mathbb{R}^+ = ]0, +\infty[$ :
    \[
    d_{log}(a, b) = |\ln(b/a)| = |\ln(b) - \ln(a)|
    \]
    En base 10 : $d_{log_{10}}(x, y) = |\log_{10}(y/x)|$.

    \item \textbf{Distance SNCF} sur un ensemble de points. Si on considère trois points $0, X, Y$ non alignés :

\begin{verbatim}
```python
#save_to: SNCF_distance.png
import matplotlib.pyplot as plt

fig, ax = plt.subplots()

points = {'O': (0, 0), 'X': (2, 0), 'Y': (2, 2)}

# Draw points
for name, coord in points.items():
    ax.plot(*coord, 'o', markersize=8, label=name)
    ax.annotate(name, coord, textcoords="offset points", xytext=(0,5), ha='center')

# Connect O-X and O-Y
ax.plot([points['O'][0], points['X'][0]], [points['O'][1], points['X'][1]], 'k-')
ax.plot([points['O'][0], points['Y'][0]], [points['O'][1], points['Y'][1]], 'k-')

ax.set_xlim([-1, 4])
ax.set_ylim([-1, 3])
ax.set_aspect('equal', adjustable='box')
ax.axis('off')
plt.savefig('SNCF_distance.png')
```
\end{verbatim}

\begin{figure}[H]
    \centering
    \includegraphics[width=0.3\textwidth]{SNCF_distance.png}
    \caption{Distance SNCF}
    \label{fig:sncf_distance}
\end{figure}
    La distance SNCF entre $X$ et $Y$ est définie par
    \[
    d_{SNCF}(X, Y) = d(0, X) + d(0, Y)
    \]

    \item \textbf{Distance usuelle dans $\mathbb{R}^2$} :
    \[
    \delta(X, Y) =
    \begin{cases}
        d(X, Y) & \text{si } 0, X, Y \text{ alignés} \\
        d(0, X) + d(0, Y) & \text{sinon}
    \end{cases}
    \]
\end{enumerate}


\section{Espaces métriques}

\begin{definition}
Un espace métrique est un couple $(E, d)$ où $E$ est un ensemble et $d : E \times E \to \mathbb{R}^+$ est une distance sur $E$.
\end{definition}

\subsection{Inégalité triangulaire inverse}
Dans un espace métrique $(E, d)$, on a l'inégalité triangulaire inverse :
Pour tous $x, y, z \in E$,
\[
|d(x, z) - d(y, z)| \leq d(x, y)
\]

\subsection{Distance induite sur un sous-ensemble}
Si $(E, d)$ est un espace métrique et $U \subset E$ est un sous-ensemble de $E$, la restriction de $d$ à $U \times U$ fait de $(U, d)$ un espace métrique.

\section{Boules dans les espaces métriques}

Soit $(E, d)$ un espace métrique, $x_0 \in E$ et $r \geq 0$.

\begin{definition}
\begin{enumerate}
    \item La \textbf{boule ouverte} de centre $x_0$ et de rayon $r$ est l'ensemble
    \[
    B(x_0, r) = \{x \in E : d(x_0, x) < r\}
    \]
    \item La \textbf{boule fermée} de centre $x_0$ et de rayon $r$ est l'ensemble
    \[
    B_f(x_0, r) = \{x \in E : d(x_0, x) \leq r\}
    \]
    \item La \textbf{sphère} de centre $x_0$ et de rayon $r$ est l'ensemble
    \[
    S(x_0, r) = \{x \in E : d(x_0, x) = r\}
    \]
\end{enumerate}
\end{definition}

\subsection{Propriétés des boules}

\begin{lemma}
\begin{enumerate}
    \item $B(x_0, 0) = \emptyset$ et $B_f(x_0, 0) = \{x_0\}$
    \item Pour $0 \leq r_1 < r_2$, on a $B(x_0, r_1) \subset B_f(x_0, r_1) \subset B(x_0, r_2)$
    \item Si $d(x_0, x_1) + r_1 \leq r$, alors $B(x_1, r_1) \subset B(x_0, r)$
\end{enumerate}
\end{lemma}

\begin{proof}
\begin{enumerate}
    \item Les propriétés (1) et (2) sont évidentes d'après la définition des boules.

    \item Pour montrer (3), supposons $x \in B(x_1, r_1)$. Alors $d(x_1, x) < r_1$.
    Par l'inégalité triangulaire, on a
    \[
    d(x_0, x) \leq d(x_0, x_1) + d(x_1, x) < d(x_0, x_1) + r_1 \leq r
    \]
    Donc $d(x_0, x) < r$, ce qui signifie que $x \in B(x_0, r)$. D'où $B(x_1, r_1) \subset B(x_0, r)$.
\end{enumerate}
\end{proof}


\end{document}
```