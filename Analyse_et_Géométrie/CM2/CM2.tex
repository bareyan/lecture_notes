```latex
\documentclass{article}
\usepackage{amssymb,amsmath,amsthm}
\usepackage{graphicx}
\usepackage{color}
\usepackage{float}
\usepackage{fancyhdr}
\usepackage{array}
\usepackage{listings}

\newtheorem{theorem}{Theorem}
\newtheorem{lemma}{Lemma}
\newtheorem{proposition}{Proposition}
\newtheorem{definition}{Definition}
\newtheorem{remark}{Remark}
\newtheorem{solution}{Solution}
\newtheorem{example}{Example}

\usepackage[margin=1in]{geometry}

\begin{document}
\sloppy

\section{Espaces métriques}

\begin{definition}
Soit $E$ un ensemble. Une application $d : E \times E \to \mathbb{R}^+$ est appelée distance sur $E$ si :
\begin{enumerate}
    \item $d(x, y) \geq 0$ (positivité)
    \item $d(x, y) = d(y, x)$ (symétrie)
    \item $d(x, y) \leq d(x, z) + d(z, y)$ (inégalité triangulaire)
    \item $d(x, y) = 0 \Leftrightarrow x = y$ (axiome de séparation)
\end{enumerate}
$(E, d)$ est appelé espace métrique.
\end{definition}

\begin{proposition}
[Inégalité triangulaire]
Dans un espace métrique $(E, d)$, on a aussi l'inégalité suivante:
\[
|d(x, y) - d(x, z)| \leq d(y, z)
\]
\end{proposition}

\subsection{Exemples}

\begin{example}
\begin{enumerate}
    \item $E = \mathbb{R}$. On définit $d(x, y) = |x - y|$. \\
    Boule $B(x_0, r) = \{x \in \mathbb{R} : d(x, x_0) < r \} = ]x_0 - r, x_0 + r[$.

    \item $E = \mathbb{R}^d$, $d = 2, 3, \dots$. On a différentes normes :
    \begin{itemize}
        \item Norme euclidienne: $||x||_2 = (\sum_{i=1}^d x_i^2)^{1/2}$
        \item Norme $1$: $||x||_1 = \sum_{i=1}^d |x_i|$
        \item Norme $\infty$: $||x||_\infty = \max_{1 \leq i \leq d} |x_i|$
    \end{itemize}
    Pour $E = \mathbb{R}^d$, on définit la distance $d_2(x, y) = ||y - x||_2 = ||\overrightarrow{xy}||_2$. De même, on peut définir $d_1(x, y) = ||y - x||_1$ et $d_\infty(x, y) = ||y - x||_\infty$.

    Boule $B_2(0, r)$ pour $d_2$ dans $\mathbb{R}^2$:

    \begin{verbatim}
    ```python
#save_to: boule_d2.png
import matplotlib.pyplot as plt
import matplotlib.patches as patches
import numpy as np

r=1
fig, ax = plt.subplots()
circle = patches.Circle((0, 0), r, edgecolor='black', facecolor='lightgray', fill=True, linewidth=1)
ax.add_patch(circle)
ax.axhline(0, color='black',linewidth=0.5)
ax.axvline(0, color='black',linewidth=0.5)
ax.set_aspect('equal', adjustable='box')
ax.set_xlim([-1.5*r, 1.5*r])
ax.set_ylim([-1.5*r, 1.5*r])
ax.set_xticks([-r, 0, r])
ax.set_yticks([-r, 0, r])
ax.set_xlabel('$x$')
ax.set_ylabel('$y$')
ax.set_title('$B_2(0, r)$')
plt.savefig('boule_d2.png')
    ```
    \end{verbatim}

    \begin{figure}[H]
        \centering
        \includegraphics[width=0.5\textwidth]{boule_d2.png}
        \caption{Boule $B_2(0, r)$ dans $\mathbb{R}^2$}
        \label{fig:boule_d2}
    \end{figure}

    Boule $B_\infty(0, r)$ pour $d_\infty$ dans $\mathbb{R}^2$:

    \begin{verbatim}
    ```python
#save_to: boule_d_infini.png
import matplotlib.pyplot as plt
import matplotlib.patches as patches
import numpy as np

r=1
fig, ax = plt.subplots()
square = patches.Rectangle((-r, -r), 2*r, 2*r, edgecolor='black', facecolor='lightgray', fill=True, linewidth=1)
ax.add_patch(square)
ax.axhline(0, color='black',linewidth=0.5)
ax.axvline(0, color='black',linewidth=0.5)
ax.set_aspect('equal', adjustable='box')
ax.set_xlim([-1.5*r, 1.5*r])
ax.set_ylim([-1.5*r, 1.5*r])
ax.set_xticks([-r, 0, r])
ax.set_yticks([-r, 0, r])
ax.set_xlabel('$x$')
ax.set_ylabel('$y$')
ax.set_title('$B_\infty(0, r)$')
plt.savefig('boule_d_infini.png')
    ```
    \end{verbatim}

    \begin{figure}[H]
        \centering
        \includegraphics[width=0.5\textwidth]{boule_d_infini.png}
        \caption{Boule $B_\infty(0, r)$ dans $\mathbb{R}^2$}
        \label{fig:boule_d_infini}
    \end{figure}

        Boule $B_1(0, r)$ pour $d_1$ dans $\mathbb{R}^2$:

    \begin{verbatim}
    ```python
#save_to: boule_d1.png
import matplotlib.pyplot as plt
import matplotlib.patches as patches
import numpy as np

r=1
fig, ax = plt.subplots()
diamond = patches.Polygon([[r, 0], [0, r], [-r, 0], [0, -r]], closed=True, edgecolor='black', facecolor='lightgray', fill=True, linewidth=1)
ax.add_patch(diamond)
ax.axhline(0, color='black',linewidth=0.5)
ax.axvline(0, color='black',linewidth=0.5)
ax.set_aspect('equal', adjustable='box')
ax.set_xlim([-1.5*r, 1.5*r])
ax.set_ylim([-1.5*r, 1.5*r])
ax.set_xticks([-r, 0, r])
ax.set_yticks([-r, 0, r])
ax.set_xlabel('$x$')
ax.set_ylabel('$y$')
ax.set_title('$B_1(0, r)$')
plt.savefig('boule_d1.png')
    ```
    \end{verbatim}

    \begin{figure}[H]
        \centering
        \includegraphics[width=0.5\textwidth]{boule_d1.png}
        \caption{Boule $B_1(0, r)$ dans $\mathbb{R}^2$}
        \label{fig:boule_d1}
    \end{figure}

    \begin{remark}
    \textbf{Important:} notion de proximité, pas la forme.
    \end{remark}

    Dans $\mathbb{R}^n$, on a les relations entre les distances:
    \begin{align*}
        d_\infty(x, y) &\leq d_1(x, y) \leq n d_\infty(x, y) \\
        d_\infty(x, y) &\leq d_2(x, y) \leq \sqrt{n} d_\infty(x, y)
    \end{align*}
\end{enumerate}
\end{example}

\section{Parties bornées}

\begin{definition}
Soit $(E, d)$ un espace métrique et $A \subset E$. $A$ est dite \textbf{bornée} si
\[
\exists R > 0 \text{ et } \exists x_0 \in E \text{ tel que } A \subset B(x_0, R).
\]
\end{definition}

\begin{lemma}
Les propriétés suivantes sont équivalentes:
\begin{enumerate}
    \item $A$ est bornée.
    \item $\forall x_0 \in E, \exists r > 0$ tel que $A \subset B(x_0, r)$.
    \item $\exists r > 0$ tel que $\forall x, y \in A$, on a $d(x, y) < r$.
\end{enumerate}
\end{lemma}

\begin{solution}
[Démonstration:]
\textbf{1) $\Rightarrow$ 2) $\Rightarrow$ 3) $\Rightarrow$ 1)}

\textbf{Preuve que 1) $\Rightarrow$ 2)}

Hyp: $\exists x_1 \in E, \exists r_1 > 0$ tel que $A \subset B(x_1, r_1)$.

Soit $x_0 \in E$. But: trouver $r$ tel que $A \subset B(x_0, r)$.

Si $x \in A$, alors $x \in B(x_1, r_1)$, on a $d(x, x_1) < r_1$.

On veut $d(x_0, x) < r$.

\begin{align*}
    d(x_0, x) &\leq d(x_0, x_1) + d(x_1, x) \\
              &\leq d(x_0, x_1) + r_1 \\
              &< r \text{ si } r > d(x_0, x_1) + r_1
\end{align*}
Il suffit de prendre $r = d(x_0, x_1) + r_1$.

\textbf{2) $\Rightarrow$ 3)}

On fixe $x_0 \in E$. D'après 2), $\exists r_0 > 0$ tel que $A \subset B(x_0, r_0)$.
Alors $\forall x, y \in A$,
\begin{align*}
    d(x, y) &\leq d(x, x_0) + d(x_0, y) \\
              &< r_0 + r_0 = 2r_0
\end{align*}
On prend $r = 2r_0$.

\textbf{3) $\Rightarrow$ 1)}

On fixe $x_0 \in E$ (n'importe lequel). D'après 3), $\exists r > 0$ tel que $\forall x, y \in A$, $d(x, y) < r$.
Alors $\forall x \in A$, $d(x_0, x) \leq d(x_0, y) + d(y, x) < d(x_0, y) + r$.
On fixe $y \in A$. Alors $d(x_0, y) < \infty$ est fixe. On prend $R = d(x_0, y) + r$.
Alors $\forall x \in A$, $d(x_0, x) < R$, donc $A \subset B(x_0, R)$.
\end{solution}

\begin{proposition}
[Propriétés élémentaires]
\begin{enumerate}
    \item Toute partie finie est bornée.
    \item Si $A$ bornée et $B \subset A$ alors $B$ bornée.
    \item L'union d'un nb fini de bornées est bornée.
\end{enumerate}
\end{proposition}

\begin{solution}
[Démonstration:]
\textbf{1) Tout partie finie est bornée}

Soit $A = \{a_1, \dots, a_n\}$ une partie finie de $E$. On fixe $x_0 \in E$.
Pour chaque $a_i$, $d(x_0, a_i) < \infty$. Soit $r_i = d(x_0, a_i) + 1$. Alors $a_i \in B(x_0, r_i)$.
On prend $R = \max_{1 \leq i \leq n} r_i$. Alors $a_i \in B(x_0, R)$ pour tout $i$.
Donc $A \subset B(x_0, R)$.

\textbf{2) Si $A$ bornée et $B \subset A$ alors $B$ bornée}

Si $A$ est bornée, $\exists x_0, R$ tq $A \subset B(x_0, R)$. Comme $B \subset A$, on a $B \subset B(x_0, R)$. Donc $B$ est bornée.

\textbf{3) L'union d'un nb fini de bornées est bornée (Partie)}

Soient $A_1, \dots, A_n$ sont bornées. Je fixe $x_0 \in E$.
$A_i$ bornée ($1 \leq i \leq n$) donc $\exists r_i > 0$ tel que $A_i \subset B(x_0, r_i)$.
Soit $r = \max_{1 \leq i \leq n} r_i$.
Si $x \in \bigcup_{i=1}^n A_i$, alors $x \in A_i$ pour un $i$. Donc $x \in B(x_0, r_i) \subset B(x_0, r)$.
Donc $\bigcup_{i=1}^n A_i \subset B(x_0, r)$.
\end{solution}

\section{Fonctions bornées}

\begin{definition}
Soit $B$ un ensemble. Une fonction $F: B \to E$ est \textbf{bornée} si
\[
F(B) = \{F(b) : b \in B\} \subset E
\]
est bornée.
\end{definition}

\section{Distances entre ensembles}

\begin{definition}
Soit $A, B$ deux parties de $E$. On pose
\[
d(A, B) = \inf_{\substack{x \in A \\ y \in B}} d(x, y).
\]
\end{definition}

\begin{remark}
$\forall x \in A, y \in B$, $d(A, B) \leq d(x, y)$.
$\forall \epsilon > 0$, $\exists x \in A, y \in B$ tq $d(x, y) \leq d(A, B) + \epsilon$.
\end{remark}

\begin{proposition}
[Notation Proposition]
\[
d(x, A) = \inf_{y \in A} d(x, y) = d(\{x\}, A).
\]
\end{proposition}

\section{Topologie des espaces métriques}

\textbf{Concepts importants:} distance $\to$ boules $B(x_0, r)$ $\to$ ensembles ouverts.

\begin{definition}
Soit $(E, d)$ espace métrique.
\begin{enumerate}
    \item $U \subset E$ est \textbf{ouvert} si
    \[
    \forall x_0 \in U, \exists r > 0 \text{ tel que } B(x_0, r) \subset U.
    \]
    \item $F \subset E$ est \textbf{fermé} si $E \setminus F$ est ouvert.
\end{enumerate}
\end{definition}

\begin{remark}
Dans $\mathbb{R}$ les intervalles ouverts sont des ouverts.
\end{remark}

\textbf{Comment montrer que l'ensemble est ouvert ou fermé.} Dans le poly.

\begin{itemize}
    \item $\emptyset$ est ouvert (par définition).
    \item $E$ est ouvert.
    \item $E$ est fermé, $\emptyset$ est fermé (comme complémentaires d'ouverts).
\end{itemize}

\section{Lemmes et théorèmes}

\begin{lemma}
$1) B(x_0, r)$ est ouvert.

$2) B_f(x_0, r)$ est fermé.
\end{lemma}

\begin{solution}
[Démo dans le poly]
\end{solution}

\begin{example}
$E = \mathbb{R}$, $d(x, y) = |y - x|$.
$A = ]0, 1[$ ouvert dans $\mathbb{R}$.
$A = ]0, 1[$ pas fermé dans $A$.
$R \setminus A = ]-\infty, 0] \cup [1, \infty[$ fermé dans $\mathbb{R}$.

Je regarde $A$ comme partie de $(A, d)$.
$A$ est fermé dans $A$.
\end{example}

\begin{theorem}
\begin{enumerate}
    \item Soit $U_i, i \in I$ une collection d'ouverts. Alors $\bigcup_{i \in I} U_i$ est ouvert (l'union quelconque d'ouverts est un ouvert).
    \item Si $U_1, \dots, U_n$ sont ouverts, alors $\bigcap_{i=1}^n U_i$ est ouvert (l'intersection d'une famille finie d'ouverts est ouvert).
    \item Si $F_i, i \in I$ sont fermés, alors $\bigcap_{i \in I} F_i$ est fermé (l'intersection quelconque de fermés est fermé).
    \item $F_1, \dots, F_n$ fermés, alors $\bigcup_{i=1}^n F_i$ est fermé.
\end{enumerate}
\end{theorem}

\begin{example}
[Examples et remarques]
$U_i = ]-\frac{1}{i}, \frac{1}{i}[$, $i \geq 0$.
$\bigcap_{i \in \mathbb{N}} U_i = \{0\}$ pas ouvert dans $\mathbb{R}$.

$F_i = [0, 1 - \frac{1}{i}]$ fermé dans $\mathbb{R}$.
$\bigcup_{i \in \mathbb{N}} F_i = [0, 1[$ pas fermé dans $\mathbb{R}$.
\end{example}

\begin{solution}
[Dem:]
\textbf{1) Soit $x \in \bigcup_{i \in I} U_i = U$}. Il existe un $i$ noté $i_0$ tel que $x \in U_{i_0}$.
$U_{i_0}$ est ouvert donc $\exists r > 0$ tel que $B(x, r) \subset U_{i_0} \subset \bigcup_{i \in I} U_i = U$. Donc $U$ est ouvert.

\textbf{2) Soit $x \in \bigcap_{i=1}^n U_i = U$}. $x \in U_i$ pour $1 \leq i \leq n$. $U_i$ ouvert donc $\exists r_i > 0$ tel que $B(x, r_i) \subset U_i$.
Soit $r = \min_{1 \leq i \leq n} r_i > 0$.
$B(x, r) \subset B(x, r_i) \subset U_i$ $1 \leq i \leq n$.
Donc $B(x, r) \subset \bigcap_{i=1}^n U_i = U$.
Donc $B(x, r) \subset U$.
\end{solution}

\end{document}
```