```latex
\documentclass{article}
\usepackage{amssymb,amsmath,amsthm}
\usepackage{graphicx}
\usepackage{color}
\usepackage{float}
\usepackage{fancyhdr}
\usepackage{array}
\usepackage{listings} % Added based on user preamble
\usepackage[french]{babel} % Added for French typesetting
\usepackage[utf8]{inputenc} % Added for UTF-8 input
\usepackage[T1]{fontenc} % Added for font encoding

\newtheorem{theorem}{Théorème} % Changed to French
\newtheorem{lemma}{Lemme} % Changed to French
\newtheorem{proposition}{Proposition} % Changed to French
\newtheorem{definition}{Définition} % Changed to French
\newtheorem{remark}{Remarque} % Changed to French
\newtheorem{solution}{Solution} % Changed to French
\newtheorem{example}{Exemple} % Changed to French
\theoremstyle{definition} % Example, Solution, etc. should be in definition style
\newtheorem{corollary}{Corollaire} % Added for completeness

\usepackage[margin=1in]{geometry}

\begin{document}
\sloppy

\section{Espaces Vectoriels Normés}

\subsection{Norme Uniforme (Sup)}
Soit $E = C^0([a,b], \mathbb{R})$ l'espace des fonctions continues sur $[a,b]$ à valeurs réelles.
La norme uniforme (ou norme sup) est définie par :
\[ \|u\|_\infty = \sup_{x \in [a,b]} |u(x)| \]
Cette norme est aussi appelée $\|.\|_U$ dans les notes.

Considérons l'intégrale $I(u) = \int_a^b u(x) dx$. Montrons que $I$ est une application linéaire continue de $(E, \|.\|_\infty)$ dans $(\mathbb{R}, |.|)$.

\begin{proposition}
L'application $I: (E, \|.\|_\infty) \to (\mathbb{R}, |. |)$ définie par $I(u) = \int_a^b u(x) dx$ est linéaire et continue.
\end{proposition}

\begin{proof}
Linéarité : Pour $\lambda, \mu \in \mathbb{R}$ et $u, v \in E$,
\begin{align*} I(\lambda u + \mu v) &= \int_a^b (\lambda u(x) + \mu v(x)) dx \\ &= \lambda \int_a^b u(x) dx + \mu \int_a^b v(x) dx \\ &= \lambda I(u) + \mu I(v) \end{align*}
Continuité (Bornée) :
\begin{align*} |I(u)| &= \left| \int_a^b u(x) dx \right| \\ &\le \int_a^b |u(x)| dx \\ &\le \int_a^b \sup_{t \in [a,b]} |u(t)| dx \\ &= \int_a^b \|u\|_\infty dx \\ &= (b-a) \|u\|_\infty \end{align*}
Donc $|I(u)| \le C \|u\|_\infty$ avec $C = (b-a)$. L'application $I$ est donc continue (bornée).
\end{proof}

Calcul de la norme de $I$.

\begin{proposition}
La norme de l'application linéaire $I$ est $\|I\| = (b-a)$.
\end{proposition}

\begin{proof}
On sait que $\|I\| = \sup_{\|u\|_\infty = 1} |I(u)|$.
On a montré $|I(u)| \le (b-a) \|u\|_\infty$, donc $\|I\| \le (b-a)$.
Prenons la fonction constante $u_0(x) = 1$. Alors $\|u_0\|_\infty = 1$.
$I(u_0) = \int_a^b 1 dx = (b-a)$.
Donc $|I(u_0)| = (b-a) = (b-a) \|u_0\|_\infty$.
Ainsi, le sup est atteint et $\|I\| = (b-a)$.
\end{proof}

\subsection{Convergence Uniforme}
Soit $(f_n)$ une suite de fonctions dans $C^0([a,b], \mathbb{R})$.
\begin{definition}
On dit que $(f_n)$ converge uniformément vers $f$ si $\|f_n - f\|_\infty \to 0$.
\end{definition}

Ceci est équivalent à :
\[ \forall \epsilon > 0, \exists N \in \mathbb{N} \text{ tel que } \forall n \ge N, \forall x \in [a,b], |f_n(x) - f(x)| \le \epsilon \]
La convergence uniforme implique la convergence simple.

\begin{lemma}
Si $(f_n)$ converge uniformément vers $f$ sur $[a,b]$ et si chaque $f_n$ est continue, alors $f$ est continue.
L'espace $(C^0([a,b], \mathbb{R}), \|.\|_\infty)$ est un espace de Banach (complet).
\end{lemma}

\begin{lemma}
Si $(f_n)$ converge uniformément vers $f$ sur $[a,b]$, alors
\[ \lim_{n \to \infty} \int_a^b f_n(x) dx = \int_a^b \lim_{n \to \infty} f_n(x) dx = \int_a^b f(x) dx \]
\end{lemma}
\begin{proof}
On veut montrer que $\left| \int_a^b f_n(x) dx - \int_a^b f(x) dx \right| \to 0$.
\begin{align*} \left| \int_a^b (f_n(x) - f(x)) dx \right| &\le \int_a^b |f_n(x) - f(x)| dx \\ &\le \int_a^b \sup_{t \in [a,b]} |f_n(t) - f(t)| dx \\ &= \int_a^b \|f_n - f\|_\infty dx \\ &= (b-a) \|f_n - f\|_\infty \end{align*}
Comme $\|f_n - f\|_\infty \to 0$ par convergence uniforme, on a $(b-a) \|f_n - f\|_\infty \to 0$.
\end{proof}

\subsection{Exemple de Calcul de Norme}

Considérons $E = C^0([-1,1], \mathbb{R})$ muni de la norme $\|.\|_\infty$.
Soit $A: E \to E$ définie par $(Au)(x) = \int_0^x u(t) dt$. $A$ est linéaire.

\begin{proposition}
L'opérateur $A: E \to E$ défini par $(Au)(x) = \int_0^x u(t) dt$ est borné et sa norme $\|A\| = 1$ (pour la norme $\|.\|_\infty$).
\end{proposition}
\begin{proof}
Montrons que $A$ est bornée et calculons sa norme $\|A\|$.
\begin{align*} |(Au)(x)| &= \left| \int_0^x u(t) dt \right| \\ &\le \left| \int_0^x |u(t)| dt \right| \\ &\le \left| \int_0^x \|u\|_\infty dt \right| \\ &= \|u\|_\infty |x| \end{align*}
Donc $\|Au\|_\infty = \sup_{x \in [-1,1]} |(Au)(x)| \le \sup_{x \in [-1,1]} \|u\|_\infty |x| = \|u\|_\infty \sup_{x \in [-1,1]} |x| = \|u\|_\infty \times 1$.
On a $\|Au\|_\infty \le 1 \times \|u\|_\infty$. Donc $A$ est bornée et $\|A\| \le 1$.

Pour montrer que $\|A\| = 1$, cherchons une fonction $u$ telle que $\|u\|_\infty = 1$ et $\|Au\|_\infty$ est proche de 1.
Prenons $u_0(x) = 1$. Alors $\|u_0\|_\infty = 1$.
$(Au_0)(x) = \int_0^x 1 dt = x$.
$\|Au_0\|_\infty = \sup_{x \in [-1,1]} |x| = 1$.
Comme $\|Au_0\|_\infty = 1 \times \|u_0\|_\infty$, le sup est atteint et $\|A\| = 1$.
\end{proof}

\begin{proposition}
Soit $\|A\|_1 = \sup_{\|u\|_\infty \le 1} \int_{-1}^1 |(Au)(x)| dx$. Alors $\|A\|_1 = 1$.
\end{proposition}
\begin{proof}
Soit $\|A\|_1 = \sup_{\|u\|_\infty \le 1} \int_{-1}^1 |(Au)(x)| dx$.
\begin{align*} \int_{-1}^1 |(Au)(x)| dx &\le \int_{-1}^1 \|u\|_\infty |x| dx \\ &= \|u\|_\infty \int_{-1}^1 |x| dx \\ &= \|u\|_\infty \left( \int_{-1}^0 -x dx + \int_0^1 x dx \right) \\ &= \|u\|_\infty \left( \left[-\frac{x^2}{2}\right]_{-1}^0 + \left[\frac{x^2}{2}\right]_0^1 \right) \\ &= \|u\|_\infty \left( (0 - (-\frac{1}{2})) + (\frac{1}{2} - 0) \right) = \|u\|_\infty \times 1 \end{align*}
Donc $\|A\|_1 \le 1$.
Prenons $u_0(x) = 1$.
$\int_{-1}^1 |(Au_0)(x)| dx = \int_{-1}^1 |x| dx = 1$.
Donc $\|A\|_1 = 1$.
\end{proof}

Considérons la fonction $u(x) = 1 - |x|$ sur $[-1, 1]$.
\begin{verbatim}
#save_to: u_plot.png
import matplotlib.pyplot as plt
import matplotlib.patches as patches
import numpy as np

x = np.linspace(-1, 1, 400)
u = 1 - np.abs(x)

plt.figure(figsize=(5, 4))
plt.plot(x, u, label='$u(x) = 1 - |x|$')
plt.title('Fonction $u(x)$')
plt.xlabel('x')
plt.ylabel('u(x)')
plt.grid(True)
plt.ylim(0, 1.1)
plt.xlim(-1.1, 1.1)
plt.axhline(0, color='black',linewidth=0.5)
plt.axvline(0, color='black',linewidth=0.5)
plt.legend()
plt.savefig('u_plot.png')
\end{verbatim}

\begin{figure}[H]
\centering
\includegraphics[ max width=0.6\textwidth, max height=0.4\textheight, keepaspectratio]{u_plot.png}
\caption{Fonction $u(x) = 1 - |x|$.}
\label{fig:u_plot}
\end{figure}

Considérons une suite de fonctions $(f_n)$ dans $C^0([a,b], \mathbb{R})$.
Soit $f_n(x)$ la fonction "chapeau" centrée en $x_0$, de hauteur $C_n$ et de largeur $2/n$.
\[ f_n(x) = \begin{cases} 0 & \text{si } x \le x_0 - 1/n \\ C_n (1 - n|x-x_0|) & \text{si } x_0 - 1/n \le x \le x_0 + 1/n \\ 0 & \text{si } x \ge x_0 + 1/n \end{cases} \]
Supposons $a < x_0 - 1/n$ et $x_0 + 1/n < b$.
L'intégrale de $f_n$ est l'aire du triangle :
\[ \int_a^b f_n(x) dx = \frac{1}{2} \times \text{base} \times \text{hauteur} = \frac{1}{2} \times \frac{2}{n} \times C_n = \frac{C_n}{n} \]
Si on veut $\|f_n\|_1 = \int_a^b |f_n(x)| dx = 1$, il faut choisir $C_n = n$.
Dans ce cas, $f_n(x) = n(1 - n|x-x_0|)$ sur $[x_0 - 1/n, x_0 + 1/n]$.
La norme sup est $\|f_n\|_\infty = \max f_n(x) = f_n(x_0) = C_n = n$.
Donc on a $\|f_n\|_1 = 1$ mais $\|f_n\|_\infty = n \to \infty$.
Les normes $\|.\|_1$ et $\|.\|_\infty$ ne sont pas équivalentes sur $C^0([a,b])$.

\begin{verbatim}
#save_to: fn_plot.png
import matplotlib.pyplot as plt
import matplotlib.patches as patches
import numpy as np

def fn(x, x0, n, Cn):
    cond1 = x <= x0 - 1/n
    cond2 = (x >= x0 - 1/n) & (x <= x0 + 1/n)
    cond3 = x >= x0 + 1/n
    
    y = np.zeros_like(x)
    y[cond2] = Cn * (1 - n * np.abs(x[cond2] - x0))
    return y

x0 = 0.5
a = 0
b = 1
n = 5
Cn = n # pour que l'intégrale soit 1

x = np.linspace(a, b, 500)
y = fn(x, x0, n, Cn)

plt.figure(figsize=(6, 4))
plt.plot(x, y, label=f'$f_{n}(x)$ avec $n={n}, C_n={Cn}$')

# Zone hachurée pour l'intégrale
x_fill = np.linspace(x0 - 1/n, x0 + 1/n, 100)
y_fill = fn(x_fill, x0, n, Cn)
plt.fill_between(x_fill, y_fill, color='skyblue', alpha=0.4, label='Aire = $\int f_n(x) dx = C_n/n = 1$')

plt.axhline(0, color='black',linewidth=0.5)
plt.xlabel('x')
plt.ylabel('$f_n(x)$')
plt.title('Fonction chapeau $f_n(x)$')
plt.legend()
plt.grid(True)
plt.text(x0, -0.1*Cn, '$x_0$', ha='center')
plt.text(x0 - 1/n, -0.1*Cn, '$x_0 - 1/n$', ha='center')
plt.text(x0 + 1/n, -0.1*Cn, '$x_0 + 1/n$', ha='center')
plt.ylim(-0.2*Cn, 1.1*Cn)
plt.savefig('fn_plot.png')
\end{verbatim}

\begin{figure}[H]
\centering
\includegraphics[ max width=0.7\textwidth, max height=0.4\textheight, keepaspectratio]{fn_plot.png}
\caption{Fonction chapeau $f_n(x)$ avec $C_n=n$ pour que $\|f_n\|_1=1$.}
\label{fig:fn_plot}
\end{figure}

Considérons la preuve que si $\|f_n\|_1 \to 0$, alors $\|f_n\|_\infty$ ne tend pas forcément vers 0.
Soit $f_n(x)$ comme ci-dessus avec $C_n = \sqrt{n}$.
$\|f_n\|_1 = C_n/n = \sqrt{n}/n = 1/\sqrt{n} \to 0$.
$\|f_n\|_\infty = C_n = \sqrt{n} \to \infty$.

Considérons l'opérateur $m: C^0([a,b]) \to C^0([a,b])$ défini par $(m f)(x) = m(x) f(x)$ où $m(x)$ est une fonction continue donnée.
Soit $A = m$. $A$ est linéaire.
$\|(Af)(x)\| = |m(x) f(x)| = |m(x)| |f(x)|$.
$\|Af\|_\infty = \sup_x |m(x) f(x)| \le (\sup_x |m(x)|) (\sup_x |f(x)|) = \|m\|_\infty \|f\|_\infty$.
Donc $\|A\| \le \|m\|_\infty$.
Pour montrer l'égalité, supposons $m$ non nulle. Soit $x_0$ tel que $|m(x_0)| = \|m\|_\infty$.
Considérons une suite de fonctions $(f_n)$ "pic" centrées en $x_0$ telles que $f_n(x_0)=1$, $\|f_n\|_\infty=1$ et le support de $f_n$ se contracte vers $x_0$.
Par exemple, $f_n(x) = \max(0, 1 - n|x-x_0|)$.
\begin{verbatim}
#save_to: fn_proof_plot.png
import matplotlib.pyplot as plt
import matplotlib.patches as patches
import numpy as np

def fn(x, x0, n):
    return np.maximum(0, 1 - n * np.abs(x - x0))

x0 = 0.5
a = 0
b = 1
n = 10 # Exemple avec n=10

x = np.linspace(a, b, 500)
y = fn(x, x0, n)

plt.figure(figsize=(6, 4))
plt.plot(x, y, label=f'$f_{n}(x)$ pic en $x_0={x0}$')

plt.axhline(0, color='black',linewidth=0.5)
plt.xlabel('x')
plt.ylabel('$f_n(x)$')
plt.title(f'Fonction pic $f_n(x)$ pour la preuve ($n={n}$)')
plt.legend()
plt.grid(True)
plt.text(x0, -0.1, '$x_0$', ha='center')
plt.text(x0 - 1/n, -0.1, '$x_0 - 1/n$', ha='center')
plt.text(x0 + 1/n, -0.1, '$x_0 + 1/n$', ha='center')
plt.ylim(-0.2, 1.2)
plt.savefig('fn_proof_plot.png')
\end{verbatim}

\begin{figure}[H]
\centering
\includegraphics[ max width=0.7\textwidth, max height=0.4\textheight, keepaspectratio]{fn_proof_plot.png}
\caption{Fonction pic $f_n(x)$ utilisée dans la preuve.}
\label{fig:fn_proof_plot}
\end{figure}

$|(Af_n)(x)| = |m(x) f_n(x)|$.
$\|Af_n\|_\infty = \sup_x |m(x) f_n(x)|$.
Comme $f_n$ est concentrée autour de $x_0$, et $f_n(x_0)=1$, $\|Af_n\|_\infty$ sera proche de $|m(x_0)| \|f_n\|_\infty = \|m\|_\infty$.
Plus formellement :
Soit $\epsilon > 0$. Par continuité de $m$, il existe $\delta > 0$ tel que si $|x-x_0| < \delta$, alors $|m(x) - m(x_0)| < \epsilon$.
Choisissons $n$ assez grand pour que $1/n < \delta$. Alors le support de $f_n$ est dans $[x_0 - \delta, x_0 + \delta]$.
Pour $x$ dans le support de $f_n$, on a $|m(x)| \ge |m(x_0)| - |m(x) - m(x_0)| > \|m\|_\infty - \epsilon$.
$\|Af_n\|_\infty = \sup_{|x-x_0|\le 1/n} |m(x) f_n(x)|$.
Puisque $f_n(x_0)=1$, $\|Af_n\|_\infty \ge |m(x_0)f_n(x_0)| = |m(x_0)| = \|m\|_\infty$.
D'autre part, $\|Af_n\|_\infty = \sup_{|x-x_0|\le 1/n} |m(x) f_n(x)| \le \sup_{|x-x_0|\le 1/n} |m(x)| \times \|f_n\|_\infty$.
Comme $\|f_n\|_\infty = 1$, $\|Af_n\|_\infty \le \sup_{|x-x_0|\le 1/n} |m(x)|$.
Par continuité, $\lim_{n\to\infty} \sup_{|x-x_0|\le 1/n} |m(x)| = |m(x_0)| = \|m\|_\infty$.
Donc $\lim_{n\to\infty} \|Af_n\|_\infty = \|m\|_\infty$.
Puisque $\|Af_n\|_\infty \le \|A\| \|f_n\|_\infty = \|A\|$, en passant à la limite, on obtient $\|m\|_\infty \le \|A\|$.
Comme on avait déjà $\|A\| \le \|m\|_\infty$, on conclut que $\|A\| = \|m\|_\infty$.

\subsection{Normes équivalentes}

\begin{definition}[Normes topologiquement équivalentes]
Soit $E$ un espace vectoriel. Soit $N_1$ et $N_2$ deux normes sur $E$.
On dit que $N_1$ et $N_2$ sont topologiquement équivalentes si $(E, N_1)$ et $(E, N_2)$ ont les mêmes ensembles ouverts.
\end{definition}

\begin{definition}[Normes équivalentes]
Soit $E$ un espace vectoriel. Soit $N_1$ et $N_2$ deux normes sur $E$.
On dit que $N_1$ et $N_2$ sont équivalentes (on écrit $N_1 \sim N_2$) s'il existe $C_1, C_2 > 0$ telles que
    \[ \forall X \in E, \quad N_1(X) \le C_1 N_2(X) \quad \text{et} \quad N_2(X) \le C_2 N_1(X) \]
    Ceci peut se réécrire : il existe $C > 0$ tel que
    \[ \forall X \in E, \quad C^{-1} N_2(X) \le N_1(X) \le C N_2(X) \]
\end{definition}

\begin{theorem}[Equivalence topologique et équivalence des normes]
Deux normes sur $E$ sont topologiquement équivalentes si et seulement si elles sont équivalentes.
\end{theorem}
\begin{proof} (Esquisse basée sur 6.6.3 du textbook)
Soit $Id: (E, N_1) \to (E, N_2)$ et $Id: (E, N_2) \to (E, N_1)$.
Les deux topologies sont les mêmes si et seulement si ces deux applications identité sont continues.
Une application linéaire est continue si et seulement si elle est bornée (Théorème 6.14).
$Id: (E, N_1) \to (E, N_2)$ est continue $\iff$ elle est bornée $\iff \exists C_1 > 0$ tel que $\|Id(x)\|_{N_2} \le C_1 \|x\|_{N_1}$, i.e., $N_2(x) \le C_1 N_1(x)$.
$Id: (E, N_2) \to (E, N_1)$ est continue $\iff$ elle est bornée $\iff \exists C_2 > 0$ tel que $\|Id(x)\|_{N_1} \le C_2 \|x\|_{N_2}$, i.e., $N_1(x) \le C_2 N_2(x)$.
Ces deux conditions réunies correspondent à la définition de normes équivalentes.
\end{proof}

\begin{theorem}[\textbf{Admission - Equivalence des normes en dimension finie}]
Soit $E$ un espace vectoriel sur $\mathbb{K}$ ($\mathbb{R}$ ou $\mathbb{C}$) de dimension finie. Alors toutes les normes sur $E$ sont équivalentes.
\end{theorem}
\begin{proof} (Esquisse basée sur Thm 6.9 du textbook)
On peut supposer $\mathbb{K} = \mathbb{R}$ (en identifiant $\mathbb{C}$ à $\mathbb{R}^2$).
Soit $(e_1, ..., e_n)$ une base de $E$. On identifie $E$ à $\mathbb{R}^n$.
La norme $\|x\|_\infty = \max_{1\le i \le n} |x_i|$ est une norme sur $\mathbb{R}^n$.
Soit $N$ une norme quelconque sur $\mathbb{R}^n$. Montrons que $N$ est équivalente à $\|.\|_\infty$.
Pour $x = \sum x_i e_i$, on a
$N(x) = N(\sum x_i e_i) \le \sum |x_i| N(e_i) \le (\sum N(e_i)) \max |x_i| = C \|x\|_\infty$ avec $C = \sum N(e_i)$.
Montrons qu'il existe $a > 0$ tel que $a \|x\|_\infty \le N(x)$.
La fonction $N: (\mathbb{R}^n, \|.\|_\infty) \to \mathbb{R}$ est continue.
En effet, $|N(x) - N(y)| \le N(x-y) \le C \|x-y\|_\infty$.
Soit $S = \{ x \in \mathbb{R}^n \mid \|x\|_\infty = 1 \}$. $S$ est la sphère unité pour $\|.\|_\infty$.
$S$ est fermée (car l'application $x \mapsto \|x\|_\infty$ est continue) et bornée.
Par le théorème de Borel-Lebesgue (Thm 3.36), $S$ est compacte.
La fonction continue $N$ atteint ses bornes sur le compact $S$.
Comme $N(x) > 0$ pour $x \ne 0$ (donc pour $x \in S$), le minimum de $N$ sur $S$ est $a > 0$.
Donc $\forall x \in S$, $N(x) \ge a$.
Pour tout $x \ne 0$, le vecteur $y = x / \|x\|_\infty$ est dans $S$.
Donc $N(y) \ge a$.
$N(x / \|x\|_\infty) \ge a \implies \frac{1}{\|x\|_\infty} N(x) \ge a \implies N(x) \ge a \|x\|_\infty$.
On a donc montré $a \|x\|_\infty \le N(x) \le C \|x\|_\infty$. $N$ est équivalente à $\|.\|_\infty$.
Comme l'équivalence est une relation d'équivalence, toutes les normes sont équivalentes entre elles.
\end{proof}

\section{Applications Linéaires et Bornées}

\begin{definition}[Application linéaire bornée]
Soient $(E, \|.\|_E)$ et $(F, \|.\|_F)$ deux espaces vectoriels normés. Une application linéaire $A: E \to F$ est dite \textbf{bornée} (ou continue) s'il existe une constante $C \ge 0$ telle que
\[ \forall x \in E, \quad \|Ax\|_F \le C \|x\|_E \]
\end{definition}

\begin{proposition}
Pour une application linéaire $A: E \to F$, les propriétés suivantes sont équivalentes :
\begin{enumerate}
    \item $A$ est continue.
    \item $A$ est continue en $0_E$.
    \item $A$ est bornée.
\end{enumerate}
\end{proposition}

\begin{definition}[Norme d'opérateur]
Si $A: E \to F$ est une application linéaire bornée, on définit sa \textbf{norme} (appelée norme d'opérateur ou norme uniforme) par :
\[ \|A\| = \sup_{x \in E, x \ne 0_E} \frac{\|Ax\|_F}{\|x\|_E} = \sup_{x \in E, \|x\|_E = 1} \|Ax\|_F = \sup_{x \in E, \|x\|_E \le 1} \|Ax\|_F \]
C'est la plus petite constante $C$ telle que $\|Ax\|_F \le C \|x\|_E$ pour tout $x \in E$.
\end{definition}

\begin{proposition}
Soit $A \in B(E, F)$ (l'espace des applications linéaires bornées de E dans F).
\begin{enumerate}
    \item $\|.\|$ est une norme sur $B(E, F)$.
    \item On a $\|Ax\|_F \le \|A\| \|x\|_E$ pour tout $x \in E$.
    \item $\|A\|$ est la plus petite constante $C$ telle que $\|Ax\|_F \le C \|x\|_E$.
\end{enumerate}
\end{proposition}

\begin{proposition}
Si $A \in B(E, F)$ et $B \in B(F, G)$, alors $B \circ A \in B(E, G)$ et
\[ \|B \circ A\| \le \|B\| \|A\| \]
Si $E=F=G$, on note $B(E) = B(E,E)$, alors pour $A, B \in B(E)$,
\[ \|AB\| \le \|A\| \|B\| \]
\end{proposition}
\begin{proof}
$\| (B \circ A) x \|_G = \| B(Ax) \|_G \le \|B\| \|Ax\|_F \le \|B\| (\|A\| \|x\|_E) = (\|B\| \|A\|) \|x\|_E$.
Donc $\|B \circ A\| \le \|B\| \|A\|$.
\end{proof}

\section{Le cas des matrices}

On identifie une matrice $A \in M_n(\mathbb{C})$ (ou $M_n(\mathbb{R})$) avec l'application linéaire associée $A: \mathbb{K}^n \to \mathbb{K}^n$.
L'espace vectoriel $M_n(\mathbb{K})$ est de dimension finie $n^2$. Toutes les normes y sont donc équivalentes.
La norme la plus utile est la norme uniforme (norme d'opérateur) obtenue à partir de la norme euclidienne $\|.\|_2$ sur $\mathbb{K}^n$.
\[ \|A\| = \sup_{x \in \mathbb{K}^n, \|x\|_2 = 1} \|Ax\|_2 \]

\begin{definition}[Matrice adjointe]
Soit $A = [a_{ij}] \in M_n(\mathbb{C})$. La matrice \textbf{adjointe} de $A$, notée $A^*$, est la matrice $B = [b_{ij}]$ telle que $b_{ij} = \overline{a_{ji}}$. (Transposée conjuguée).
On a la propriété : $\forall x, y \in \mathbb{C}^n, \quad (Ax | y) = (x | A^*y)$, où $(u|v) = \sum_{i=1}^n u_i \overline{v_i}$ est le produit scalaire hermitien standard.
\end{definition}

\begin{definition}[Matrice autoadjointe]
$A$ est dite \textbf{autoadjointe} (ou hermitienne) si $A = A^*$.
\end{definition}

\begin{lemma}
Pour $A \in M_n(\mathbb{C})$ :
\[ \|A^*\| = \|A\| \quad \text{et} \quad \|A^*A\| = \|A\|^2 \]
\end{lemma}
\begin{proof}
On utilise $\|A\| = \sup_{\|x\|=1, \|y\|=1} |(Ax|y)|$.
$|(Ax|y)| = |(x|A^*y)|$. Donc $\|A\| = \sup_{\|x\|=1, \|y\|=1} |(x|A^*y)| = \|A^*\|$.
Pour la seconde égalité :
$\|Ax\|_2^2 = (Ax|Ax) = (x|A^*Ax)$.
Donc $\|A\|^2 = \sup_{\|x\|=1} \|Ax\|_2^2 = \sup_{\|x\|=1} (x|A^*Ax)$.
La matrice $B = A^*A$ est autoadjointe. Pour une matrice autoadjointe $B$, on sait que $\sup_{\|x\|=1} (x|Bx)$ est égal à la plus grande valeur propre de $B$. (Ceci est lié au quotient de Rayleigh).
D'autre part, $\|A^*A\| = \sup_{\|x\|=1} \|(A^*A)x\|_2$.
Comme $A^*A$ est autoadjointe, $\|A^*A\|$ est égal au maximum du module de ses valeurs propres.
Les valeurs propres de $A^*A$ sont réelles et positives ou nulles. Soit $\lambda_{\max}$ la plus grande valeur propre.
Alors $\|A\|^2 = \sup_{\|x\|=1} (x|A^*Ax) = \lambda_{\max}(A^*A)$.
Et $\|A^*A\| = \max |\lambda_i(A^*A)| = \lambda_{\max}(A^*A)$ car $\lambda_i \ge 0$.
Donc $\|A\|^2 = \|A^*A\|$.
\end{proof}

\begin{theorem}[Calcul de la norme matricielle]
Soit $A \in M_n(\mathbb{C})$. Soient $\lambda_1, ..., \lambda_n$ les valeurs propres de la matrice autoadjointe positive $A^*A$. Alors
\[ \|A\| = \sqrt{\max_{1 \le i \le n} \lambda_i} \]
Les racines carrées des valeurs propres de $A^*A$ sont appelées les valeurs singulières de $A$. Donc $\|A\|$ est la plus grande valeur singulière de $A$.
\end{theorem}
\begin{proof}
Comme $\|A\|^2 = \|A^*A\|$ et $A^*A$ est autoadjointe, sa norme $\|A^*A\|$ est égale au maximum du module de ses valeurs propres (qui sont réelles et $\ge 0$).
Donc $\|A\|^2 = \max \lambda_i (A^*A)$.
D'où $\|A\| = \sqrt{\max \lambda_i}$.
\proofname{Démonstration}
\end{proof}

\begin{lemma}[Inégalité de Cauchy-Schwarz pour la norme matricielle]
Soit $A \in M_n(\mathbb{C})$.
$\|Ax\|_2^2 = (x|A^*Ax)$.
Comme $A^*A$ est autoadjointe, elle admet une base orthonormée $(v_1, ..., v_n)$ de vecteurs propres avec les valeurs propres réelles $\lambda_1, ..., \lambda_n \ge 0$.
Soit $x = \sum_{i=1}^n c_i v_i$. Alors $\|x\|_2^2 = \sum |c_i|^2$.
$A^*Ax = A^*A (\sum c_i v_i) = \sum c_i (A^*A v_i) = \sum c_i \lambda_i v_i$.
$(x|A^*Ax) = (\sum c_j v_j | \sum c_i \lambda_i v_i) = \sum_{i,j} \overline{c_j} c_i \lambda_i (v_j|v_i) = \sum_{i=1}^n \lambda_i |c_i|^2$.
Donc $\|Ax\|_2^2 = \sum_{i=1}^n \lambda_i |c_i|^2$.
On a $\|Ax\|_2^2 = \sum \lambda_i |c_i|^2 \le (\max \lambda_j) \sum |c_i|^2 = (\max \lambda_j) \|x\|_2^2$.
$\max \lambda_j = \|A\|^2$.
Donc $\|Ax\|_2^2 \le \|A\|^2 \|x\|_2^2$, ce qui redonne $\|Ax\|_2 \le \|A\| \|x\|_2$.
\end{lemma}

\begin{definition}[Norme de Hilbert-Schmidt]
\[ \|A\|_{HS} = \left( \sum_{i=1}^n \sum_{j=1}^n |a_{ij}|^2 \right)^{1/2} = \sqrt{Tr(A^*A)} \]
où $Tr$ est la trace de la matrice.
\end{definition}

\begin{proposition}
On a $\|A\| \le \|A\|_{HS}$.
\end{proposition}
\begin{proof}
Par Cauchy-Schwarz sur $\mathbb{K}^n$:
$\|Ax\|_2^2 = \sum_{i=1}^n \left| \sum_{j=1}^n a_{ij} x_j \right|^2 \le \sum_{i=1}^n \left( \sum_{j=1}^n |a_{ij}|^2 \right) \left( \sum_{k=1}^n |x_k|^2 \right)$
$\|Ax\|_2^2 \le \left( \sum_{i=1}^n \sum_{j=1}^n |a_{ij}|^2 \right) \|x\|_2^2 = \|A\|_{HS}^2 \|x\|_2^2$.
Donc $\|A\| = \sup_{\|x\|=1} \|Ax\|_2 \le \|A\|_{HS}$.
\end{proof}

\end{document}
```