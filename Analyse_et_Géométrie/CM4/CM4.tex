```latex
\documentclass{article}
\usepackage{amsmath}
\usepackage{amsfonts}
\usepackage{amssymb}
\usepackage{amsthm}
\usepackage{graphicx}
\usepackage{listings}
\usepackage{verbatim}

\newtheorem{theorem}{Theorem}
\newtheorem{lemma}{Lemma}
\newtheorem{proposition}{Proposition}
\newtheorem{definition}{Definition}
\newtheorem{remark}{Remark}
\newtheorem{solution}{Solution}
\newtheorem{example}{Example}

\begin{document}
\sloppy

\section{Suites de Cauchy et Complétude}

\begin{definition}[Suite de Cauchy]
Une suite $(x_n)_{n \in \mathbb{N}}$ dans un espace métrique $E$ est dite suite de Cauchy si pour tout $\varepsilon > 0$, il existe $N \in \mathbb{N}$ tel que pour tous $n, p \geq N$, on a $d(x_p, x_n) \leq \varepsilon$.
\end{definition}

\begin{proposition}
Toute suite convergente est une suite de Cauchy.
\end{proposition}
\begin{proof}
Supposons que $\lim_{n \to \infty} x_n = x$. Pour $\varepsilon > 0$, on peut trouver $N \in \mathbb{N}$ tel que $d(x, x_n) \leq \varepsilon/2$ pour tout $n \geq N$. Alors pour tous $n, p \geq N$, on a
\begin{align*}
d(x_n, x_p) &\leq d(x_n, x) + d(x, x_p) \\
&\leq \varepsilon/2 + \varepsilon/2 = \varepsilon.
\end{align*}
Ainsi $(x_n)_{n \in \mathbb{N}}$ est une suite de Cauchy.
\end{proof}

\begin{proposition}
Toute suite de Cauchy est bornée.
\end{proposition}
\begin{proof}
Soit $(x_n)_{n \in \mathbb{N}}$ une suite de Cauchy. Par définition (en prenant $\varepsilon = 1$), il existe $N$ tel que $d(x_n, x_p) \leq 1$ pour $n, p \geq N$. En particulier $d(x_n, x_N) \leq 1$ pour $n \geq N$. On a donc pour tout $n \in \mathbb{N}$,
$$d(x_n, x_N) \leq \max( \{d(x_1, x_N), \dots, d(x_{N-1}, x_N)\} \cup \{1\}) =: r_0.$$
Ainsi $x_n \in B(x_N, r_0)$ pour tout $n \in \mathbb{N}$.
\end{proof}

\begin{definition}[Espace complet]
Un espace métrique $(E, d)$ est dit complet si toute suite de Cauchy dans $E$ est convergente.
\end{definition}

\begin{theorem}
$\mathbb{R}^d$ muni de la distance canonique est complet.
\end{theorem}

\section{Intérieur et Adhérence}

\begin{definition}[Intérieur]
Soit $A \subset E$. Un point $x \in E$ est intérieur à $A$ s'il existe $\delta > 0$ tel que $B(x, \delta) \subset A$. L'ensemble des points intérieurs à $A$ se note $\mathrm{Int}(A)$ et s'appelle l'intérieur de $A$.
\end{definition}

\begin{proposition}
$\mathrm{Int}(A)$ est le plus grand ouvert inclus dans $A$, ou de manière équivalente la réunion de tous les ouverts inclus dans $A$.
\end{proposition}

\begin{definition}[Adhérence]
Soit $A \subset E$. Un point $x \in E$ est adhérent à $A$ si $B(x, r) \cap A \neq \emptyset$ pour tout $r > 0$. L'ensemble des points adhérents à $A$ se note $\mathrm{Adh}(A)$ et s'appelle l'adhérence ou la fermeture de $A$.
\end{definition}

\begin{proposition}
$\mathrm{Adh}(A)$ est le plus petit fermé contenant $A$, ou de manière équivalente l'intersection de tous les fermés contenant $A$.
\end{proposition}

\begin{proposition}
$x \in \mathrm{Adh}(A)$ si et seulement s'il existe une suite $(x_n)_{n \in \mathbb{N}}$ d'éléments de $A$ telle que $x = \lim_{n \to \infty} x_n$.
\end{proposition}

\begin{example}
Soit $A = \{(x,y) \in \mathbb{R}^2 : 2x+3y < 4\}$. Déterminer $\mathrm{Int}(A)$ et $\mathrm{Adh}(A)$.
\begin{itemize}
    \item $\mathrm{Int}(A) = A = \{(x,y) \in \mathbb{R}^2 : 2x+3y < 4\}$. $A$ est ouvert.
    \item $\mathrm{Adh}(A) = C = \{(x,y) \in \mathbb{R}^2 : 2x+3y \leq 4\}$. $C$ est fermé et contient $A$.
\end{itemize}
\end{example}

\begin{example}
Soit $A = \{(x,y) \in \mathbb{R}^2 : x > 0, y = \sin(1/x)\}$. Déterminer $\mathrm{Adh}(A)$ et $\mathrm{Int}(A)$.
\begin{itemize}
    \item $\mathrm{Int}(A) = \emptyset$. Car $\mathrm{Int}(A)$ est un ouvert inclus dans $A$. Or $A$ ne contient aucune boule ouverte.
    \item $\mathrm{Adh}(A) = A \cup \{(0,y) : y \in [-1, 1]\}$.
\end{itemize}
\end{example}

\section{Exercices Résolus}

\begin{example}
Soit $A = \{(x,y) \in \mathbb{R}^2 : |x| < 1, |y| < 1\}$. Déterminer $\mathrm{Int}(A)$ et $\mathrm{Adh}(A)$.
\end{example}

\begin{solution}
\begin{enumerate}
    \item On dessine $A$, c'est un carré ouvert.
    \item On pense que $\mathrm{Int}(A) = B = A = \{(x,y) \in \mathbb{R}^2 : |x| < 1, |y| < 1\}$ et $\mathrm{Adh}(A) = C = \{(x,y) \in \mathbb{R}^2 : |x| \leq 1, |y| \leq 1\}$.
    \item Montrons que $B = \mathrm{Int}(A)$.
    \begin{itemize}
        \item $B$ est ouvert et $B \subset A$. Vrai par définition de $B$.
        \item Soit $X \in A \setminus B = \emptyset$. Donc il n'y a pas de points de $A$ qui ne sont pas dans $B$. Ainsi $B = \mathrm{Int}(A)$.
    \end{itemize}
    \item Montrons que $C = \mathrm{Adh}(A)$.
    \begin{itemize}
        \item $C$ est fermé et $A \subset C$. Vrai par définition de $C$.
        \item Montrons que $C \subset \mathrm{Adh}(A)$. Pour chaque $X \in C$, on cherche une suite $(X_n)$ avec $X_n \in A$ et $\lim X_n = X$. Soit $X = (x, y) \in C$, i.e., $|x| \leq 1, |y| \leq 1$. On prend $X_n = (x - 1/n, y - 1/n)$ (si $x = 1$, on prend $x - 1/n$, similarly for $y$). Plus précisément, soit $X_n = (x_n, y_n)$ avec $x_n = x - \frac{1}{n} \text{sign}(x)$ si $x \neq 0$ et $x_n = -1/n$ si $x = 0$, et $y_n = y - \frac{1}{n} \text{sign}(y)$ si $y \neq 0$ et $y_n = -1/n$ si $y = 0$. Alors $X_n \in A$ et $\lim X_n = X$.
    \end{itemize}
\end{enumerate}
\end{solution}

\begin{example}
Soit $A = \{(x,y) \in \mathbb{R}^2 : x > 0, y = \sin(1/x)\}$. Déterminer $\mathrm{Adh}(A)$ et $\mathrm{Int}(A)$.
\end{example}

\begin{solution}
\begin{enumerate}
    \item On dessine $A$. C'est le graphe de $\sin(1/x)$ pour $x > 0$.
    \item On pense que $\mathrm{Int}(A) = \emptyset$ et $\mathrm{Adh}(A) = A \cup \{(0,y) : y \in [-1, 1]\}$. Soit $C = A \cup \{(0,y) : y \in [-1, 1]\}$.
    \item Montrons que $\mathrm{Int}(A) = \emptyset$. Si $\mathrm{Int}(A) \neq \emptyset$, alors $\mathrm{Int}(A)$ est un ouvert non vide inclus dans $A$. Donc $\mathrm{Int}(A)$ contient une boule $B(X_0, r) \subset A$. Mais $A$ est le graphe d'une fonction, il n'y a pas de boule dans $A$. Donc $\mathrm{Int}(A) = \emptyset$.
    \item Montrons que $\mathrm{Adh}(A) = C = A \cup \{(0,y) : y \in [-1, 1]\}$.
    \begin{itemize}
        \item $C$ est fermé et $A \subset C$. $A$ n'est pas fermé. $C$ est fermé, car si $(X_n) \in C$ et $X_n \to X$, alors $X \in C$. Si $X_n = (x_n, y_n) \in A$, alors $x_n > 0, y_n = \sin(1/x_n)$. Si $X_n \to X = (x, y)$, alors $x_n \to x, y_n \to y$. Si $x > 0$, alors $X \in A \subset C$. Si $x = 0$, on ne peut pas dire que $y = \sin(1/x)$. Mais on sait que $-1 \leq \sin(1/x_n) \leq 1$, donc $-1 \leq y_n \leq 1$, donc $-1 \leq y \leq 1$. Donc si $x = 0$, $X = (0, y)$ avec $y \in [-1, 1]$, donc $X \in C$.
        \item Montrons que $C \subset \mathrm{Adh}(A)$. Pour $X \in C$, si $X \in A$, alors $X \in \mathrm{Adh}(A)$. Si $X \in C \setminus A = \{(0,y) : y \in [-1, 1]\}$, i.e., $X = (0, y)$ avec $y \in [-1, 1]$. On doit montrer que $X \in \mathrm{Adh}(A)$. On cherche une suite $X_n \in A$ avec $X_n \to X$. On prend $X_n = (\frac{1}{n \pi + \arcsin(y)}, \sin(n \pi + \arcsin(y))) = (\frac{1}{n \pi + \arcsin(y)}, y)$. Alors $X_n \in A$ et $X_n \to (0, y) = X$. Donc $X \in \mathrm{Adh}(A)$.
    \end{itemize}
\end{enumerate}
\end{solution}

\end{document}
```